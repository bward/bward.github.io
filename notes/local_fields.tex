\documentclass[a4paper]{article}
\title{Part III Local Fields}
\author{Based on lectures by Dr C. Johansson}
\date{Michaelmas 2016\\University of Cambridge}
\usepackage{mathtools}
\usepackage{amsthm}
\usepackage{amssymb}
\usepackage{textcomp}
\usepackage{enumitem}
\usepackage{graphicx}
\usepackage{tikz-cd}
\usepackage{enumitem}
\newtheorem*{definition}{Definition}
\newtheorem{lemma}{Lemma}
\newtheorem{theorem}[lemma]{Theorem}
\newtheorem{prop}[lemma]{Proposition}
\newtheorem{corollary}[lemma]{Corollary}
\renewcommand{\baselinestretch}{1.3}
\newcommand*\conj[1]{\overline{#1}}
\newcommand*\dom[1]{\textnormal{dom}\,#1}
\newcommand*\cod[1]{\textnormal{cod}\,#1}
\newcommand*\ob[1]{\textnormal{ob}\,#1}
\newcommand*\mor[1]{\textnormal{mor}\,#1}
\newcommand*\abs[1]{\left|#1\right|}
\begin{document}
\maketitle
\tableofcontents

\section{Basic Theory}
\begin{definition}[Absolute value]
	Let $K$ be a field. An \textbf{absolute value} on $K$ is a function $\abs{\cdot}:K\to\mathbb{R}_{\geq 0}$ s.t.
	\begin{enumerate}[label=\roman*.]
		\item $\abs{x}=0 \iff x=0$
		\item $\abs{xy}=\abs{x}\abs{y} \quad \forall x,y \in K$
		\item $\abs{x+y} \leq \abs{x} + \abs{y}$
	\end{enumerate}
\end{definition}

\begin{definition}[Valued field]
	A \textbf{valued field} is a field with an absolute value.
\end{definition}

\begin{definition}[Equivalence of absolute values]
	Let $K$ be a field and let $\abs{\cdot}$, $\abs{\cdot}^{'}$ be absolute values on K. We say that $\abs{\cdot}$ and $\abs{\cdot}^{'}$ are \textbf{equivalent} if the associated metrics induce the same topology.
\end{definition}

\begin{definition}[Non-archimedean absolute value]
	An absolute value $\abs{\cdot}$ on a field $K$ is called \textbf{non-archimedean} if $\abs{x+y}\leq\max(\abs{x},\abs{y})$ (the \textbf{strong triangle inequality}).
	
	Metrics s.t. $d(x,z)\leq\max(d(x,y),d(y,z))$ are called \textbf{ultrametrics}.
\end{definition}

Assumption: unless otherwise mentioned, all absolute values will be non-archimedean. These metrics are weird!

\begin{prop}
	Let $K$ be a valued field. Then $\mathcal{O}=\{x \,|\, \abs{x}\leq 1\}$ is an open subring of $K$, called the \textbf{valuation ring} of K. $\forall r \in (0,1]$, $\{x \,|\, x < r\}$ and $\{x \,|\, x \leq r\}$ are open ideals of $\mathcal{O}$.
	
	Moreover, $\mathcal{O}^x = \{x \,|\, \abs{x}=1\}$.
\end{prop}

\begin{prop}
	Let $K$ be a valued field.
	\begin{enumerate}[label=\roman*.]
		\item Let $(x_n)$ be a sequence in $K$. If $x_n - x_{n+1} \to 0$ then $(x_n)$ is Cauchy
	\end{enumerate}
	Assume that $K$ is complete
	\begin{enumerate}[label=\roman*.]
		\setcounter{enumi}{1}
		\item Let $(x_n)$ be a sequence in $K$. If $x_n - x_{n+1} \to 0$ then $(x_n)$ converges
		\item Let $\sum_{n=0}^{\infty}y_n$ be a series in $K$. If $y_n\to0$, then $\sum_{n=0}^{\infty}y_n$ converges
	\end{enumerate}
\end{prop}

\begin{definition}
	Let $R \subseteq S$ be rings. Then $s \in S$ is \textbf{integral over R} if $\exists$ monic $f(x) \in R[x]$ s.t. $f(s)=0$.
\end{definition}

\begin{prop}
	Let $R \subseteq S$ be rings. Then $s_1, \dots,s_n\in S$ are all integral over $R \iff R[s_1,\dots,s_n] \subseteq S$ is a finitely generated $R$-module.
\end{prop}

\begin{corollary}
	let $R \subseteq S$ be rings. If $s_1, s_2 \in S$ are integral over $R$, then $s_1+s_2$ and $s_1s_2$ are integral over $R$. In particular, the set $\tilde{R}\subseteq S$ of all elements in $S$ integral over $R$ is a ring, called the \textbf{integral closure} of $R$ in $S$.
\end{corollary}

\begin{definition}
	Let $R$ be a ring. A topology on $R$ is called a \textbf{ring topology} on $R$ if addition and multiplication are continuous maps $R\times R\to R$. A ring with a ring topology is called a \textbf{topological ring}.
\end{definition}

\begin{definition}
	Let $R$ be a ring, $I \subseteq R$ an ideal. A subset $U \subseteq R$ is called \linebreak \textbf{I-adically open} if $\forall x \in U$ $\exists n \geq 1$ s.t. $x + I^n \subseteq U$.
\end{definition}

\begin{prop}
	The set of all $I$-adically open sets form a topology on $R$, called the \textbf{I-adic topology}.
\end{prop}

\begin{definition}
	Let $R_1, R_2, \dots$ be topological rings with continuous homomorphisms $f_n:R_{n+1}\to R_n$ $\forall n \geq 1$. The \textbf{inverse limit} of the $R_i$ is the ring
	\begin{align*}
		\varprojlim_n R_n &= \left\{(x_n) \in \prod_n R_n \,|\, f_n(x_{n+1})=x_n \forall n \geq 1  \right\}\\
		&\subseteq \prod_n R_n
	\end{align*}
\end{definition}

\begin{prop}
	The inverse limit topology is a ring topology.
\end{prop}

\begin{definition}
	Let $R$ be a ring, $I$ an ideal. The \textbf{I-adic completion} of $R$ is the topological ring $\varprojlim_n R/I^n$ ($R/I^n$ has the discrete topology, and $R/I^{n+1} \to R/I^{n}$ is the natural map).
	
	There exists a map $\nu: R \to \varprojlim R/I^n$, $r \mapsto (r \mod I^n)_n$ This map is a continuous ring homomorphism when $R$ is given the $I$-adic topology. We say that $R$ is \textbf{I-adically complete} if $\nu$ is a bijection.
	
	If $I=xR$ then we often call the $I$-adic topology the \textbf{x-adic topology}.
\end{definition}

\section{The p-adic Numbers}
Let $p$ be a prime number throughout.

If $x \in \mathbb{Q} \backslash \{0\}$ then $\exists!$ representation $x=p^n \frac{a}{b}$, where $n \in \mathbb{Z}$, $a \in \mathbb{Z}$, $b \in \mathbb{Z}_{>0}$ and $(a,p)=(b,p)=(a,b)=1$.

We define the \textbf{p-adic absolute value} on $\mathbb{Q}$ to be the function $\abs{\cdot}_p: \mathbb{Q} \to \mathbb{R}_{\geq 0}$ given by
\begin{align*}
	\abs{x}_p =
	\begin{cases}
	0 &\text{if } x=0\\
	p^{-n}  &\text{if } x=p^n \frac{a}{b}\ (\neq 0) \text{ as before}
	\end{cases}
\end{align*}
Then $\abs{\cdot}_p$ is an absolute value.

\begin{definition}
	The \textbf{p-adic numbers} $\mathbb{Q}_p$ are the completion of $\mathbb{Q}$ w.r.t. $\abs{\cdot}_p$.
	
	The valuation ring $\mathbb{Z}_p = \{x \in \mathbb{Q}_p \,|\, \abs{x}_p \leq 1 \}$ is called the \textbf{p-adic integers}.
\end{definition}

\begin{prop}
	$\mathbb{Z}_p$ is the closure of $\mathbb{Z}$ inside $\mathbb{Q}_p$.
\end{prop}

\begin{prop}
	The non-zero ideals of $\mathbb{Z}_p$ are $p_n\mathbb{Z}_p$ for $n \geq 0$. Moreover, $\mathbb{Z}/p^n\mathbb{Z} \cong \mathbb{Z}_p/p^n\mathbb{Z}_p$
\end{prop}

\begin{corollary}
	$\mathbb{Z}_p$ is a PID with a unique prime element $p$ (up to units).
\end{corollary}

\begin{prop}
	The topology on $\mathbb{Z}$ induced by $\abs{\cdot}_p$ is the $p$-adic topology.
\end{prop}

\begin{prop}
	$\mathbb{Z}_p$ is $p$-adically complete and is (isomorphic to) the $p$-adic completion of $\mathbb{Z}$.
\end{prop}

\begin{corollary}
	Every $a \in \mathbb{Z}_p$ has a unique expansion $$a = \sum_{i=0}^\infty a_ip^i$$ with $a_i \in \{0, 1, \dots, p-1\}$
	
	Every $a \in \mathbb{Q}_p^\times$ has a unique expansion $$a = \sum_{i=n}{\infty}a_ip^i$$ $n \in \mathbb{Z}$, $n = -\log_p\abs{a}_p$, $a_n \neq 0$.
\end{corollary}

\section{Valued Fields}
\begin{definition}
	Let $K$ be a field. A \textbf{valuation} on K is a function $v : K \to \mathbb{R} \cup \{\infty\}$ s.t.
	\begin{enumerate}[label=\roman*.]
		\item $v(x) = \infty \iff x = 0$
		\item $v(xy) = v(x)+v(y)$
		\item $v(x+y) \geq \min(v(x), v(y))$
	\end{enumerate}
	$\forall x, y \in K$.
\end{definition}

Here we use the conventions $r+\infty = \infty$, $r \leq \infty\ \forall r \in \mathbb{R} \cup \{\infty\}$. $v$ a valuation $\implies \text{ if } \abs{x} = c^{-v(x)},\, c \in \mathbb{R}_{>1}$, then $\abs{\cdot}$ is an absolute value. Conversely, if $\abs{\cdot}$ is an absolute value then $v(x)=-\log_c\abs{x}$.

Let $K$ be a valued field.
\begin{itemize}
	\item $\mathcal{O} = \mathcal{O}_K = \{x \in K \,|\, \abs{x} \leq 1\}$ is the \textbf{valuation ring}
	\item $\mathfrak{m} = \mathfrak{m}_K = \{x \in K \,|\, \abs{x} < 1\}$ is the \textbf{maximal ideal}
	\item $k = k_K = \mathcal{O}/\mathfrak{m}$ is the \textbf{residue field}
\end{itemize}

\begin{definition}
	If $K$ is a valued field and $F(x)=a_0 + a_1 x + \dots + a_n x^n \in K[x]$ is a polynomial, we say that F is \textbf{primitive} if $\max_i \abs{a_i} = 1$ ($\implies F \in \mathcal{O}[x]$).
\end{definition}

\begin{theorem}[Hensel's Lemma]
	Assume that $K$ is complete and that $F \in K[x]$ is primitive. Put $f = F \mod \mathfrak{m} \in k[x]$. If $\exists$ factorisation $f(x)=g(x)h(x)$ with $(g,h)=1$, then $\exists$ factorisation $F(x)=G(x)H(x)$ in $\mathcal{O}[x]$ with $g \equiv G$, $h \equiv H \mod \mathfrak{m}$ and $\deg g$ = $\deg G$.
\end{theorem}

\begin{proof}
	Put $d = \deg F$, $m = \deg g$, so $\deg h \leq d-m$. Pick lifts $G_0, H_0 \in \mathcal{O}[x]$ of $g, h$ with $\deg G_0 = \deg g$, $\deg H_0 \leq d-m$.
	
	$(g, h) = 1 \implies \exists A, B \in \mathcal{O}[x]$ s.t. $AG_0 + BH_0 \equiv 1 \mod \mathfrak{m}$.
	
	Pick $\pi \in \mathfrak{m}$ s.t. $F-G_0H_0 \equiv AG_0+BH_0-1 \mod \pi$.
	
	Want to find $G = G_0 + \pi P_1 + \pi^2 P_2 + \dots$, $H = H_0 + \pi Q_1 + \pi^2 Q_2 + \dots \in \mathcal{O}[x]$ with $P_i, Q_i \in \mathcal{O}[x]$, $\deg P_i < m$, $\deg Q_i \leq d-m$.
	
	Define
	\begin{align*}
		G_{n-1} &= G_0 + \pi P_1 + \dots + \pi^{n-1}P_{n-1}\\
		H_{n-1} &= H_0 + \pi Q_1 + \dots + \pi^{n-1}Q_{n-1}
	\end{align*}
	We want $F \equiv G_{n-1}H_{n-1} \mod \pi^n$, then take the limit.
	
	Induction on $n$: $n = 1\ \checkmark$
	
	Assume we have $G_{n-1}, H_{n-1}$, $G_n = G_{n-1}+\pi^nP_n, H_n = H_{n-1} + \pi^n Q_n$. Expanding $F-H_nG_n$, we want $$F-G_{n-1}H_{n-1} \equiv \pi^n(G_{n-1}Q_n + H_{n-1}P_n) \mod \pi^{n+1}$$ and divide by $\pi^n$ $$G_{n-1}Q_n + H_{n-1}P_n = \frac{1}{\pi^n}\left(F-G_{n-1}H_{n-1}\right) \mod \pi$$ Let $F_n := F-G_{n-1}H_{n-1}$. $AG_o + BH_0 \equiv 1 \mod \pi \implies F_n \equiv AG_0F_n + BH_0F_n \mod \pi$.
	
	Write $BF_n = QG_0 + P_n$ with $\deg P_n < \deg G_0$, $P_n \in \mathcal{O}[x]$ $$\implies G_0(AF_n + H_0Q) + H_0P_n \equiv F_n \mod \pi$$ Now omit all coefficients from $AF_n+H_0Q$ divisible by $\pi$ to get $Q_n$.
\end{proof}

\begin{corollary}
	Let $F(x)=a_0 + a_1 x + \dots + a_n x^n \in K[x]$, $K$ complete, $a_0a_n \neq 0$. If $F$ is irreducible, then $\abs{a_i} \leq \max(\abs{a_0}, \abs{a_n})\ \forall i$.
\end{corollary}

\begin{corollary}
	$F \in \mathcal{O}[x]$ monic, $K$ complete. If $F \mod \mathfrak{m}$ has a simple root $\bar{\alpha} \in k$, then $F$ has a (unique) simple root $\alpha \in \mathcal{O}$ lifting $\bar{\alpha}$.
\end{corollary}
\end{document}