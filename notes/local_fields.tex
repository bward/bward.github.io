\documentclass[a4paper]{article}
\title{Part III Local Fields}
\author{Based on lectures by Dr C. Johansson}
\date{Michaelmas 2016\\University of Cambridge}
\usepackage{mathtools}
\usepackage{amsthm}
\usepackage{amssymb}
\usepackage{textcomp}
\usepackage{enumitem}
\usepackage{graphicx}
\usepackage{tikz-cd}
\usepackage{enumitem}
\newtheorem*{definition}{Definition}
\newtheorem{lemma}{Lemma}
\newtheorem{prop}[lemma]{Proposition}
\newtheorem{corollary}[lemma]{Corollary}
\renewcommand{\baselinestretch}{1.3}
\newcommand*\conj[1]{\overline{#1}}
\newcommand*\dom[1]{\textnormal{dom}\,#1}
\newcommand*\cod[1]{\textnormal{cod}\,#1}
\newcommand*\ob[1]{\textnormal{ob}\,#1}
\newcommand*\mor[1]{\textnormal{mor}\,#1}
\newcommand*\abs[1]{\left|#1\right|}
\begin{document}
\maketitle
\tableofcontents

\section{Basic Theory}
\begin{definition}[Absolute value]
	Let $K$ be a field. An \textbf{absolute value} on $K$ is a function $\abs{\cdot}:K\to\mathbb{R}_{\geq 0}$ s.t.
	\begin{enumerate}[label=\roman*.]
		\item $\abs{x}=0 \iff x=0$
		\item $\abs{xy}=\abs{x}\abs{y} \quad \forall x,y \in K$
		\item $\abs{x+y} \leq \abs{x} + \abs{y}$
	\end{enumerate}
\end{definition}

\begin{definition}[Valued field]
	A \textbf{valued field} is a field with an absolute value.
\end{definition}

\begin{definition}[Equivalence of absolute values]
	Let $K$ be a field and let $\abs{\cdot}$, $\abs{\cdot}^{'}$ be absolute values on K. We say that $\abs{\cdot}$ and $\abs{\cdot}^{'}$ are \textbf{equivalent} if the associated metrics induce the same topology.
\end{definition}

\begin{definition}[Non-archimedean absolute value]
	An absolute value $\abs{\cdot}$ on a field $K$ is called \textbf{non-archimedean} if $\abs{x+y}\leq\max(\abs{x},\abs{y})$ (the \textbf{strong triangle inequality}).
	
	Metrics s.t. $d(x,z)\leq\max(d(x,y),d(y,z))$ are called \textbf{ultrametrics}.
\end{definition}

Assumption: unless otherwise mentioned, all absolute values will be non-archimedean. These metrics are weird!

\begin{prop}
	Let $K$ be a valued field. Then $\mathcal{O}=\{x \,|\, \abs{x}\leq 1\}$ is an open subring of $K$, called the \textbf{valuation ring} of K. $\forall r \in (0,1]$, $\{x \,|\, x < r\}$ and $\{x \,|\, x \leq r\}$ are open ideals of $\mathcal{O}$.
	
	Moreover, $\mathcal{O}^x = \{x \,|\, \abs{x}=1\}$.
\end{prop}

\begin{prop}
	Let $K$ be a valued field.
	\begin{enumerate}[label=\roman*.]
		\item Let $(x_n)$ be a sequence in $K$. If $x_n - x_{n+1} \to 0$ then $(x_n)$ is Cauchy
	\end{enumerate}
	Assume that $K$ is complete
	\begin{enumerate}[label=\roman*.]
		\setcounter{enumi}{1}
		\item Let $(x_n)$ be a sequence in $K$. If $x_n - x_{n+1} \to 0$ then $(x_n)$ converges
		\item Let $\sum_{n=0}^{\infty}y_n$ be a series in $K$. If $y_n\to0$, then $\sum_{n=0}^{\infty}y_n$ converges
	\end{enumerate}
\end{prop}

\begin{definition}
	Let $R \subseteq S$ be rings. Then $s \in S$ is \textbf{integral over R} if $\exists$ monic $f(x) \in R[x]$ s.t. $f(s)=0$.
\end{definition}

\begin{prop}
	Let $R \subseteq S$ be rings. Then $s_1, \dots,s_n\in S$ are all integral over $R \iff R[s_1,\dots,s_n] \subseteq S$ is a finitely generated $R$-module.
\end{prop}

\begin{corollary}
	let $R \subseteq S$ be rings. If $s_1, s_2 \in S$ are integral over $R$, then $s_1+s_2$ and $s_1s_2$ are integral over $R$. In particular, the set $\tilde{R}\subseteq S$ of all elements in $S$ integral over $R$ is a ring, called the \textbf{integral closure} of $R$ in $S$.
\end{corollary}

\begin{definition}
	Let $R$ be a ring. A topology on $R$ is called a \textbf{ring topology} on $R$ if addition and multiplication are continuous maps $R\times R\to R$. A ring with a ring topology is called a \textbf{topological ring}.
\end{definition}

\begin{definition}
	Let $R$ be a ring, $I \subseteq R$ an ideal. A subset $U \subseteq R$ is called \linebreak \textbf{I-adically open} if $\forall x \in U$ $\exists n \geq 1$ s.t. $x + I^n \subseteq U$.
\end{definition}

\begin{prop}
	The set of all $I$-adically open sets form a topology on $R$, called the \textbf{I-adic topology}.
\end{prop}

\begin{definition}
	Let $R_1, R_2, \dots$ be topological rings with continuous homomorphisms $f_n:R_{n+1}\to R_n$ $\forall n \geq 1$. The \textbf{inverse limit} of the $R_i$ is the ring
	\begin{align*}
		\varprojlim_n R_n &= \left\{(x_n) \in \prod_n R_n \,|\, f_n(x_{n+1})=x_n \forall n \geq 1  \right\}\\
		&\subseteq \prod_n R_n
	\end{align*}
\end{definition}

\begin{prop}
	The inverse limit topology is a ring topology.
\end{prop}

\begin{definition}
	Let $R$ be a ring, $I$ an ideal. The \textbf{I-adic completion} of $R$ is the topological ring $\varprojlim_n R/I^n$ ($R/I^n$ has the discrete topology, and $R/I^{n+1} \to R/I^{n}$ is the natural map).
\end{definition}
\end{document}