\documentclass[a4paper]{article}
\title{Part III Combinatorics}
\author{Based on lectures by Prof B. Bollob\'{a}s}
\date{Michaelmas 2016\\University of Cambridge}
\usepackage{mathtools}
\usepackage{amsthm}
\usepackage{amssymb}
\usepackage{textcomp}
\usepackage{enumitem}
\usepackage{graphicx}
\usepackage{tikz-cd}
\usepackage{enumitem}
\newtheorem*{definition}{Definition}
\newtheorem{theorem}{Theorem}
\newtheorem{lemma}[theorem]{Lemma}
\newtheorem{prop}[theorem]{Proposition}
\newtheorem{corollary}[theorem]{Corollary}
\renewcommand{\baselinestretch}{1.3}
\newcommand*\conj[1]{\overline{#1}}
\newcommand*\dom[1]{\textnormal{dom}\,#1}
\newcommand*\cod[1]{\textnormal{cod}\,#1}
\newcommand*\ob[1]{\textnormal{ob}\,#1}
\newcommand*\mor[1]{\textnormal{mor}\,#1}
\newcommand*\abs[1]{\left|#1\right|}
\begin{document}
\maketitle
\tableofcontents

\section{Introduction}
Let $X, Y, \dots$ be sets
\begin{definition}
	We call $\mathcal{A}\subset \mathcal{P}(X)$ a \textbf{set system} or \textbf{family of sets}. $\mathcal{A}$ is naturally identified with a bipartite graph $G_\mathcal{A}(U,W)$ with $U=\mathcal{A}$, $W=\bigcup_{A\in\mathcal{A}}A$ or $W=X$. Indeed, $Ax\in E(G_\mathcal{A}) \iff x \in A$.
\end{definition}

\begin{definition}
	Given $\mathcal{A}\in\mathcal{P}(X)$, a \textbf{set of distinct representatives} (SDR) is an injection $f:\mathcal{A}\to X$ s.t. $f(A)\in A$ $\forall A \in \mathcal{A}$. In its bipartite graph, an SDR corresponds to a complete matching $U \to W$.
\end{definition}

\begin{theorem}[Hall, 1935]
	A set system $\mathcal{A}$ has an SDR if $\forall \mathcal{A}'\subset\mathcal{A}$, $\abs{\cup_{A\in\mathcal{A}'}A} \geq \abs{\mathcal{A}}'$.
\end{theorem}
\setcounter{theorem}{0}
\renewcommand{\thetheorem}{\arabic{theorem}'}
\begin{theorem}
	A bipartite graph $G(U,W)$ has a complete matching $U \to W$ if $\forall S \subset U$, $\abs{\Gamma(S)} \geq \abs{S}$
\end{theorem}
\renewcommand{\thetheorem}{\arabic{theorem}}
\begin{corollary}
	Suppose $G(U, W)$ bipartite, $d(u) \geq d(w)$ $\forall u \in U,\ w \in W$. Then $\exists$ a complete matching $U \to W$.
\end{corollary}

\begin{definition}
	A bipartite graph $G(U,W)$ is $(r,s)$\textbf{-regular} if $d(u)=r$ and $d(w)=s$ $\forall u\in U$, $w \in W$.
\end{definition}

Instant from Cor 2: if $G(U,W)$ is $(r,s)$-regular then $\exists$ a complete matching from $U$ to $W$ if $\abs{U} \leq \abs{W}$.

\begin{corollary}
	Let $0 \leq i,j \leq n$, ${n \choose i} \leq {n \choose j}$. Then $\exists$ a complete matching $f: [n]^{(i)} \to [n]^{(j)}$ s.t. $f(A) \subset A$ if $j \leq i$, and $f(A) \supset A$ if $i \leq j$.
\end{corollary}

\begin{theorem}
	Let $G=G(U,W)$ be a connected $(r,s)$-regular graph. Then for $\emptyset \neq A \subset U$,
	$$\frac{\abs{\Gamma(A)}}{\abs{W}} \geq \frac{\abs{A}}{\abs{U}}$$
	Also, equality holds iff $A=U$.
\end{theorem}

The \textbf{cube} $Q^n \cong \mathcal{P}(n) \cong [2]^n =$ set of all 0, 1 sequences of length $n$. $Q^n$ is also a graph: $AB$ is an edge if $\abs{A \vartriangle B}=1$. It is also a poset: $A<B$ if $A \subset B$.

$Q^n$ has a natural orientation: $\overrightarrow{AB}$ if $A=B \cup \{a\}$.

\begin{center}
	\begin{tikzcd}
		& 123 \arrow[ld] \arrow[d] \arrow[rd]& \\
		12 \arrow[d] \arrow[rd] & 13 \arrow[ld] \arrow[rd] & 23 \arrow[ld] \arrow[d] \\
		1 \arrow[rd] & 2 \arrow[d] & 3 \arrow[ld] \\
		& \emptyset &
	\end{tikzcd}
\end{center}

The order on $Q^n \cong \mathcal{P}(n)$ is induced by this oriented graph.
\section{Sperner Systems}
\setcounter{theorem}{0}
\begin{definition}
	A set system $\mathcal{A} \subset \mathcal{P}(n)$ is \textbf{Sperner} if $A,B \in \mathcal{A}$, $A \neq B \implies A \not\subset B$
\end{definition}
\begin{theorem}[Sperner, 1928]
	If $\mathcal{A} \subset \mathcal{P}(n)$ is Sperner then $$\abs{\mathcal{A}} \leq {n \choose \lfloor\frac{n}{2}\rfloor}$$
\end{theorem}
\begin{definition}
	The \textbf{weight} $w(A)$ of a set $A \in \mathcal{P}(n)$ is $w(A) = \frac{1}{{n \choose \abs{A}}}$
\end{definition}
\begin{theorem}
	Let $\mathcal{A}$ be a Sperner system on $X$, $\abs{X}=n$. Then $$w(\mathcal{A}) = \sum_{A \in \mathcal{A}}w(A) \leq 1$$
\end{theorem}
\end{document}