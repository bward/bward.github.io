\documentclass[a4paper]{article}
\title{Part III Algebraic Geometry}
\author{Based on lectures by Dr C. Birkar}
\date{Michaelmas 2016\\University of Cambridge}
\usepackage{mathtools}
\usepackage{amsthm}
\usepackage{amssymb}
\usepackage{textcomp}
\usepackage{enumitem}
\usepackage{graphicx}
\usepackage{tikz-cd}
\renewcommand{\baselinestretch}{1.3}
\newtheorem*{definition}{Definition}
\newtheorem*{fact}{Fact}
\newtheorem*{remark}{Remark}
\newtheorem{theorem}{Theorem}
\newtheorem{lemma}[theorem]{Lemma}
\DeclareMathOperator{\Ima}{Im}
\DeclareMathOperator{\Ker}{Ker}
\newcommand{\properideal}{%
	\mathrel{\ooalign{$\lneq$\cr\raise.22ex\hbox{$\lhd$}\cr}}}
\begin{document}
\maketitle
\tableofcontents

\section{Sheaves}

\begin{definition}[Presheaf]
	Let $X$ be a topological space. A \textbf{presheaf} $\mathcal{F}$ consists of a collection of abelian groups, $\mathcal{F}(U)$, where $U \subseteq X$ are the open subsets of $X$ s.t. $\mathcal{F}(\emptyset)=0$.
	
	$\exists$ a homomorphism $\mathcal{F}(U)\to\mathcal{F}(V)$, $s \mapsto s|_V$ for each inclusion $V\subseteq U$ of open sets. $\mathcal{F}(U)\to\mathcal{F}(U)$ is the identity map. If $W\subseteq V \subseteq U$ are open sets then $\forall s \in \mathcal{F}(U)$, $(s|_V)|_W=s|_W$.
\end{definition}

\begin{definition}[Sheaf]
	A \textbf{sheaf} $\mathcal{F}$ is a presheaf s.t. if $U=\bigcup U_i$, $U$, $U_i$ open and if $s_i \in \mathcal{F}(U_i)$ s.t. $s_i|_{U_i\cap U_j}=s_j|_{U_i \cap U_j}$ $\forall i,j$ then $\exists ! s\in\mathcal{F}(U)$ s.t. $s|_{U_i}=s_i$ $\forall i$.
\end{definition}

\begin{definition}[Stalk]
	Let $X$ be a topological space, $\mathcal{F}$ a presheaf, $x \in X$. Define the \textbf{stalk} of $\mathcal{F}$ at $x$ by $\mathcal{F}_x = \lim_{U \ni x}\mathcal{F}(U)$.
	
	More explicitly, each element of $\mathcal{F}_x$ is given by a pair $(U, s)$ where $x \in U$ open, $s \in \mathcal{F}(U)$ subject to the condition $$(U,s)=(V,t) \text{ if } \exists x \in W \subseteq U \cap V \text{ s.t. } s|_W=t|_W$$
\end{definition}

\begin{definition}[Morphism]
	Let $X$ be a topological space, $\mathcal{F}$, $\mathcal{G}$ presheaves. A \textbf{morphism} $\varphi: \mathcal{F} \to \mathcal{G}$ is given by a collection of homomorphisms $\mathcal{F}(U)\overset{\varphi(U)}{\to}\mathcal{G}(U)$ s.t. if $V \subseteq U$, the diagram
	\begin{center}

	\begin{tikzcd}
		\mathcal{F}(U) \arrow[r, "\varphi_U"] \arrow[d] & \mathcal{G}(U) \arrow[d] \\
		\mathcal{F}(V) \arrow[r, "\varphi_V"] & \mathcal{G}(V)
	\end{tikzcd}
	\end{center}
	
	\noindent commutes. We say $\varphi$ is an \textbf{isomorphism} if it has an inverse.
\end{definition}

\begin{remark}
	$\forall x \in X$, any morphism $\varphi: \mathcal{F} \to \mathcal{G}$, we get a homomorphism
	\begin{align*}
	\varphi_x:\mathcal{F}_x &\to \mathcal{G}_x\\
	(U, s) &\mapsto (U, \varphi_U(s))
	\end{align*}
\end{remark}

\begin{definition}
	Let $X$ be a topological space, $\mathcal{F}$ a presheaf. Then $\exists$ a sheaf $\mathcal{F}^+$ and a morphism $\alpha:\mathcal{F}\to\mathcal{F}^+$ s.t. if $\varphi:\mathcal{F}\to\mathcal{G}$ is a morphism into a sheaf $\mathcal{G}$, then $\varphi$ factors uniquely
	\begin{center}
	\begin{tikzcd}[row sep=tiny]
		&\mathcal{F}^+ \arrow[dd]\\
		\mathcal{F}\arrow[ru, "\alpha"] \arrow[rd, "\varphi"] &\\
		&\mathcal{G}
	\end{tikzcd}
	\end{center}
	
	\noindent for some morphism $\mathcal{F}^+ \to \mathcal{G}$. We call $\mathcal{F}^+$ the sheaf \textbf{associated} to $\mathcal{F}$.
	
	$\mathcal{F}^+$ is constructed as follows:
	\[\mathcal{F}^+(U):=\left\{\text{functions }s:U\to\bigsqcup_{x\in U}\mathcal{F}_x
	\ \middle|\ 
	\begin{tabular}{c}
	$\forall x \in U,\  s(x) \in \mathcal{F}_x,\ \exists x \in W \subseteq V \text{ and }$\\$ t \in\mathcal{F}(W)\text{ s.t. } s(y)=(V,t)\in\mathcal{F}_y\ \forall y \in W$
	\end{tabular}
	\right\}\]
\end{definition}

\begin{definition}[Kernel and Image]
	Let $X$ be a topological space, $\mathcal{F}\overset{\varphi}{\to}\mathcal{G}$ a morphism of presheaves. The \textbf{kernel} of $\varphi$, denoted $\Ker\varphi$, is defined by
	$$(\Ker\varphi)(U)=\Ker(\varphi_U:\mathcal{F}(U)\to\mathcal{G}(U)$$
	
	The \textbf{presheaf image} of $\varphi$, denoted $\Ima(\varphi^{pre})$ is defined by
	$$(\Ima\varphi^{pre})(U)=\Ima(\varphi_U)$$
	
	Now assume $\mathcal{F}$ and $\mathcal{G}$ are sheaves. Define the kernel of $\varphi=\Ker\varphi$ as above, which is a sheaf. Define the image of $\varphi$ by $\Ima(\varphi^{pre})^+$, denoted $\Ima \varphi$.
\end{definition}

\section{Sheaves II}
\begin{theorem}
	Suppose $\varphi:\mathcal{F}\to\mathcal{G}$ is a morphism of sheaves on a topological space $X$. Then
	\begin{enumerate}[label=\roman*.]
		\item $\varphi$ is injective $\iff \varphi_x:\mathcal{F}_x\to\mathcal{G}_x$ is injective $\forall x \in X$
		\item $\varphi$ is surjective $\iff \varphi_x:\mathcal{F}_x\to\mathcal{G}_x$ is surjective $\forall x \in X$
		\item $\varphi$ is an isomorphism $\iff \varphi_x:\mathcal{F}_x\to\mathcal{G}_x$ is an isomorphism $\forall x \in X$
	\end{enumerate}
\end{theorem}

\begin{definition}
	Let $X$ be a topological space. A \textbf{complex of sheaves} is a sequence
	$$\dots \to \mathcal{F}_{-2} \overset{\varphi_{-2}}{\to} \mathcal{F}_{-1} \overset{\varphi_{-1}}{\to} \mathcal{F}_0 \overset{\varphi_0}{\to} \mathcal{F}_1 \overset{\varphi_1}{\to} \mathcal{F}_2 \overset{\varphi_2}{\to} \dots$$
	of sheaves s.t. $\Ima \varphi_i \subseteq \Ker \varphi_{i+1}$ $\forall i$. We say it is an \textbf{exact sequence} if $\Ima \varphi_i = \Ker\varphi_{i+1}$ $\forall i$. An exact sequence
	$$0 \to \mathcal{F} \to \mathcal{G} \to \mathcal{H} \to 0$$
	is called a \textbf{short exact sequence}.
\end{definition}

\begin{definition}[Constant sheaf]
	Let $X$ be a topological space and $A$ an abelian group. Define a presheaf $\mathcal{F}$ by $\mathcal{F}(U)=A$ $\forall \text{ open } U \neq \emptyset$. We call $\mathcal{F}^+$ the \textbf{constant sheaf} associated to $A$.
\end{definition}

\begin{definition}[Direct image]
	Let $f:X\to Y$ be a continuous map of topological spaces, and let $\mathcal{F}$ be a presheaf on $X$. The \textbf{direct image} $f_*\mathcal{F}$ is defined by $$(f_*\mathcal{F})(V) = \mathcal{F}(f^{-1}V)$$
\end{definition}

\begin{definition}[Skyscraper sheaf]
	Let $X$ be a topological space, $x \in X$, $A$ an abelian group. Define $\mathcal{F}$ by \[\mathcal{F} = \left\{
	\begin{tabular}{r}
	$A$ if $x \in U$\\$0$ if $x \not\in U$
	\end{tabular}\right.
	\]
	We call $\mathcal{F}$ the \textbf{skyscraper sheaf} associated to $A$ at $x$.
\end{definition}

\section{Basics of Commutative Algebra}
Ref: Atiyah and Macdonald, \textit{Introduction to Commutative Algebra}.

Convention: all rings in this course are commutative with 1 (unless ring $=0$).

\begin{definition}
	Let $A$ be a ring. A \textbf{prime ideal} is an ideal $P \trianglelefteq A$ s.t. $ab \in P \implies a \in P$ or $b \in P$. \textbf{maximal ideal} is an ideal $M \trianglelefteq A$ s.t. if $M \subseteq I \properideal A$, then $M=I$. (Maximal ideals are prime).
	
	$A$ is an \textbf{integral domain} if 0 is a prime ideal.
	
	$A$ is a \textbf{local ring} if it has a unique maximal ideal.
	
	A \textbf{local homomorphism} is a homomorphism of local rings $\alpha: A \to B$ s.t. $\alpha(\text{maximal ideal of } A) \subseteq$ maximal ideal of $B$.
\end{definition}

\begin{definition}[Radical]
	Let $A$, be a ring, $I \trianglelefteq A$. The \textbf{radical} of I is defined as $$\sqrt{I} = \{a\in A \,|\, a^n \in I\text{, some } n \in N\}$$
\end{definition}
\noindent$\sqrt{I}$ is an ideal. Fact: $\sqrt{I} = \bigcap_{I \subseteq \text{ prime } P \trianglelefteq A} P$

\begin{fact}
	Let $K$ be an algebraically closed field. Then $I \trianglelefteq K[t_1,\dots,t_n]$ is maximal $\iff I = \langle t_1-a_1,\dots,t_n-a_n\rangle$ for some $a_1,\dots,a_n \in K$.
\end{fact}

\begin{definition}[Ring of fractions]
	Let $A$ be a ring. $S \subseteq A$ is a \textbf{multiplicatively closed set} if $1 \in S$ and $s,t \in S \implies st \in S$.
	
	Pick such a set. Let $S^{-1}A = \{\frac{a}{s} \,|\, a \in A,\, s \in S\}$ subject to $\frac{a}{s} = \frac{b}{t} \iff \exists u \in S$ s.t. $u(at-bs)=0$.
	
	$S^{-1}A$ is a ring: $\frac{a}{s}+\frac{b}{t} = \frac{at+bs}{st}$, $\frac{a}{s}\cdot\frac{b}{t}=\frac{ab}{st}$.
	
	We call $S^{-1}A$ the \textbf{ring of fractions} of $A$ w.r.t $S$.
\end{definition}
We have a homomoprhism $\alpha:A \to S^{-1}A$, $a \mapsto \frac{a}{1}$. If $s \in S$, then $\alpha(s) = \frac{s}{1}$ is a unit in $S^{-1}A$. Also, $\alpha{a}=0 \iff \exists u \in S$ s.t. $ua=0$.

If $I \trianglelefteq A$, then $S^{-1}I = \{\frac{a}{s}\in S^{-1}A \,|\, a \in I\}$ is an ideal of $S^{-1}A$.

\begin{fact}
	Every ideal of $S^{-1}A$ is of the form $S^{-1}I$ for some $I \trianglelefteq A$.
	
	$$\{P \trianglelefteq A \,|\, P \text{ prime, } P \cap S = \emptyset\} \overset{1-1}{\longleftrightarrow} \{\text{prime ideals of } S^{-1}A\}$$
\end{fact}

\begin{definition}
	Let $A$ be a ring, $S$ a multiplicatively closed set, $M$ an $A$-module. Let $S^{-1}M = \{\frac{m}{s} \,|\, m\in M,\, s\in S\}$ subject to: $\frac{m}{s}=\frac{n}{t} \iff \exists u \in S$ s.t. $u(tm-sn)=0$.
	Thus $\frac{m}{s}+\frac{n}{t}=\frac{tm+sn}{st}$, $\frac{a}{s}\cdot\frac{m}{t}=\frac{am}{st}$, $\forall a\in A ,\, m,n \in M,\, s,t \in S$. So $S^{-1}M$ is an $S^{-1}A$-module.
\end{definition}

If $M \to N$ is an $A$-homomorphism, then we get an $S^{-1}A$-homomorphism $S^{-1}M \to S^{-1}N$.

\begin{fact}
	$S^{-1}(M/K)=(S^{-1}M)/(S^{-1}K)$, $K\subseteq M$ a submodule.
\end{fact}

\begin{definition}[Direct product and direct sum]
	Let $A$ be a ring, $\{M_i\}_{i\in I}$ a family of $A$-modules. Define the \textbf{direct product} $$\prod_{i \in I}M_i = \left\{(m_i)_{i \in I} \,|\, m_i \in M_i\right\}$$
	
	This is an $A$-module: $(m_i)+(n_i) = (m_i+n_i)$, $a\cdot(m_i)=(am_i)$ $\forall m_i,n_i \in M_i ,\, a \in A$.
	
	Also define the \textbf{direct sum}
	$$\bigoplus_{i \in I} M_i = \left\{(m_i)\in \prod_{i \in I}M_i \,|\, m_i=0 \text{ for all but finitely many } i\right\}$$
\end{definition}
\end{document}