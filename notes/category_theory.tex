\documentclass[a4paper]{article}
\title{Part III Category Theory}
\author{Based on lectures by Prof P.T. Johnstone}
\date{Michaelmas 2016\\University of Cambridge}
\usepackage{mathtools}
\usepackage{amsthm}
\usepackage{amssymb}
\usepackage{textcomp}
\usepackage{enumerate}
\usepackage{graphicx}
\usepackage{tikz-cd}
\usepackage{enumitem}
\newtheorem*{definition}{Definition}
\newtheorem{lemma}{Lemma}
\renewcommand{\baselinestretch}{1.3}
\newcommand*\conj[1]{\overline{#1}}
\newcommand*\dom[1]{\textnormal{dom}\,#1}
\newcommand*\cod[1]{\textnormal{cod}\,#1}
\newcommand*\ob[1]{\textnormal{ob}\,#1}
\newcommand*\mor[1]{\textnormal{mor}\,#1}
\begin{document}
\maketitle
\tableofcontents

\section{Definitions and Examples}
\begin{definition}[Category]
	A category $\mathcal{C}$ consists of 
	\begin{enumerate}[label=\alph*.]
		\item a collection $\ob{\mathcal{C}}$ of \textbf{objects} $A$, $B$, $C$, \dots
		\item a collection $\mor{\mathcal{C}}$ of \textbf{morphisms} $f$, $g$, $h$, \dots
		\item two operations \dom, \cod from morphisms to objects. We write $f:A\to B$ or $A\overset{f}{\to}B$ to mean '$f$ is a morphism and $\dom{f}=A$ and $\cod{f}=B$'
		\item an operation assigning to each object $A$ a morphism $1_A:A\to A$
		\item a partial binary operation $(f, g) \mapsto gf$, s.t. $gf$ is defined $\iff \dom{g}=\cod{f}$, and then $gf:\dom{f}\to\cod{g}$
	\end{enumerate}
\end{definition}

\begin{definition}[Functor]
	Let $\mathcal{C}$ and $\mathcal{D}$ be categories. A \textbf{functor} $\mathcal{C}\to\mathcal{D}$ consists of
	\begin{enumerate}[label=\alph*.]
		\item a mapping $A \to FA$ from $\ob{\mathcal{C}}$ to $\ob{\mathcal{D}}$
		\item a mapping $f \to Ff$ from $\mor{\mathcal{C}}$ to $\mor{\mathcal{D}}$
	\end{enumerate}
	satisfying $\dom{Ff}=F\dom{f}$, $\cod{Ff}=F\cod{f}$ for all $f$, $F(1_A)=1_{FA}$ for all $A$, and $F(gf)=(Fg)(Ff)$ whenever $gf$ is defined.
\end{definition}

\begin{definition}
	By a \textbf{contravariant functor} $\mathcal{C} \to \mathcal{D}$ we mean a functor $\mathcal{C} \to \mathcal{D}^{op}$ (or equivalently $\mathcal{C}^{op} \to \mathcal{D}$). A functor $\mathcal{C} \to \mathcal{D}$ is sometimes said to be \textbf{covariant}.
\end{definition}

\begin{definition}[Natural transformation]
	Let $\mathcal{C}$ and $\mathcal{D}$ be two categories and $F, G: \mathcal{C} \rightrightarrows \mathcal{D}$ two functors. A \textbf{natural transformation} $\alpha:F\to G$ assigns to each $A \in \ob{\mathcal{C}}$ a morphism $\alpha_A:FA\to GA$ in $\mathcal{D}$, such that
	\begin{center}
		\begin{tikzcd}
			FA \arrow[r, "Ff"] \arrow[d, "\alpha_A"] & FB \arrow[d, "\alpha_B"]\\
			GA \arrow[r, "Gf"]& GB
		\end{tikzcd}
	\end{center}
	commutes.
\end{definition}

We can compose natural transformations: given $\alpha: F \to G$ and $\beta: G \to H$, the mapping $A \mapsto \beta_A \alpha_A$ is the $A$-component of a natural transformation $\beta\alpha: F \to H$.

\begin{definition}
	Given categories $\mathcal{C}, \mathcal{D}$, we write $[\mathcal{C},\mathcal{D}]$ for the category of all functors $\mathcal{C} \to \mathcal{D}$ and natural transformations between them.
\end{definition}

\begin{lemma}
	Given $F,G: \mathcal{C} \to \mathcal{D}$ and $\alpha: F \to G$, $\alpha$ is an isomorphism in $[\mathcal{C}, \mathcal{D}] \iff$ each $\alpha_A$ is an isomorphism in $\mathcal{D}$.
\end{lemma}

\begin{definition}[Faithful and full]
	Let $F: \mathcal{C} \to \mathcal{D}$ be a functor.
	\begin{enumerate}[label=\alph*.]
		\item We say that $F$ is \textbf{faithful} if, given $f, g \in \mor{\mathcal{C}}$, the equations $\dom{f}=\dom{g}$, $\cod{f}=\cod{g}$ and $Ff = Fg$ imply $f=g$.
		\item $F$ is \textbf{full} if, given any $g: FA \to FB$ in $\mathcal{D}$, there exists $f: A \to B$ in $\mathcal{C}$ with $Ff =g$.
		\item We say a subcategory $\mathcal{C}'$ of $\mathcal{C}$ is \textbf{full} if the inclusion $\mathcal{C}' \hookrightarrow \mathcal{C}$ is a full functor.
	\end{enumerate}
\end{definition}

For example, \textbf{Gp} is a full subcategory of the category \textbf{Mon} of monoids, but \textbf{Mon} is a non-full subcategory of the category \textbf{Sgp} of semigroups.

\begin{definition}[Equivalence of categories]
	Let $\mathcal{C}$ and $\mathcal{D}$ be categories. An \textbf{equivalence} between $\mathcal{C}$ and $\mathcal{D}$ is a pair of functors $F: \mathcal{C} \to \mathcal{D}$, $G: \mathcal{D} \to \mathcal{C}$ together with natural isomorphisms $\alpha: 1_\mathcal{C} \to GF$, $\beta: FG \to 1_\mathcal{D}$. We write $\mathcal{C} \simeq \mathcal{D}$ to mean that $\mathcal{C}$ and $\mathcal{D}$ are equivalent.
	
	We say a property $P$ of categories is \textbf{categorical} if whenever $\mathcal{C}$ has $P$ and $\mathcal{C}\simeq\mathcal{D}$ then $\mathcal{D}$ has $P$.
\end{definition}

For example, being a groupoid is a categorical property, but being a group is not.

\begin{definition}[Slice category]
	Given an object $B$ of a category $\mathcal{C}$, define the \textbf{slice category} $\mathcal{C}/B$ to have morphisms $A \overset{f}{\to} B$ as objects, and morphisms $(A \overset{f}{\to} B) \to (A' \overset{f'}{\to} B)$ are morphisms $h: A \to A'$ making
	\begin{center}
		\begin{tikzcd}[column sep=tiny]
				A \arrow[rr, "h"] \arrow[rd, "f"] & & A'\arrow[ld, "f'"]\\
				& B &
		\end{tikzcd}
	\end{center}
	commute.
\end{definition}
\end{document}