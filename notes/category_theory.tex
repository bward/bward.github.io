\documentclass[a4paper]{article}
\title{Part III Category Theory}
\author{Based on lectures by Prof P.T. Johnstone}
\date{Michaelmas 2016\\University of Cambridge}
\usepackage{mathtools}
\usepackage{amsthm}
\usepackage{amsmath}
\usepackage{amssymb}
\usepackage{textcomp}
\usepackage{enumerate}
\usepackage{graphicx}
\usepackage{tikz-cd}
\usepackage{enumitem}
\usepackage{stmaryrd}
\theoremstyle{definition}
\newtheorem{definition}{Definition}
\theoremstyle{remark}
\newtheorem{remark}[definition]{Remark}
\theoremstyle{default}
\newtheorem{lemma}[definition]{Lemma}
\newtheorem{corollary}[definition]{Corollary}
\newtheorem{theorem}[definition]{Theorem}
\numberwithin{definition}{section}
\renewcommand{\baselinestretch}{1.3}
\newcommand*\conj[1]{\overline{#1}}
\newcommand*\dom[1]{\textnormal{dom}\,#1}
\newcommand*\cod[1]{\textnormal{cod}\,#1}
\newcommand*\ob[1]{\textnormal{ob}\,#1}
\newcommand*\mor[1]{\textnormal{mor}\,#1}
\newcommand*\circbox[1]{\raisebox{.5pt}{\textcircled{\raisebox{-.9pt} {#1}}}}
\newcommand*\parallelpair[2]{\overset{#1}{\underset{#2}{\rightrightarrows}}}
\newcommand*\rightleft[2]{\overset{#1}{\underset{#2}{\rightleftarrows}}}
\newcommand*\col[2]{\begin{psmallmatrix}#1 \\ #2 \end{psmallmatrix}}
\DeclareMathOperator{\Set}{\textbf{Set}}
\DeclareMathOperator{\Sub}{\textbf{Sub}}
\DeclareMathOperator{\Adj}{\textbf{Adj}}
\DeclareMathOperator{\CMon}{\textbf{CMon}}
\DeclareMathOperator{\AbGp}{\textbf{AbGp}}
\DeclareMathOperator{\Mod}{\textbf{Mod}}
\DeclareMathOperator{\Add}{\textbf{Add}}
\DeclareMathOperator{\coeq}{coeq}
\DeclareMathOperator{\coker}{coker}
\DeclareMathOperator{\img}{im}
\DeclareMathOperator{\coim}{coim}
\DeclareMathOperator{\colim}{colim}
\begin{document}
\maketitle
\tableofcontents
\newpage
\section{Definitions and Examples}
\begin{definition}[Category]
	A category $\mathcal{C}$ consists of 
	\begin{enumerate}[label=\alph*.]
		\item a collection $\ob{\mathcal{C}}$ of \textbf{objects} $A$, $B$, $C$, \dots
		\item a collection $\mor{\mathcal{C}}$ of \textbf{morphisms} $f$, $g$, $h$, \dots
		\item two operations \dom, \cod from morphisms to objects. We write $f:A\to B$ or $A\overset{f}{\to}B$ to mean '$f$ is a morphism and $\dom{f}=A$ and $\cod{f}=B$'
		\item an operation assigning to each object $A$ a morphism $1_A:A\to A$
		\item a partial binary operation $(f, g) \mapsto gf$, s.t. $gf$ is defined $\iff \dom{g}=\cod{f}$, and then $gf:\dom{f}\to\cod{g}$
	\end{enumerate}
	satisfying
	\begin{enumerate}[label=\alph*.]
		\setcounter{enumi}{5}
		\item $f 1_A = f$ and $1_B f=f$ $\forall f: A \to B$
		\item $h(fg) = (hg)f$ whenever $gf$ and $hg$ are defined
	\end{enumerate}
\end{definition}

\begin{definition}[Functor]
	Let $\mathcal{C}$ and $\mathcal{D}$ be categories. A \textbf{functor} $\mathcal{C}\to\mathcal{D}$ consists of
	\begin{enumerate}[label=\alph*.]
		\item a mapping $A \to FA$ from $\ob{\mathcal{C}}$ to $\ob{\mathcal{D}}$
		\item a mapping $f \to Ff$ from $\mor{\mathcal{C}}$ to $\mor{\mathcal{D}}$
	\end{enumerate}
	satisfying $\dom{Ff}=F\dom{f}$, $\cod{Ff}=F\cod{f}$ for all $f$, $F(1_A)=1_{FA}$ for all $A$, and $F(gf)=(Fg)(Ff)$ whenever $gf$ is defined.
\end{definition}

\begin{definition}
	By a \textbf{contravariant functor} $\mathcal{C} \to \mathcal{D}$ we mean a functor $\mathcal{C} \to \mathcal{D}^{op}$ (or equivalently $\mathcal{C}^{op} \to \mathcal{D}$). A functor $\mathcal{C} \to \mathcal{D}$ is sometimes said to be \textbf{covariant}.
\end{definition}

\begin{definition}[Natural transformation]
	Let $\mathcal{C}$ and $\mathcal{D}$ be two categories and $F, G: \mathcal{C} \rightrightarrows \mathcal{D}$ two functors.
	A \textbf{natural transformation} $\alpha:F\to G$ assigns to each $A \in \ob{\mathcal{C}}$ a morphism $\alpha_A:FA\to GA$ in $\mathcal{D}$, such that
	\begin{center}
		\begin{tikzcd}
			FA \arrow[r, "Ff"] \arrow[d, "\alpha_A"] & FB \arrow[d, "\alpha_B"]\\
			GA \arrow[r, "Gf"]& GB
		\end{tikzcd}
	\end{center}
	commutes.
\end{definition}

We can compose natural transformations: given $\alpha: F \to G$ and $\beta: G \to H$, the mapping $A \mapsto \beta_A \alpha_A$ is the $A$-component of a natural transformation $\beta\alpha: F \to H$.

\begin{definition}
	Given categories $\mathcal{C}, \mathcal{D}$, we write $[\mathcal{C},\mathcal{D}]$ for the category of all functors $\mathcal{C} \to \mathcal{D}$ and natural transformations between them.
\end{definition}

\begin{lemma}
	Given $F,G: \mathcal{C} \to \mathcal{D}$ and $\alpha: F \to G$, $\alpha$ is an isomorphism in $[\mathcal{C}, \mathcal{D}] \iff$ each $\alpha_A$ is an isomorphism in $\mathcal{D}$.
\end{lemma}

\begin{definition}[Faithful and full]
	Let $F: \mathcal{C} \to \mathcal{D}$ be a functor.
	\begin{enumerate}[label=\alph*.]
		\item We say that $F$ is \textbf{faithful} if, given $f, g \in \mor{\mathcal{C}}$, the equations $\dom{f}=\dom{g}$, $\cod{f}=\cod{g}$ and $Ff = Fg$ imply $f=g$.
		\item $F$ is \textbf{full} if, given any $g: FA \to FB$ in $\mathcal{D}$, there exists $f: A \to B$ in $\mathcal{C}$ with $Ff =g$.
		\item We say a subcategory $\mathcal{C}'$ of $\mathcal{C}$ is \textbf{full} if the inclusion $\mathcal{C}' \hookrightarrow \mathcal{C}$ is a full functor.
	\end{enumerate}
\end{definition}

For example, \textbf{Gp} is a full subcategory of the category \textbf{Mon} of monoids, but \textbf{Mon} is a non-full subcategory of the category \textbf{Sgp} of semigroups.

\begin{definition}[Equivalence of categories]
	Let $\mathcal{C}$ and $\mathcal{D}$ be categories.
	An \textbf{equivalence} between $\mathcal{C}$ and $\mathcal{D}$ is a pair of functors $F: \mathcal{C} \to \mathcal{D}$, $G: \mathcal{D} \to \mathcal{C}$ together with natural isomorphisms $\alpha: 1_\mathcal{C} \to GF$, $\beta: FG \to 1_\mathcal{D}$.
	We write $\mathcal{C} \simeq \mathcal{D}$ to mean that $\mathcal{C}$ and $\mathcal{D}$ are equivalent.
	
	We say a property $P$ of categories is \textbf{categorical} if whenever $\mathcal{C}$ has $P$ and $\mathcal{C}\simeq\mathcal{D}$ then $\mathcal{D}$ has $P$.
\end{definition}

For example, being a groupoid is a categorical property, but being a group is not.

\begin{definition}[Slice category]
	Given an object $B$ of a category $\mathcal{C}$,
	define the \textbf{slice category} $\mathcal{C}/B$ to have morphisms $A \overset{f}{\to} B$ as objects,
	and morphisms $(A \overset{f}{\to} B) \to (A' \overset{f'}{\to} B)$ are morphisms $h: A \to A'$ making
	\begin{center}
		\begin{tikzcd}[column sep=tiny]
				A \arrow[rr, "h"] \arrow[rd, "f"] & & A'\arrow[ld, "f'"]\\
				& B &
		\end{tikzcd}
	\end{center}
	commute.
\end{definition}

\begin{lemma}
	Let $F: \mathcal{C} \to \mathcal{D}$ be a functor.
	Then $F$ is part of an equivalence $\mathcal{C} \simeq \mathcal{D} \iff$ F is full, faithful and \textbf{essentially surjective},
	i.e. for every $B \in \ob \mathcal{D}$, there exists $A \in \ob \mathcal{C}$ s.t. $FA \cong B$.
\end{lemma}

\begin{definition}
	\begin{enumerate}[label=\alph*.]
		\item A \textbf{skeleton} of a category $\mathcal{C}$ is a full subcategory $\mathcal{C}'$ containing exactly one object from each isomorphism class of objects of $\mathcal{C}$.
		\item We say $\mathcal{C}$ is \textbf{skeletal} if it's a skeleton of itself. Equivalently, any isomorphism $f$ in $\mathcal{C}$ satisfies $\dom f = \cod f$.
	\end{enumerate}
\end{definition}

For example, $\text{\textbf{Mat}}_K$ is skeletal. The full subgategory of standard vector spaces $K^n$ is a skeleton of $\text{\textbf{fd Mod}}_K$.

\begin{remark}
	The following statements are each equivalent to the Axiom of Choice:
	\begin{enumerate}
		\item Every small category has a skeleton
		\item Any small category is equivalent to each of its skeletons
		\item Any two skeletons of a given small category are isomorphic
	\end{enumerate}
\end{remark}

\begin{definition}
	Let $f: A \to B$ be a morphism in a category $\mathcal{C}$.
	\begin{enumerate}[label=\alph*.]
		\item $f$ is a \textbf{monomorphism} if, given $g,h:D \rightrightarrows A$, the equation $fg = fh$ implies $g=h$. We write $A \rightarrowtail B$ if $f$ is monic.
		\item Dually, $f$ is an \textbf{epimorphism} if, given $k,l: B \rightrightarrows C$, $kf = lf$ implies $k=l$. We write $A \twoheadrightarrow B$ if $f$ is epic.
		\item $\mathcal{C}$ is a \textbf{balanced} category if every $f \in \mor \mathcal{C}$ which is both monic and epic is an isomorphism.
	\end{enumerate}
\end{definition}

\section{The Yoneda Lemma}
\begin{definition}
	A category $\mathcal{C}$ is \textbf{locally small} if, for any two objects $A, B$ of $\mathcal{C}$, the morphism $A \to B$ are parametrised by a set $\mathcal{C}(A, B)$.
\end{definition}

Given local smallness, $B \mapsto \mathcal{C}(A,B)$ becomes a functor $\mathcal{C}(A, -): \mathcal{C} \to \Set$: if $g: B \to B'$,
the mapping $f \mapsto gf : \mathcal{C}(A, B) \to \mathcal{C}(A, B')$ is functorial since $h(gf) = (hg)f$ for any $h: B' \to B''$.

Similarly, $A \mapsto \mathcal{C}(A, B)$ becomes a functor $\mathcal{C}^{op} \to \Set$.

\begin{lemma}[Yoneda]
	Let $\mathcal{C}$ be a locally small category, $A \in \ob \mathcal{C}$ and $F: \mathcal{C} \to \Set$. Then
	\begin{enumerate}[label=\roman*.]
		\item There is a bijection between natural transformations $\mathcal{C}(A, -) \to F$ and elements of $FA$.
		\item Moreover, this bijection is natural in both $A$ and $F$.
	\end{enumerate}
\end{lemma}
\begin{proof}
	Bijection: given $\alpha : \mathcal{C}(A, -) \to F$, define $\Phi(\alpha) = \alpha_A(1_A) \in FA$.
	
	Given $x \in FA$, define $\Psi(x): \mathcal{C}(A, -) \to F$ by $$\Psi(x)_B(A \overset{f}{\to} B) = (Ff)(x) \in FB$$ $\Psi(x)$ is natural: given $g: B \to C$, we have
	\begin{align*}
	\Psi(x)_C(\mathcal{C}(A, g)(f)) &= \Psi(x)_C(gf) \\
	&= F(gf)(x) \\
	&= (Fg)(Ff)(x) \\
	&= (Fg)\Psi(x)_B(f)
	\end{align*}
	
	$\Phi\Psi(x) = x$ since $F(1_A)(x) = x$, and $\Psi\Phi(\alpha) = \alpha$ since, for any $f: A \to B$,
	\begin{align*}
	\Psi\Phi(\alpha)_B(f) &= Ff(\Phi(\alpha)) \\
	&= Ff(\alpha_A(1_A)) \\
	&= \alpha_B(\mathcal{C}(A,f)(1_A)) \\
	&= \alpha_B(f)
	\end{align*}
\end{proof}

\begin{corollary}
	The mapping $A \to \mathcal{C}(A, -)$ is a full and faithful functor $\mathcal{C}^{op} \to [\mathcal{C}, \Set]$.
\end{corollary}
\begin{proof}
	Given two objects $A,B$, 2.2(i) gives us a bijection from $\mathcal{C}(B,A)$ to the collection of natural transformations $\mathcal{C}(A, -) \to \mathcal{C}(B, -)$ (by taking $F: C \mapsto \mathcal{C}(B, C)$).
	We need to show this is functorial, but given $f \in \mathcal{C}(B, A)$, $\Psi(F)_A$ sends $1_A$ to $\mathcal{C}(B, f)(1_A) = f$, so it's the natural transformation $g \mapsto gf$.
	
	Hence, given $e: C \to B$, $\Psi(fe)(g) = g(fe) = (gf)(e) = \Psi(e)\Psi(f)g$
\end{proof}

We call this functor the \textbf{Yoneda embedding}. Hence any locally small category $\mathcal{C}$ is equivalent to a full subcategory of $[\mathcal{C}^{op}, \Set]$.

\begin{definition}
	A functor $\mathcal{C} \to \Set$ is \textbf{representable} if it's isomorphic to $\mathcal{C}(A, -)$ for some $A$.
	
	A \textbf{representation} of $F: \mathcal{C} \to \Set$ is a pair $(A, x)$ where $A \in \ob \mathcal{C}$, $x \in FA$ and $\Psi(x): \mathcal{C}(A, -) \to F$ is an isomorphism.
	We also call $x$ a \textbf{universal element} of $F$.
\end{definition}

\begin{corollary}['Representations are unique up to unique isomorphism']
	If $(A, x)$ and $(B, y)$ are both representations of $F: \mathcal{C} \to \Set$, then there's a unique isomorphism $f: A \to B$ s.t $Ff(x)=y$.
\end{corollary}

\begin{definition}[Product and coproduct]
	Given two objects $A, B$ of a locally small category $\mathcal{C}$, we define their \textbf{product} to be a representation of the functor
	$$\mathcal{C}(-,A) \times \mathcal{C}(-, B): \mathcal{C}^{op} \to \Set$$
	i.e. an object $A \times B$ equipped with morphisms $\pi_1 : A \times B \to A$, $\pi_2: A \times B \to B$
	s.t. given any pair $(f:C \to A, g: C \to B)$, there exists a unique $h: C \to A \times B$
	s.t. $\pi_1h=f$ and $\pi_2h=g$.
	
	More generally, we can define the product $\prod_{i \in I} A_i$ of a family $\{A_i \,|\, i \in I\}$ of objects,
	or the product of the empty family, i.e. a \textbf{terminal object} 1 s.t. for every $A$ there's a unique $A \to 1$.
	
	Dualizing, we get the notion of \textbf{coproduct} or \textbf{sum}.
\end{definition}

\begin{definition}[Equaliser and coequaliser]
	Given a parallel pair $f, g: A \rightrightarrows B$ in a locally small category $\mathcal{C}$,
	the assignment $C \mapsto FC = \{h: C \to A \,|\, fh=gh\}$ is a subfunctor $F$ of $\mathcal{C}(-, A)$.
	A representation of $F$ is called an \textbf{equaliser} of $(f, g)$.
	
	In elementary terms, it's an object $E$ equipped with $e: E \to A$ s.t. $fe=ge$,
	s.t. any $h$ with $fh = gh$ factors uniquely as $h=ek$
	
	Dually, we have the notion of \textbf{coequaliser},
	i.e. a morphism $q:B \to Q$ satisfying $qf=qg$, and universal among such.
\end{definition}

\begin{definition}
	\begin{enumerate}[label=\alph*.]
		\item We say a monomorphism is \textbf{regular} if it occurs as an equaliser (dually, regular epimorphism).
		\item We say $f: A \to B$ is a \textbf{split monomorphism} if there exists $g: B \to A$ with $gf = 1_A$.
	\end{enumerate}
\end{definition}

Every split monomorphism is regular:
if $gf = 1_A$, $f$ is an equaliser of $(1_B, fg)$ [see sheet 1, q2].

\begin{definition}
	Let $\mathcal{C}$ be a (locally small) category, $\mathcal{G}$ a collection of objects of $\mathcal{C}$.
	\begin{enumerate}[label=\alph*.]
		\item Say $\mathcal{G}$ is a \textbf{separating family} if the functors $\mathcal{C}(G, -),\, G \in \mathcal{G}$ are jointly faithful,
		i.e. if given $f,g : A \rightrightarrows B$ with $f \neq g$,
		there exists $G \in \mathcal{G}$ and $h: G \to A$ with $fh \neq gh$.
		\item Say $\mathcal{G}$ is a \textbf{detecting family} if the $\mathcal{C}(G, -),\, G \in \mathcal{G}$ jointly reflect isomorphisms,
		i.e. if given $f: A \to B$ s.t. every $g: G \to B$ with $G \in \mathcal{G}$ factors uniquely through $f$,
		$f$ is an isomorphism.
	\end{enumerate}
\end{definition}

\begin{lemma}
	\begin{enumerate}[label=\roman*.]
		\item If $\mathcal{C}$ is balanced, then any separating family is detecting
		\item If $\mathcal{C}$ has equalisers, then every detecting family is separating
	\end{enumerate}
\end{lemma}

\begin{definition}
	An object $P$ is \textbf{projective} if $\mathcal{C}(P, -)$ preserves epimorphisms, i.e. if given
	\begin{center}
		\begin{tikzcd}
			& P \arrow[d, "f"] \\
			A \arrow[r, twoheadrightarrow, "e"] & B
		\end{tikzcd}
	\end{center}
	there exists $g: P \to A$ with $eg = f$.
	
	Dually, $P$ is \textbf{injective} in $\mathcal{C}$ if it's projective in $\mathcal{C}^{op}$.
	
	If $P$ satisfies this property $\forall e$ in some class $\mathcal{E}$ of epimorphisms,
	we call it $\mathcal{E}$-projective.
\end{definition}

\begin{corollary}
	Representable functors are (pointwise) projective in $[\mathcal{C}, \Set]$
\end{corollary}
\begin{proof}
	Given
	\begin{center}
		\begin{tikzcd}
			& \mathcal{C}(A, -) \arrow[d, "\beta"] \\
			F \arrow[r, twoheadrightarrow, "\alpha"] & G
		\end{tikzcd}
	\end{center}
	$\beta$ corresponds to some $y \in GA$. 
	$\alpha_A$ is surjective, so $\exists x \in FA$ with $\alpha_A(x)=y$.
	$x$ corresponds to $\gamma: \mathcal{C}(A, -) \to F$ with $\alpha\gamma = \beta$.
\end{proof}

\section{Adjunctions}
\begin{definition}[D.M. Khan, 1958]
	Let $\mathcal{C}$ and $\mathcal{D}$ be categories and $F: \mathcal{C} \to \mathcal{D}$, $G: \mathcal{D} \to \mathcal{C}$ be two functors.
	An \textbf{adjunction} between $F$ and $G$ is a bijection between morphisms $FA \to B$ in $\mathcal{D}$ and morphisms $A \to GB$ in $\mathcal{C}$, 
	which is natural in $A$ and $B$.

	(If $\mathcal{C}$ and $\mathcal{D}$ are locally small,
	this says that $(A, B) \to \mathcal{D}(FA, B)$ and $(A, B) \to \mathcal{C}(A, GB)$
	are naturally isomorphic functors $\mathcal{C}^{op} \times \mathcal{D} \to \Set$).
	
	We say $F$ is \textbf{left adjoint} to $G$,
	or $G$ is \textbf{right adjoint} to $F$,
	and write $F \dashv G$.
\end{definition}

\begin{theorem}
	Let $G: \mathcal{D} \to \mathcal{C}$ be a functor.
	Given $A \in \ob \mathcal{C}$,
	let $(A \downarrow G)$ be the category whose objects are pairs $(B, f)$
	with $B \in \ob \mathcal{D}$,
	$f: A \to GB$ and whose morphisms $(B, f) \to (B', f')$
	are morphisms $g: B \to B'$ in $\mathcal{D}$ such that
	\begin{center}
		\begin{tikzcd}
			A \arrow[r, "f"] \arrow[rd, "f'"] & GB \arrow[d, "Gg"] \\
			& GB'
		\end{tikzcd}
	\end{center}
	commutes.
	Then specifying a left adjoint for $G$ is equivalent to specifying an initial object of $(A \downarrow G)$ for each $A$.
	\label{33}
\end{theorem}

\begin{proof}
	First suppose $G$ has a left adjoint $F$.
	Let $\eta_A: A \to GFA$ be the morphism corresponding to $1_{FA}:FA \to FA$.
	The pair $(FA, \eta_A)$ is an object of $(A \downarrow G)$.
	We'll show it's initial.
	
	Given $g: FA \to B$,
	the composite
	\begin{tikzcd}A \arrow[r, "\eta_A"]& GFA \arrow[r, "Gg"]& GB\end{tikzcd}
	must correspond to
	\begin{tikzcd}FA \arrow[r, "1"] & FA \arrow[r, "g"] & B\end{tikzcd}
	under the adjunction.
	
	So, for any object $(B, f)$ of $(A \downarrow G)$,
	the unique morphism $(FA, \eta_A) \to (B, f)$ in $(A \downarrow G)$ is the morphism $FA \to B$ corresponding to $f$.
	
	Conversely, suppose we're given an initial object $(FA, \eta_A)$ of $(A \downarrow G)$ for each $G$.
	Given $f: A \to A'$,
	the composite
	\begin{tikzcd}A \arrow[r, "f"] & A' \arrow[r, "\eta_{A'}"] & GFA'\end{tikzcd}
	is an object of $(A \downarrow G)$,
	so there's a unique morphism $Ff: FA \to FA'$ making
	\begin{center}
		\begin{tikzcd}
			A \arrow[r, "\eta_A"] \arrow[d, "f"] & GFA \arrow[d, "GFf"] \\
			A' \arrow[r, "\eta_{a'}"] & GFA'
		\end{tikzcd}
	\end{center}
	commute.
	
	$f \mapsto Ff$ is functorial:
	given $f': A' \to A''$,
	then $(Ff')(Ff)$ and $F(f'f)$ are both morphisms $(FA, \eta_A) \to (FA'', \eta_{A''}f'f)$ in $(A \downarrow G)$,
	so they're equal.
	
	Finally, given $f: A \to GB$,
	the morphism $g: FA \to B$ corresponding to it is the unique morphism $(FA, \eta_A) \to (B, f)$ in $(A \downarrow G)$.
	
	The naturality of this bijection is given by naturality of $\eta$,
	and naturality in $B$ is immediate.
\end{proof}

\begin{corollary}
	If $F,\, F'$ are both left-adjoint to $G$,
	then there's a canonical natural isomorphism $F \to F'$.
\end{corollary}
\begin{proof}
	For each $A$,
	$(FA, \eta_A)$ and $(F'A, \eta'_A)$ are both initial in $(A \downarrow G)$,
	so there's a unique isomorphism $\alpha_A: (FA, \eta_A) \to (F'A, \eta'_A)$.
	
	$\alpha$ is natural: given $f: A \to A'$,
	$\alpha_{A'}f$ and $(Ff)\alpha_A$ are both morphisms $(FA, \eta_A) \to (F'A', \eta'_{A'}f)$ in $(A \downarrow G)$.
	So they're equal.
\end{proof}

\begin{lemma}
	Given
	\begin{tikzcd}
		\mathcal{C} \arrow[r, "F", shift left] & \mathcal{D} \arrow[r, "H", shift left] \arrow[l, "G", shift left] & \mathcal{E} \arrow[l, "K", shift left]
	\end{tikzcd},
	if $F \dashv G$ and $H \dashv K$ then $HF \dashv GK$.
\end{lemma}
\begin{proof}
	We have bijections
	$$\mathcal{E}(HFA, C) \cong \mathcal{D}(FA, KC) \cong \mathcal{C}(A, GKC)$$
	which are natural in $A$ and $C$.
\end{proof}

\begin{corollary}
	Given a commutative square
	\begin{tikzcd}
		\mathcal{C} \arrow[r, "F"]  \arrow[d, "G"] & \mathcal{D} \arrow[d, "H"] \\
		\mathcal{E} \arrow[r, "K"] & \mathcal{F}
	\end{tikzcd}
	of categories and functors,
	suppose all the functors in the diagram have left adjoints.
	Then the diagram
	\begin{tikzcd}
		\mathcal{F} \arrow[r] \arrow[d] & \mathcal{E} \arrow[d] \\
		\mathcal{D} \arrow[r] & \mathcal{C}
	\end{tikzcd}
	of left adjoints commutes up to natural isomorphism.
\end{corollary}

Given $F \dashv G$,
we have a natural transformation $\eta: 1_\mathcal{C} \to GF$
defined as in 3.2. We call $\eta$ the \textbf{unit} of the adjunction.

Dually, we have $\epsilon: FG \to 1_\mathcal{D}$, the \textbf{counit}.
$\epsilon_B: FGB \to B$ corresponds to $1_{GB}: GB \to GB$.

\begin{theorem}
	Suppose we're given $F: \mathcal{C} \to \mathcal{D}$ and $G: \mathcal{D} \to \mathcal{C}$.
	Specifying an adjunction $F \dashv G$ is equivalent to specifying natural transformations
	$\eta: 1_\mathcal{C} \to GF$ and $\epsilon: FG \to 1_\mathcal{D}$ such that
	\begin{center}
	\begin{tikzcd}
		F \arrow[r, "F\eta"] \arrow[rd, "1_F"] & FGF \arrow[d, "\epsilon_F"] & \text{and} & G \arrow[r, "\eta_G"] \arrow[rd, "1_G"]  & GFG \arrow[d, "G\epsilon"] \\
		                                       &  F                          &            &                                          &  G
	\end{tikzcd}
	\end{center}
	commute.
	(We say $\eta$ and $\epsilon$ satisfy the \textbf{triangular identities}).
\end{theorem}
\begin{proof}
	Given $F \dashv G$, we define $\eta$ and $\epsilon$ as already described.
	Since $\epsilon_{FA}: FGFA \to FA$ corresponds to $1_{GFA}$,
	the composite $\epsilon_{FA}(F\eta_A)$ corresponds to
	\begin{tikzcd}A \arrow[r, "\eta_A"] & GFA \arrow[r, "1_{GFA}"] & GFA \end{tikzcd},
	so it must be $1_{FA}$.
	
	Similarly for the other identity.
	
	Conversely, given $\eta$ and $\epsilon$ satisfying the $\triangle^r$ identities,
	we map $f: A \to GB$ to the composite
	\begin{tikzcd}FA \arrow[r, "Ff"] & FGB \arrow[r, "\epsilon_B"] & B\end{tikzcd}
	and $g: FA \to B$ to the composite
	\begin{tikzcd}A \arrow[r, "\eta_A"] & GFA \arrow[r, "Gg"] & GB\end{tikzcd}.
	
	We have
	\begin{align*}
	\Phi(\begin{tikzcd}[ampersand replacement=\&] A \arrow[r, "f"]\& GB\end{tikzcd}) &=
		\begin{tikzcd}[ampersand replacement=\&]FA \arrow[r, "Ff"] \& FGB \arrow[r, "\epsilon_B"] \& B\end{tikzcd} \\
	\Psi(\begin{tikzcd}[ampersand replacement=\&] FA \arrow[r, "g"]\& B\end{tikzcd}) &=
		\begin{tikzcd}[ampersand replacement=\&]A \arrow[r, "\eta_A"] \& GFA \arrow[r, "Gg"] \& GB\end{tikzcd}
	\end{align*}
	
	So
	\begin{align*}
		\Psi\Phi(f) & =
			\begin{tikzcd}[ampersand replacement=\&] A \arrow[r, "\eta_A"] \& GFA \arrow[r, "GFf"] \& GFGB \arrow[r, "G\epsilon_B"] \& GB\end{tikzcd} \\
			&= \begin{tikzcd}[ampersand replacement=\&] A \arrow[r, "f"] \& GB \arrow[r, "\eta_{GB}"] \& GFGB \arrow[r, "G\epsilon_B"] \& GB\end{tikzcd} \\
			&= f
	\end{align*}
	
	And dually $\Phi\Psi(g) = g$.
	
	Naturality of $\Phi$ in $A$ is immediate from its definition,
	and naturality in $B$ follows from that of $\epsilon$.
\end{proof}

\begin{lemma}
	Suppose given
	\begin{tikzcd}
		\mathcal{C} \arrow[r, "F", shift left] & \mathcal{D} \arrow[l, "G", shift left]
	\end{tikzcd}
	and natural isomorphisms $\alpha: 1_\mathcal{C} \to GF$,
	$\beta: FG \to 1_\mathcal{D}$.
	Then there exist natural isomorphisms $\alpha',\,\beta'$ 
	which additionally satisfy the triangular identities.
	In particular ($F \dashv G$).
\end{lemma}
\begin{proof}
	We define $\alpha' = \alpha$ and take $\beta'$ to be the composite
	\begin{center}
		\begin{tikzcd}
			FG \arrow[r, "\beta_{FG}^{-1}"] & FGFG \arrow[r, "F_{\alpha_G}^{-1}"] & FG \arrow[r, "\beta"] & 1_\mathcal{D}
		\end{tikzcd}
	\end{center}
	Note that, since
	\begin{tikzcd}
		FGFG \arrow[r, "FG\beta"] \arrow[d, "\beta_{FG}"] & FG \arrow[d, "\beta"] \\
		FG \arrow[r, "\beta"] & 1_\mathcal{D}
	\end{tikzcd}
	commutes and $\beta$ is monic,
	we have $FG\beta = \beta_FG$.
	
	Similarly, $GF\alpha = \alpha_{GF}: GF \to GFGF$.
	
	Now
	\begin{align*}
		\beta_F' \circ F_{\alpha'} &=
			\begin{tikzcd}[ampersand replacement=\&]F \arrow[r, "F\alpha"]\& FGF \arrow[r, "\beta_{FGF}^{-1}"]\& FGFGF \arrow[r, "F\alpha_{GF}^{-1}"]\& FGF \arrow[r, "\beta_F"] \& F\end{tikzcd} \\
		&= \begin{tikzcd}[ampersand replacement=\&]F \arrow[r, "\beta_F^{-1}"]\& FGF \arrow[r, "FGF\alpha"]\& FGFGF \arrow[r, "FGF\alpha^{-1}"]\& FGF \arrow[r, "\beta_F"] \& F\end{tikzcd} \\
		&= 1_F
	\end{align*}
	and
	\begin{align*}
	G\beta' \circ \alpha'_G &=
	\begin{tikzcd}[ampersand replacement=\&]G \arrow[r, "\alpha_G"]\& GFG \arrow[r, "GFG\beta^{-1}"]\& GFGFG \arrow[r, "GF^{-1}_{\alpha_G}"]\& GFG \arrow[r, "G\beta"] \& G\end{tikzcd} \\
	&= \begin{tikzcd}[ampersand replacement=\&]G \arrow[r, "G\beta^{-1}"]\& GFG \arrow[r, "\alpha_{GFG}"]\& GFGFG \arrow[r, "\alpha_{GFG}^{-1}"]\& GFG \arrow[r, "\beta_F"] \& G\end{tikzcd} \\
	&= 1_G
	\end{align*}
\end{proof}

\begin{lemma}
	Suppose
	\begin{tikzcd}
		\mathcal{C} \arrow[r, "F", shift left] & \mathcal{D} \arrow[l, "G", shift left]
	\end{tikzcd},
	($F \dashv G$) is an adjunction with counit $\epsilon$.
	Then
	\begin{enumerate}[label=\roman*.]
		\item $\epsilon$ is (pointwise) epic $\iff$ $G$ is faithful
		\item $\epsilon$ is an isomorphism $\iff$ $G$ is full and faithful
	\end{enumerate} 
\end{lemma}
\begin{proof}
	\begin{enumerate}[label=\roman*.]
		\item Given $g: B \to B'$,
		the morphism $Gg: GB \to GB'$ corresponds to
		\begin{center}
			\begin{tikzcd} FGB \arrow[r, "\epsilon_B"] & B \arrow[r, "g"] & B' \end{tikzcd}
		\end{center}
		So, for fixed $B$,
		composition with $\epsilon_B$ is injective on morphisms $B \to B'$
		$\iff$ $(g \mapsto Gg)$ is injective on morphisms $B \to B'$.
		
		Hence $G$ is faithful $\iff \epsilon_B$ is epic $\forall B$.
		
		\item Similarly, $\epsilon_B$ is 0 $\forall B$
		$\implies G$ is bijective on morphisms with given domain and codomain,
		i.e. $G$ is full and faithful.
		
		Conversely, if $G$ is full and faithful,
		$1_{FGB}$ factors uniquely as \newline
		\begin{tikzcd}FGB \arrow[r, "\epsilon_B"] & B \arrow[r, "g"] & FGB\end{tikzcd},
		so $\epsilon_B$ is split monic.
		But it's epic by (i), hence an isomorphism.
	\end{enumerate}
\end{proof}

\begin{definition}
	\begin{enumerate}[label=\roman*.]
		\item A \textbf{reflection} is an adjunction satisfying the conditions of 3.8(ii).
		\item A \textbf{reflective} subcategory of $\mathcal{C}$ is a full subcategory $\mathcal{C}'$
		for which the inclusion $\mathcal{C}' \hookrightarrow \mathcal{C}$ has a left adjoint.
	\end{enumerate}
	
	Dually, \textbf{coreflection} and \textbf{coreflective} subcategory.
\end{definition}

\section{Limits}
\begin{definition}
	\begin{enumerate}[label=\alph*.]
		\item Let $J$ be a category (almost always small, often finite).
		A \textbf{diagram of shape J} in a category $\mathcal{C}$ is a functor $D: J \to \mathcal{C}$.
		
		E.g. if $J$ is the finite category
		\begin{tikzcd}
			\cdot \arrow[r] \arrow[d] \arrow[rd] & \cdot \arrow[d]\\
			\cdot \arrow[r] & \cdot
		\end{tikzcd}
		, a diagram of shape $J$ is a commutative square.
		If $J$ is the category
		\begin{tikzcd}
			\cdot \arrow[r] \arrow[d] \arrow[rd, shift left] \arrow[rd, shift right] & \cdot \arrow[d]\\
			\cdot \arrow[r] & \cdot
		\end{tikzcd}
		, a diagram of shape $J$ is a not-necessarily-commutative square.
		
		The objects $D(j),\,j \in \ob J$ are called \textbf{vertices} of $D$,
		and the morphisms $D(\alpha),\, \alpha \in \mor J$ are called \textbf{edges} of $D$.
		
		\item Let $D: J \to \mathcal{C}$ be a diagram in $\mathcal{C}$.
		A \textbf{cone over D} is a pair $(A,\, (\lambda_j \,|\, j \in \ob J))$ where $\lambda_j: A \to D(j)\ \forall j$,
		and
		\begin{tikzcd}[column sep=tiny]
			& A \arrow[ld, "\lambda_j"] \arrow[rd, "\lambda_{j'}"] & \\
			D(j) \arrow[rr, "D(\alpha)"]&& D(j')
		\end{tikzcd}
		commutes for each $\alpha: j \to j'$ in $J$.
		
		$A$ is called the \textbf{apex} of the cone,
		and the $\lambda_j$ are its \textbf{legs}.
		
		Equivalently, $\lambda$ is a natural transformation $\triangle A \to D$,
		where $\triangle A$ is the \textbf{constant diagram} with all vertices A and all edges $1_A$.
		
		A \textbf{morphism}
		$f: (A, (\lambda_j)) \to (B, (\mu_j))$ of cones over $D$ is a morphism $f: A \to B$ s.t.
		\begin{tikzcd}[column sep=tiny]
		A \arrow[rr, "f"] \arrow[rd, "\lambda_j"] && B \arrow[ld, "\mu_j"]\\
		& D(j) &
		\end{tikzcd}
		commutes for each $j$.
		We have a category \textbf{Cone}($D$) of cones over $D$.
		
		Note that $A \mapsto \triangle A$ is a functor $\mathcal{C} \to [J, \mathcal{C}]$ and
		\textbf{Cone}(D) is in fact the category $(\triangle \downarrow D)$.
		
		A cocone over $D: J \to \mathcal{C}$ is a cone over $D: J^{op} \to \mathcal{C}^{op}$.
		We write \textbf{Cocone}$(D)$ for the category of cocones over D.
	\end{enumerate}
\end{definition}
\begin{definition}
	\begin{enumerate}[label=\roman*.]
		\item A \textbf{limit} (resp. \textbf{colimit}) for a diagram $D: J \to \mathcal{C}$ is a terminal object of \textbf{Cone}(D) (respectively an initial object of \textbf{Cocone}(D)).
		\item We say $\mathcal{C}$ has limits (resp. colimits) of shape $J$ if $\triangle: \mathcal{C} \to [J, \mathcal{C}]$ has a right (resp. left) adjoint.
		
		(This is equivalent to making a choice of limit (resp. colimit) for every diagram of shape J).
	\end{enumerate}
\end{definition}

\begin{definition}[Pullback]
	Let $J$ be
	\begin{tikzcd}& \cdot \arrow[d] \\ \cdot \arrow[r] & \cdot\end{tikzcd}.
	A diagram of shape J looks like 
	\begin{tikzcd}& A \arrow[d, "f"] \\ B \arrow[r, "g"] & C\end{tikzcd}.
	A cone over it consists of
	\begin{tikzcd}D \arrow[r, "h"] \arrow[rd, "l"] \arrow[d, "k"] & A \\ C & B\end{tikzcd}
	satisfying $fh=l=gk$.
	Equivalently, it's a pair
	\begin{tikzcd}D \arrow[r, "h"] \arrow[d, "k"] & A \\ C &\end{tikzcd}
	completing the diagram to a commutative square.
	
	A universal such pair is called a \textbf{pullback} (or \textbf{fibre product});
	in $\Set$ it can be defined as $\{(a,b) \in A \times B \,|\, f(a)=g(b)\}$.
	A colimit of shape $J^{op}$ is called a \textbf{pushout}.
\end{definition}

\begin{theorem}
	Let $\mathcal{C}$ be a category.
	\begin{enumerate}[label=\roman*.]
		\item If $\mathcal{C}$ has equalisers and all finite (resp. all small) products,
		then $\mathcal{C}$ has all finite (resp. all small) limits.
		\item If $\mathcal{C}$ has pullbacks and a terminal object, 
		then $\mathcal{C}$ has all finite limits.
	\end{enumerate}
\end{theorem}
\begin{proof}
	\begin{enumerate}[label=\roman*.]
	\item 
	Given $D: J \to \mathcal{C}$, first form the products
	\begin{center}
	\begin{tabular}{ccc}
	$P = \prod_{j \in \ob J} D(j)$ & and & $Q = \prod_{\alpha \in \mor J} D(\cod \alpha)$
	\end{tabular}
	\end{center}
	
	Define
	\begin{tikzcd}P \ar[r, "f", shift left] \ar[r, "g" below, shift right] & Q\end{tikzcd}
	by $\pi_\alpha f = \pi_{\cod \alpha}: P \to D(\cod \alpha)$ and
	$\pi_\alpha g = D(\alpha) \circ \pi_{\dom \alpha}: P \to D(\dom \alpha) \to D(\cod \alpha)$,
	and let $e:E \to P$ be the equaliser of $(f, g)$.
	
	Claim $(E, (\pi_je \,|\, j \in \ob J))$ is a limit cone for $D$.
	It is a cone since, for any $\alpha: j \to j'$,
	$D(\alpha)\pi_je = \pi_\alpha ge = \pi_\alpha fe = \pi_{j'} e$.
	
	Given any cone $(C, (\lambda_j \,|\, j \in \ob J))$,
	the $\lambda_j$ define a unique $\lambda: C \to P$,
	and $f\lambda = g\lambda$ since $\pi_\alpha f\lambda = \pi_\alpha g \lambda \ \forall \alpha$.
	So $\lambda$ factors uniquely through $e$.
	
	\item
	Let 1 be a terminal object of $\mathcal{C}$.
	For any pair of objects $(A, B)$ the pullback of
	\begin{tikzcd}& A \ar[d]\\ B \ar[r]& 1\end{tikzcd}
	has the universal property of a product $A \times B$,
	so $\mathcal{C}$ has binary products.
	Then we can define any finite product $\prod_{i=1}^n A_i$ as $(((A_1 \times A_2)\times A_3) \times \dots) \times A_n$.
	
	So we need to show $\mathcal{C}$ has equalisers.
	Given \begin{tikzcd}A \ar[r, "f", shift left] \ar[r, "g" below, shift right] & B\end{tikzcd},
	consider the pullback of
	\begin{tikzcd}
		& B \ar[d, "{(1_A, f)}"] \\
		A \ar[r, "{(1_A, g)}"] & A \times B
	\end{tikzcd}.
	
	It consists of
	\begin{tikzcd}P \ar[r, "h"] \ar[d, "k"]& B \\ A\end{tikzcd}
	satisfying $1_Ah=1_Ak$ and $fh = gk$,
	and universal among such.
	
	But this forces $h=k$, and $h$ has the universal property of an equaliser for $(f, g)$.
	So by (i), $\mathcal{C}$ has all finite limits.
	\end{enumerate}
\end{proof}

\begin{definition}
	Let $F: \mathcal{C} \to \mathcal{D}$ be a functor.
	\begin{enumerate}[label=\alph*.]
		\item We say $F$ \textbf{preserves} limits of shape $J$ if,
		given $D: J \to \mathcal{C}$ and a limit cone $(L, (\lambda_j: j \in \ob J))$ for $D$,
		the cone $(FL, (F\lambda_j : j \in \ob J))$ is a limit for $FD: J \to \mathcal{D}$.
		\item We say $F$ \textbf{reflects} limits of shape $J$ if,
		given $D: J \to \mathcal{C}$ and a cone $(L, (\lambda_j))$ such that $(FL, (F\lambda_j))$ is a limit for $FD$,
		then $(L, (\lambda_j))$ is a limit for $D$.
		\item We say $F$ \textbf{creates} limits of shape $J$ if,
		given $D: J \to \mathcal{C}$ and a limit $(M, (\mu_j))$ for $FD$,
		there exists a cone $(L, \lambda_j)$ over $D$ whose image is isomorphic to $(M, (\mu_j))$,
		and any such cone is a limit for $D$.
	\end{enumerate}
\end{definition}

\begin{lemma}
	Suppose $\mathcal{D}$ has limits of shape $J$.
	Then $[\mathcal{C}, \mathcal{D}]$ has limits of shape $J$,
	and they're constructed pointwise
	(i.e. the forgetful functor $[\mathcal{C}, \mathcal{D}] \to \mathcal{D}^{\ob \mathcal{C}}$ creates them).
\end{lemma}
\begin{proof}
	Consider a functor $D: J \times \mathcal{C} \to \mathcal{D}$.
	For each $A \in \ob \mathcal{C}$,
	let $(LA, (\lambda_{j,A}: LA \to D(j, A) \,|\, j \in \ob J))$
	be a limit for the diagram $D(-, A): J \to \mathcal{D}$.
	
	Given any $f:A \to B$ in $\mathcal{C}$, the composites
	\begin{center}
	\begin{tikzcd}LA \ar{r}{\lambda_{j,A}} & D(j, A) \ar{r}{D(j,f)} & D(j, B)\end{tikzcd}
	\end{center}
	form a cone over $D(-, B)$,
	so they induce a unique $Lf: LA \to LB$ such that
	\begin{center}
		\begin{tikzcd}
			LA \ar{r}{Lf} \ar{d}{\lambda_{j, A}} & LB \ar{d}{\lambda_{j,B}}\\
			D(j,A) \ar{r}{D(j,f)} & D(j,B)
		\end{tikzcd}
	\end{center}
	commutes for all $j$.
	Uniqueness assures $L(gf)=L(g)L(f)$,
	so $L$ is a functor $\mathcal{C} \to \mathcal{D}$,
	and the $\lambda_{j,-}$ are natural transformations $L \to D(j, -)$.
	
	Suppose we're given any cone over $D$ in $[\mathcal{C}, \mathcal{D}]$ with apex $M$ and legs $\mu_j: M \to D(j, -)$.
	Then $(MA, (\mu_{j,A}: MA \to D(j, A) \,|\, j \in \ob J))$
	is a cone over $D(-, A)$ in $\mathcal{D}$,
	so we get a unique $\nu_A: MA \to LA$ s.t. $\lambda_{j,A}\nu_A = \mu_{j, A}$ for all $j$.
	
	Uniqueness tells us that
	\begin{center}
		\begin{tikzcd}
			MA \ar{r}{Mf} \ar{d}{\nu_A} & MB \ar{d}{\nu_B} \\
			LA \ar{r}{Lf} & LB
		\end{tikzcd}
	\end{center}
	commutes for all $f \in \mor \mathcal{C}$,
	so $\nu: M \to L$ in $[\mathcal{C}, \mathcal{D}]$,
	so it's the unique factorisation of the $\mu_{j,-}$ through the $\lambda_{j,-}$.
\end{proof}

\begin{lemma}
	A morphism $f: A \to B$ is monic $\iff$
	\begin{center}
	\begin{tikzcd}
		A \ar{r}{1_A} \ar{d}{1_A} & A \ar{d}{f}\\
		A \ar{r}{f} & B
	\end{tikzcd}
	\end{center}
	is a pullback.
\end{lemma}
\begin{proof}
	$f$ is monic $\iff$ any cone (g,h) over $(f, f)$ has $g=h$
	$\iff (g, h)$ factors uniquely through $(1_A, 1_A)$.
\end{proof}

Hence, provided $\mathcal{D}$ has pullbacks,
a morphism $\alpha: F \to G$ in $[\mathcal{C}, \mathcal{D}]$
is monic $\iff \alpha_A: FA \to GA$ is monic for each $A$.

\begin{theorem}
	If $G: \mathcal{D} \to \mathcal{C}$ has a left adjoint,
	then $G$ preserves all limits which exist in $\mathcal{D}$.
	\label{410}
\end{theorem}
\begin{proof}
	Suppose $\mathcal{C}$ and $\mathcal{D}$ both have limits of shape J and let $(F \dashv G)$.
	The diagram
	\begin{center}
		\begin{tikzcd}
			\mathcal{C} \ar{r}{F} \ar{d}{\triangle} & \mathcal{D} \ar{d}{\triangle} \\
			{[J, \mathcal{C}]} \ar{r}{{[J, F]}} & {[J, \mathcal{D}]}
		\end{tikzcd}
	\end{center}
	commutes and $[J, F]$ has a right adjoint $[J, G]$.
	So by 3.5 the diagram of right adjoints
	\begin{center}
		\begin{tikzcd}
			{[J, \mathcal{D}]} \ar{r}{{[J, G]}} \ar{d}{\lim_J} & {[J, \mathcal{C}]} \ar{d}{\lim_J} \\
			\mathcal{D} \ar{r}{G} & \mathcal{C}
		\end{tikzcd}
	\end{center}
	commutes up to isomorphism,
	i.e. $G$ preserves limits of shape $J$.
\end{proof}
\begin{proof}
	Let $D:J \to \mathcal{D}$ be a diagram with limit $(L, (\lambda_j \,|\, j \in \ob J))$.
	Given a cone $(A, (\mu_j: A \to GD(j) \,|\, j \in \ob J))$ in $\mathcal{C}$,
	we get a cone $(FA, (\bar{\mu_j} : FA \to D(j) \,|\, j \in \ob J))$ in $\mathcal{D}$,
	and hence a unique $\bar{\nu}:FA \to L$ such that $\lambda_j\bar{\nu} = \bar{\mu_j}$ for all $j$.
	
	Then $\nu: A \to GL$ is the unique morphism such that $(G\lambda_j)\nu = \mu_j \forall j$.
\end{proof}

The `primeval' Adjoint Functor Theorem says that if $\mathcal{D}$ has and $G: \mathcal{D} \to \mathcal{C}$ preserves all limits,
then $G$ has a left adjoint.

This depends on two ideas:
\begin{lemma}
	$\mathcal{C}$ has an initial object $\iff 1_\mathcal{C}: \mathcal{C} \to \mathcal{C}$ has a limit.
	\label{411}
\end{lemma}
\begin{proof}
	Suppose $\mathcal{C}$ has an initial object 0.
	The morphisms $(0 \to A \,|\, A \in \ob \mathcal{C})$ form a cone over $1_\mathcal{C}$.
	If we had another, say $(L, (\lambda_A \,|\, A \in \ob \mathcal{C}))$,
	then $\lambda_0: L \to 0$ would make
	\begin{center}
		\begin{tikzcd}[column sep=tiny]
			L \ar{rr}{\lambda_0} \ar{rd}{\lambda_A} && 0 \ar{ld} \\
			& A &
		\end{tikzcd}
	\end{center}
	commute for all $A$, and it's the only morphism which does.
	
	Conversely, suppose $(I, (\lambda_A: I \to A \,|\, A \in \ob \mathcal{C}))$ is a limit for $1_\mathcal{C}$.
	
	If $f: I \to A$, then
	\begin{center}
		\begin{tikzcd}[column sep=tiny]
			I \ar{rr}{\lambda_I} \ar{rd}{\lambda_A} && I \ar{ld}{f}\\
			& A &
		\end{tikzcd}
	\end{center}
	commutes. In particular, $\lambda_A \lambda_I = \lambda_A$ for all $A$,
	so $\lambda_I = 1_I$ since both are factorisations of the limit cone through itself.
	So $f = \lambda_A$, and hence $I$ is initial.
\end{proof}

\begin{lemma}
	Suppose $\mathcal{D}$ has and $G: \mathcal{D} \to \mathcal{C}$ preserves limits of shape $J$.
	Then, for each $A \in \ob \mathcal{C}$,
	$(A \downarrow G)$ has limits of shape $J$ and the forgetful functor $(A \downarrow G) \to \mathcal{D}$ creates them.
	\label{412}
\end{lemma}
\begin{proof}
	Suppose given $D: J \to (A \downarrow G)$.
	Write $D(j)$ as $(UD(j), f_j: A \to GUD(j))$ for each $j$.
	Let $(L, (\lambda_j \,|\, j \in \ob J))$ be a limit for $UD$,
	then $(GL, (G\lambda_j \,|\, j \in \ob J))$ is a limit for $GUD$.
	But the $f_j$ form a cone over $GUD$ with apex $A$,
	so there's a unique $h: A \to GL$ such that
	\begin{center}
		\begin{tikzcd}
			A \ar{r}{h} \ar{rd}{f_j} & GL \ar{d}{G\lambda_j} \\
			& GUD(j)
		\end{tikzcd}
	\end{center}
	commutes for all $j$.
	So there's a unique lifting of the cone over $D$ in $(A \downarrow G)$.
	
	Suppose we're given a cone $((B, g), (\mu_j \,|\, j \in \ob J))$ over $D$.
	Then
	\begin{center}
		\begin{tikzcd}
			A \ar{r}{g} \ar{rd}{h} & GB \ar{d}{G_k} \\
			& GL
		\end{tikzcd}
	\end{center}
	commutes since both ways round are factorisations of $(f_j \,|\, j \in \ob J)$ through the limit $GL$.
\end{proof}

Combining \ref{412} and \ref{411} with \ref{33},
we've proved the primeval Adjoint Functor Theorem.
However, this requries $\mathcal{D}$ to have limits for diagrams `as big as $\mathcal{D}$ itself',
and the only such categories are preorders (c.f. Q6, sheet 2).

In practice, the most we can hope for is that $\mathcal{D}$ has all small limits.
We call such a $\mathcal{D}$ \textbf{complete}.

\begin{theorem}[General Adjoint Functor Theorem]
	Suppose that $\mathcal{D}$ is complete and locally small.
	Then a functor $G: \mathcal{D} \to \mathcal{C}$ has a left adjoint if and only if
	it preserves all small limits and satisfies the `solution set condition':
	for any $A \in \ob \mathcal{C}$,
	there is a set $\{f_i: A \to GB_i \,|\, i \in I\}$ of objects of $(A \downarrow G)$
	such that any $h: A \to GC$ factors as
	\begin{center}
		\begin{tikzcd}
			A \ar{r}{f_i} & GB_i \ar{r}{Gg} & Gc
		\end{tikzcd}
	\end{center}
	for some $i \in I$ and $g: B_i \to C$.
	\label{413}
\end{theorem}
\begin{proof}
	If $G$ has a left adjoint,
	then it preserves small limits by \ref{410},
	and $\{\eta_A: A \to GFA \}$ is a singleton solution set at $A$.
	
	Conversely, each $(A \downarrow G)$ is complete by \ref{412},
	and locally small since it admits a faithful functor to $\mathcal{D}$.
	So we need to show:
	if $\mathcal{A}$ is complete and locally small,
	and has a weakly initial set of objects $\{S_i \,|\, i \in I\}$,
	then $\mathcal{A}$ has an initial object.
	
	First form $P = \prod_{i \in I} S_i$: then $P$ is weakly initial.
	
	Now form the limit
	\begin{tikzcd}I \ar{r}{a} & P\end{tikzcd}
	of the diagram
	\begin{tikzcd}P \ar[shift left]{r} \ar{r} \ar[shift right]{r} & P \end{tikzcd}
	whose edges are all morphism $P \to P$ in $\mathcal{A}$.
	
	Claim $I$ is initial:
	it's weakly initial since it admits a morphism to $P$.
	
	Suppose we had
	\begin{tikzcd}I \ar[shift left]{r}{f} \ar[shift right]{r}[below]{g} & A \end{tikzcd}.
	Let $b: E \to I$ be an equaliser for $(f, g)$:
	then there exists $c: P \to E$.
	
	Now
	\begin{tikzcd}P \ar{r}{c} & E \ar{r}{b} & I \ar{r}{a} & P\end{tikzcd}
	is an edge of the diagram whose limit is $I$,
	but so is $1_P$;
	so $abca = 1_Pa = a$.
	But $a$ is monic, so $bca = 1_I$.
	So $b$ is (split) epic,
	and $f=g$.
	So all the $(A \downarrow G)$ have initial objects,
	hence by \ref{33} $G$ has a left adjoint.
\end{proof}

The Special Adjoint Functor Theorem imposes additional conditions on $\mathcal{C}$ and $\mathcal{D}$
which ensure that every functor $\mathcal{D} \to \mathcal{C}$ preserving small limits has a left adjoint.

\begin{definition}
	\begin{enumerate}[label=\alph*.]
		\item A \textbf{subobject} of an object $A$ is a monomorphism $A' \rightarrowtail A$.
		We write $\Sub_\mathcal{C}(A)$ for the full subcategory of $\mathcal{C}/A$ whose objects are subobjects of $A$:
		note that this category is a preorder.
		\item We say $\mathcal{C}$ is \textbf{well-powered} if each $\Sub_\mathcal{C}(A)$ is equivalent to a small category,
		i.e. up to isomorphism each object has only a set of subobjects.
		
		Dually, $\mathcal{C}$ is \textbf{well-copowered} if $\mathcal{C}^{op}$ is well-powered.
	\end{enumerate}
\end{definition}

\begin{lemma}
	Suppose given a pullback
	\begin{center}
		\begin{tikzcd}
			P \ar{r}{k} \ar{d}{h} & A \ar[tail]{d}{f} \\
			B \ar{r}{g} & C
		\end{tikzcd}
	\end{center}
	with $f$ monic.
	Then $h$ is monic.
	\label{416}
\end{lemma}
\begin{proof}
	Suppose
	\begin{tikzcd}D \ar[shift left]{r}{x} \ar[shift right]{r}[below]{y} & P\end{tikzcd}
	satisfy $hx = hy$.
	Then $fkx = fky = ghx = ghy$ and $f$ is monic so $kx = ky$.
	
	Now $x=y$ since both are factorisations of the same cone through the pullback.
\end{proof}

\begin{theorem}[Special Adjoint Functor Theorem]
	Suppose both $\mathcal{C}$ and $\mathcal{D}$ are locally small,
	and $\mathcal{D}$ is complete, well-powered and has a separating set.
	Then $G: \mathcal{D} \to \mathcal{C}$ has a left adjoint $\iff$ $G$ preserves all small limits.
\end{theorem}
\begin{proof}
	The forward implication is \ref{410} again.
	
	Conversely, we first show that $(A \downarrow G)$ has the properties we've assumed for $\mathcal{D}$:
	it's complete by \ref{412}, and locally small as in \ref{413}.
	It's well-powered since subobjects of $(B, f)$ in $(A \downarrow G)$ are in bijection with subobjects $B' \rightarrowtail B$
	such that $f$ factors through $GB' \rightarrowtail GB$.
	
	It has a coseparating set:
	if $\{S_i \,|\, i \in I \}$ is a coseparating set for $\mathcal{D}$,
	then $\{(S_i, f) \,|\, i \in I,\, f:A \to GS_i \}$ is a coseperating set for $(A \downarrow G)$,
	since if
	\begin{tikzcd}{(B, f)} \ar[shift left]{r}{g} \ar[shift right]{r}[below]{g'} & {(B', f')}\end{tikzcd}
	satisfies $g \neq g'$,
	there exists $h: B' \to S_i$ for some $i$ with $hg \neq hg'$,
	and then $h$ is a morphism $(B', f') \to (S_i, (Gh)f')$ in $(A \downarrow G)$.
	
	Now we show that if $\mathcal{A}$ is complete, locally small and well-powered and has a coseperating set,
	then it has an initial object.
	
	First form $P=\prod_{i \in I}S_i$,
	where $\{ S_i \,|\, i \in I\}$ is a coseparating set.
	
	Consider the diagram
	\begin{center}
		\begin{tikzcd}[row sep=tiny]
			P' \ar[tail]{rd}& \\
			P'' \ar[tail]{r}& P\\
			\vdots &\\
			P^{(n)} \ar[tail]{ruu}&
		\end{tikzcd}
	\end{center}
	whose edges are a representative set of subobjects of $P$.
	
	Form its limit
	\begin{center}
		\begin{tikzcd}[row sep=tiny]
			& P' \\
			I \ar{ur} \ar{r} \ar{rdd} & P'' \\
			& \vdots \\
			& P^{(n)}
		\end{tikzcd}
	\end{center}
	by the argument of \ref{416} the legs $I \to P^{(-)}$ are monic,
	so $I \rightarrowtail P$ is monic and it's the least subobject of $P$.
	
	Hence in particular $I$ has no proper subobjects,
	so any two maps
	\begin{tikzcd}I \ar[shift left]{r}{f} \ar[shift right]{r}[below]{g} & A\end{tikzcd}
	must be equal, since their equaliser is an isomorphism.
	
	Now given $A \in \mathcal{A}$, form the product $Q = \prod_{i,f:A \to S_i}S_i$.
	The canonical morphism $h: A \to Q$ defined by $\pi_{i,f}h=f$ is monic since the $S_i$ form a coseparating set.
	
	We also have $k: P \to Q$ defined by $\pi_{i, f}k = \pi_i$,
	and we can form the pullback
	\begin{center}
		\begin{tikzcd}
			I \ar{r} \ar[tail]{rd} & B \ar{r}{m} \ar[tail]{d}{l} & A \ar[tail]{d}{h} \\
			& P \ar{r}{k} & Q
		\end{tikzcd}
	\end{center}
	
	By \ref{416} $l$ is monic and hence isomorphic to an edge of the diagram defining $I$,
	so $I \rightarrowtail P$ factors through it.
	So there exists a morphism $I \to A$, hence $I$ is initial.
\end{proof}

\section{Monads}
Suppose given an adjunction
\begin{tikzcd}\mathcal{C} \ar[shift left]{r}{F} & \mathcal{D} \ar[shift left]{l}{G}\end{tikzcd},
$F \dashv G$.
How much of this can we describe purely in terms of $\mathcal{C}$?

We have the composite $T=GF: \mathcal{C} \to \mathcal{C}$,
and the unit $\eta: 1_\mathcal{C} \to T$.
We also have $G\epsilon_F : GFGF \to GF$,
which we'll denote $\mu: TT \to T$.

These satisfy the commutative diagrams
\begin{center}
	\begin{tikzcd}[row sep=tiny, column sep=tiny]
		T \ar{rr}{T\eta} \ar{rrdd}{1_T} && TT \ar{dd}{\mu} && T \ar{ll}{\eta_T} \ar{lldd}{1_T} && \text{and} && TTT \ar{rr}{T\mu} \ar{dd}{\mu_T} && TT \ar{dd}{\mu} \\
		\circbox{1} &  &&  &\circbox{2}&&&&& \circbox{3}\\
		&& T && && && TT \ar{rr}{\mu} && T
	\end{tikzcd}
\end{center}
from the $\triangle^r$ identities and naturality of $\epsilon$.

\begin{definition}
	A \textbf{monad} $\mathbb{T}=(T, \eta, \mu)$ on a category $\mathcal{C}$ consists of
	a functor $T: \mathcal{C} \to \mathcal{C}$
	and natural transformations $\eta: 1_\mathcal{C} \to T$,
	$\mu: TT \to T$ satisfying the commutative diagrams \textnormal{\circbox{1}, \circbox{2} and \circbox{3}}.
\end{definition}

\begin{definition}
	Let $\mathbb{T}$ be a monad on $\mathcal{C}$.
	A \textbf{$\boldsymbol{\mathbb{T}}$-algebra} is a pair $(A, \alpha)$
	where $A \in \ob \mathcal{C}$,
	and $\alpha: TA \to A$ satisfies
	\begin{center}
		\begin{tikzcd}[row sep=tiny, column sep=tiny]
			A \ar{rr}{\eta_A} \ar{rrdd}{1_A} && TA \ar{dd}{\alpha} && \text{and} && TTA \ar{rr}{T\alpha} \ar{dd}{\mu_A} && TA \ar{dd}{\alpha} \\
			\textnormal{\circbox{4}} && && &&& \textnormal{\circbox{5}} \\
			&& A &&&& TA \ar{rr}{\alpha}&& A
		\end{tikzcd}
	\end{center}
	
	A \textbf{homomorphism} $f: (A, \alpha) \to (B, \beta)$ of $\mathbb{T}$-algebras is a morphism $f: A \to B$ such that
	\begin{center}
		\begin{tikzcd}[row sep=tiny, column sep=tiny]
			TA \ar{rr}{Tf} \ar{dd}{\alpha}&& TB \ar{dd}{\beta}\\
			& \textnormal{\circbox{6}} \\
			A \ar{rr}{f} && B
		\end{tikzcd}
	\end{center}
	commutes.
	We write $\mathcal{C}^\mathbb{T}$ for the category of $\mathbb{T}$-algebras.
\end{definition}

\begin{lemma}
	The forgetful functor $G: \mathcal{C}^\mathbb{T} \to \mathcal{C}$ has a left adjoint $F$,
	and the adjunction $(F \dashv G)$ induces the monad $\mathbb{T}$.
\end{lemma}
\begin{proof}
	We define $FA = (TA, \mu_A)$ (which is an algebra by \circbox{2} and \circbox{3}),
	and $F(A \overset{f}{\to} B) = Tf$ (which is a homomorphism by naturality of $\mu$).
	
	Clearly $GF = T$ and $\eta: 1_\mathcal{C} \to GF$.
	
	We define $\epsilon: FG \to 1_{\mathcal{C}^\mathbb{T}}$ by
	$\epsilon_{(A, \alpha)} = \alpha: (TA, \mu_A) \to (A, \alpha)$
	(which is a homomorphism by \circbox{5}).
	
	The triangular identities for $\eta$ and $\epsilon$ follow from \circbox{4} and \circbox{1},
	so $(F \dashv G)$.
	
	Finally, $G_{\epsilon_{FA}} = \mu_A$ by the definitions of $FA$ and $\epsilon$,
	so the adjunction incudes $\mathbb{T}$.
\end{proof}

Note that if
\begin{tikzcd} \mathcal{C} \ar[shift left]{r}{F} & \mathcal{D} \ar[shift left]{l}{G} \end{tikzcd}
induces $\mathbb{T}$,
then so does
\begin{tikzcd} \mathcal{C} \ar[shift left]{r}{F} & \mathcal{D}' \ar[shift left]{l}{G/\mathcal{D}'} \end{tikzcd}
where $\mathcal{D}'$ is the full subcategory of objects of the form $FA$.
So in seeking to construct $\mathcal{D}$, we may require $F$ to be bijective on objects.
But then morphisms $FA \to FB$ in $\mathcal{D}$ correspond bijectively to morphisms $A \to GFB = TB$ in $\mathcal{C}$.

\begin{definition}
	Given a monad $\mathbb{T}$ on $\mathcal{C}$,
	the \textbf{Kleisi category} $\mathcal{C}_\mathbb{T}$ is defined by:
	$\ob \mathcal{C}_\mathbb{T} = \ob\mathcal{C}$,
	morphisms $A \textcolor{red}{\to} B$ in $\mathcal{C}_\mathbb{T}$ are morphisms $A \to TB$ in $\mathcal{C}$,
	the identity $A \textcolor{red}{\to} A$ is $A \overset{\eta_A}{\to} TA$,
	and the composite of $A \textcolor{red}{\overset{f}{\to}} B \textcolor{red}{\overset{g}{\to}} C$ is 
	$A \overset{f}{\to} TB \overset{Tg}{\to} TTC \overset{\mu_C}{\to} C$.
	
	We check
	\begin{align*}
	A \textcolor{red}{\overset{1_A}{\to}} A \textcolor{red}{\overset{f}{\to}} B &= A \overset{\eta_A}{\to} TA \overset{Tf}{\to} TTB \overset{\mu_B}{\to} B \\
	&= A \overset{f}{\to} TA \overset{\eta_{TB}}{\to} TTB \overset{\mu_B}{\to} B \\
	&= \textcolor{red}{f} \text{ by \textnormal{\circbox{2}}}
	\end{align*}
	\begin{align*}
	A \textcolor{red}{\overset{f}{\to}} B \textcolor{red}{\overset{1_B}{\to}} B &= A \overset{f}{\to} TB \overset{T_{\eta_B}}{\to} TTB \overset{\mu_B}{\to} B \\
	&= \textcolor{red}{f} \text{ by \textnormal{\circbox{1}}}
	\end{align*}

	Given $A\textcolor{red}{\overset{f}{\to}} B \textcolor{red}{\overset{g}{\to}} C \textcolor{red}{\overset{h}{\to}} D$,
	\begin{align*}
		\textcolor{red}{(hg)f} &= A \overset{f}{\to} TB \overset{Tg}{\to} TTC \overset{TTh}{\to} TTTD \overset{T\mu_D}{\to} TTD \overset{\mu_D}{\to} TD \\
		&= A \overset{f}{\to} TB \overset{Tg}{\to} TTC \overset{TTh}{\to} TTTD \overset{\mu_{TD}}{\to} TTD \overset{\mu_D}{\to} TD \text{ by \textnormal{\circbox{3}}} \\
		&= A \overset{f}{\to} TB \overset{Tg}{\to} TTC \overset{\mu_C}{\to} TC \overset{T_h}{\to} TTD \overset{\mu_D}{\to} TD \\
		&= \textcolor{red}{h(gf)}
	\end{align*}
\end{definition}

\begin{lemma}
	There exists an adjunction
	\begin{tikzcd}\mathcal{C} \ar[shift left]{r}{F} & \mathcal{C}_\mathbb{T} \ar[shift left]{l}{G}\end{tikzcd}
	inducing $\mathbb{T}$.
\end{lemma}
\begin{proof}
	We define $FA=A$ and $F(A \textcolor{red}{\overset{f}{\to}} B) = A \overset{f}{\to} B \overset{\eta_B}{\to} TB$.
	This clearly preserves identities, and
	\begin{align*}
		\textcolor{red}{(Fg)(Ff)} &= A \overset{f}{\to} B \overset{\eta_B}{\to} TB \overset{Tg}{\to} TC \overset{T\eta_C}{\to} TTC \overset{\mu_C}{\to} TC \\
		&= A \overset{f}{\to} B \overset{g}{\to} C \overset{\eta_C}{\to} TC \text{ by \circbox{1} and naturality of } \eta \\
		&= \textcolor{red}{F(gf)}
	\end{align*}
	
	We define $GA = TA$ and $G(A\textcolor{red}{\overset{f}{\to}} B) = TA \overset{Tf}{\to} TTB \overset{\mu_B}{\to} TB$.
	G preserves identities by \circbox{1} and
	\begin{align*}
		G(A \textcolor{red}{\overset{f}{\to}} B \textcolor{red}{\overset{g}{\to}} C) &= TA \overset{Tf}{\to} TTB \overset{TTg}{\to} TTTC \overset{T\mu_C}{\to} TTC \overset{\mu_C}{\to} TC \\
		&= TA \overset{Tf}{\to} TTB \overset{TTg}{\to} TTTC \overset{\mu_{TC}}{\to} TTC \overset{\mu_C}{\to} TC \text{ by \circbox{3}} \\
		&= TA \overset{Tf}{\to} TTB \overset{\mu_B}{\to} TB \overset{Tg}{\to} TTC \overset{\mu_C}{\to} TC \text{ by naturality of } \mu \\
		&= (Gg)(Gf)
	\end{align*}
	
	Clearly $GFA = TA$ and
	\begin{align*}
	GF(A \overset{f}{\to} B) &= TA \overset{Tf}{\to} TB \overset{T\eta_B}{\to} TTB \overset{\mu_B}{\to} T B \\
	&= Tf \text{ by \circbox{1}}
	\end{align*}
	so $GF = T$ and $\eta: 1_\mathcal{C} \to GF$.
	
	We define $FGA \textcolor{red}{\overset{\epsilon_A}{\to}} A$ to be $TA \overset{\eta_{TA}}{\to} TA$.
	To verify naturality of $\textcolor{red}{\epsilon}$, consider
	\begin{center}
		\begin{tikzcd}
			FGA \ar[color=red]{r}{FGf} \ar[color=red]{d}{\epsilon_A} & FGB \ar[color=red]{d}{\epsilon_B} \\
			A \ar[color=red]{r}{f} & B
		\end{tikzcd}
	\end{center}
	The top and right edges yield
	$$TA \overset{Tf}{\to} TTB \overset{\mu_B}{\to} TB \overset{\eta_{TB}}{\to} TTB \overset{1_{TTB}}{\to} TTB \overset{\mu_B}{\to} TB$$
	and the left and bottom yield
	$$TA \overset{1_{TA}}{\to} TA \overset{Tf}{\to} TTB \overset{\mu_B}{\to} TB$$
	
	For the $\triangle^r$ identities,
	$$GA \overset{\eta_{GA}}{\to} GFGA \overset{G\epsilon_A}{\to} GA = TA \overset{\eta_{TA}}{\to} TTA \overset{1_{TTA}}{\to} TTA \overset{\mu_A}{\to} TA = 1_{TA}$$
	and
	\begin{align*}
		FA \overset{F\eta_A}{\to} FGFA \overset{\epsilon_{FA}}{\to} FA &= A \overset{\eta_A}{\to} TA \overset{\eta_{TA}}{\to} TTA \overset{1_{TTA}}{\to} TTA \overset{\mu_A}{\to} TA \\
		&= A \overset{\eta_A}{\to} TA\ (= FA \overset{1_{FA}}{\to} FA)
	\end{align*}
	
	Finally, $G\epsilon_{FA} = TTA \overset{1_{TTA}}{\to} TTA \overset{\mu_A}{\to} TA = \mu_A$,
	so the adjunction induces the monad $\mathbb{T}$.
\end{proof}

\begin{theorem}
	Given a monad $\mathbb{T}$ on $\mathcal{C}$,
	let $\Adj(\mathbb{T})$ be the category whose objects are adjunctions $\mathcal{C}\underset{G}{\overset{F}{\rightleftarrows}} \mathcal{D}$ inducing $\mathbb{T}$,
	and whose morphisms $(\mathcal{C}\underset{G}{\overset{F}{\rightleftarrows}} \mathcal{D}) \to (\mathcal{C}\underset{G'}{\overset{F'}{\rightleftarrows}} \mathcal{D}')$
	are functors $K: \mathcal{D} \to \mathcal{D}'$ satisfying $KF=F'$ and $G'K=G$.
	
	Then the Kleisi category $\mathcal{C}_\mathbb{T}$ is initial in $\Adj(\mathbb{T})$,
	and the Eilenberg-Moore category $\mathcal{C}^\mathbb{T}$ is terminal.
\end{theorem}
\begin{proof}
	Given $(\mathcal{C}\underset{G}{\overset{F}{\rightleftarrows}} \mathcal{D})$ in $\Adj(\mathbb{T})$,
	we define the \textbf{Eilenberg-Moore comparison functor} $K: \mathcal{D} \to \mathcal{C}^\mathbb{T}$
	by $KB=(GB, G\epsilon_B)$
	(note that $G\epsilon_B$ is an algebra structure on $GB$:
	the unit condition \circbox{4} follows from a $\triangle^r$ identity,
	and \circbox{5} follows from the naturality of $\epsilon$).
	
	$K(B \overset{g}{\to} B') = Gg: (GB, G\epsilon_B) \to (GB', G\epsilon_{B'})$
	(a homomorphism since $\epsilon$ is natural).
	
	It's clear that $K$ is a functor,
	that $G^\mathbb{T}K=G$ and that $KFA = (GFA, G\epsilon_{FA}) = (TA, \mu_A) = F^\mathbb{T}A$ and $KF(A \overset{f}{\to} B) = Tf = F^\mathbb{T}f$.
	
	Uniqueness: suppose $\bar{K}$ also satisfies $G^\mathbb{T}\bar{K} = G$ and $\bar{K}F = F^\mathbb{T}$.
	Then $\bar{K}B$ is of the form $(GB, \beta_B)$ for some algebra structure $\beta_B$,
	and that $\beta_{FA} = \mu_A = G\epsilon_{FA}$ for all $A$.
	
	Given any $B$, consider the diagram
	\begin{center}
		\begin{tikzcd}
			GFGFGB \ar{r}{GFG\epsilon_B} \ar{d}{\mu_{GB}} & GFGB \ar{d}{\beta_B}\\
			GFGB \ar{r}{G\epsilon_B} & GB
		\end{tikzcd}
	\end{center}
	which must commute, since $G\epsilon_B$ is an algebra homomorphism.
	But it would also commute with $G\epsilon_B$ in place of $\beta_B$,
	and $GFG\epsilon_B$ is (split) epic,
	so $\beta_B = G\epsilon_B$.
	
	For the \textbf{Kleisi comparison functor} $K: \mathcal{C}_\mathbb{T} \to \mathcal{D}$,
	we define $KA=FA$,
	$K(A \textcolor{red}{\overset{f}{\to}} B) = FA \overset{Ff}{\to} FGFB \overset{\epsilon_{FB}}{\to} FB$.
	
	To verify this is functorial, consider
	\begin{align*}
		K(A \textcolor{red}{\overset{f}{\to}} B \textcolor{red}{\overset{g}{\to}} C) &=
			FA \overset{Ff}{\to} FGFB \overset{FGFg}{\to} FGFGFC \overset{FG\epsilon_{FC}}{\to} FGFC \overset{\epsilon_{FC}}{\to} FC \\
			&= FA \overset{Ff}{\to} FGFB \overset{FGFg}{\to} FGFGFC \overset{\epsilon_{FGFC}}{\to} FGFC \overset{\epsilon_{FC}}{\to} FC \\
			&= FA \overset{Ff}{\to} FGFB \overset{\epsilon_{FB}}{\to} FB \overset{Fg}{\to} FGFC \overset{\epsilon_{FC}}{\to} FC \\
			&= (K\textcolor{red}{g})(K\textcolor{red}{f})
	\end{align*}
	
	$GKA = GFA = TA = G_\mathbb{T}A$
	
	$GK(A \textcolor{red}{\overset{f}{\to}} B) = TA \overset{Tf}{\to} TTB \overset{\mu_B}{\to} TB = G_\mathbb{T}(\textcolor{red}{f})$
	
	And $KF_\mathbb{T}A = FA$,
	
	$KF_\mathbb{T}(A \overset{f}{\to} B) = \begin{tikzcd} FA \ar{r}{Ff} & FB \ar{r}{F\eta_B} \ar{rd}{1_{FB}} & FGFB \ar{d}{\epsilon_{FB}} \\ && FB\end{tikzcd}$
	
	So $K$ is a morphism of $\Adj(\mathbb{T})$.
	
	Uniqueness: suppose $\bar{K}$ is any other morphism $\mathcal{C}_\mathbb{T} \to \mathcal{D}$ in $\Adj(\mathbb{T})$.
	Then $\bar{K}A = FA = KA$ for all $A$;
	since $\bar{K}$ commutes with both the $F$s and the $G$s,
	we have $\bar{K}(\epsilon_A) = \epsilon_{FA}$.
	
	We can write $A \textcolor{red}{\overset{f}{\to}} B$ as
	$A \textcolor{red}{\overset{F_\mathbb{T}f}{\to}} F_\mathbb{T}G_\mathbb{T} \textcolor{red}{\overset{\epsilon_B}{\to}} B$,
	so $\bar{K}(\textcolor{red}{f}) = \bar{K}(\textcolor{red}{\epsilon_B})Ff=K(\textcolor{red}{f})$.
\end{proof}

The Kleisi category $\mathcal{C}_\mathbb{T}$ inherits coproducts from $\mathcal{C}$ if $\mathcal{C}$ has them,
but it has few other limits or colimits in general.

\begin{theorem}
	\begin{enumerate}[label=\roman*.]
		\item The forgetful functor $G: \mathcal{C}^\mathbb{T} \to \mathcal{C}$ creates all limits which exist in $\mathcal{C}$.
		\item If $T: \mathcal{C} \to \mathcal{C}$ preserves colimits of shape $J$,
		then $G: \mathcal{C}^\mathbb{T} \to \mathcal{C}$ creates them.
	\end{enumerate}
	\label{58}
\end{theorem}
\begin{proof}
	\begin{enumerate}[label=\roman*.]
		\item 
		Let $D: J \to \mathcal{C}^\mathbb{T}$ be a diagram,
		write $D(j) = (GD(j), \delta_j)$.
		
		Let $(L, (\lambda_j : L \to GD(j)))$ be a limit for $GD$.
		The composites $TL \overset{T\lambda_j}{\to} TGD(j) \overset{\delta_j}{\to} GD(j)$
		form a cone over $GD$,
		since the edges of $GD$ are algebra homomorphisms.
		
		So they induce a unique $l: TL \to L$ such taht
		\begin{center}
			\begin{tikzcd}
				TL \ar{r}{T\lambda_j} \ar{d}{l} & TGD(j) \ar{d}{\delta_j} \\
				L \ar{r}{\lambda_j}& GD(j)
			\end{tikzcd}
		\end{center}
		commutes for each $j$.
		
		$l$ is an algebra structure:
		$l\eta_L = l_L$ since both are factorisations of $(\lambda_j)$ through itself,
		and $lTl = l\mu_L$ since they're factorisations of the same cone through $L$.
		
		So $((L,l), (\lambda_j))$ is the unique lifting of $(L, (\lambda_j))$ to a cone over $D$ in $\mathcal{C}^\mathbb{T}$.
		
		Any cone over $D$ in $\mathcal{C}^\mathbb{T}$ factors uniquely through $L$,
		and the factorisation is an algebra homomorphism.
		
		\item
		Similarly, given $D:J \to \mathcal{C}^\mathbb{T}$ as before and a colimit $(L, (\lambda_j: GD(j) \to L))$ for $GD$,
		we get a unique $l: TL \to L$ making
		\begin{center}
			\begin{tikzcd}
				TGD(j) \ar{r}{T\lambda_j} \ar{d}{\delta_j} & TL \ar{d}{l} \\
				GD(j) \ar{r}{\lambda_j} & L
			\end{tikzcd}
		\end{center}
		commute, since $(TL, (T\lambda_j))$ is a colimit.
		The rest of the proof is similar to (i).
	\end{enumerate}
\end{proof}

\begin{definition}
	An adjunction $(\mathcal{C} \overset{F}{\underset{G}{\rightleftarrows}} \mathcal{D})$, $(F \dashv G)$,
	is \textbf{monadic} if the comparison functor 
	$K: \mathcal{D} \to \mathcal{C}^\mathbb{T}$
	is part of an equivalence,
	where $\mathbb{T}$ is the monad induced by $(F \dashv G)$.
	We also say $G: \mathcal{D} \to \mathcal{C}$ is monadic if it has a left adjoint and the adjunction is monadic.
\end{definition}

Note that $K$ preserves all limits which exist in $\mathcal{D}$,
since $G$ preserves them and $G^\mathbb{T}$ creates them.

\begin{lemma}
	Suppose given $(\mathcal{C} \overset{F}{\underset{G}{\rightleftarrows}} \mathcal{D})$, $(F \dashv G)$
	inducing a monad $\mathbb{T}$ on $\mathcal{C}$.
	Suppose, for each $\mathbb{T}$-algebra $(A, \alpha)$, the pair
	$FGFA \overset{F\alpha}{\underset{\epsilon_{FA}}{\rightrightarrows}} FA$
	has a coequaliser in $\mathcal{D}$.
	Then $K: \mathcal{D} \to \mathcal{C}^\mathbb{T}$ has a left adjoint $L$.
	\label{510}
\end{lemma}
\begin{proof}
	We define $L(A, \alpha) = \coeq(FGFA \parallelpair{F\alpha}{\epsilon_{FA}} FA)$.
	
	Given $(A, \alpha) \overset{f}{\to} (B, \beta)$, we get
	\begin{center}
		\begin{tikzcd}
			FGFA \ar{d}{FGFf} \ar[shift left]{r}{F\alpha} \ar[shift right]{r}[below]{\epsilon_{FA}}& FA \ar{r} \ar{d}{Ff} & L(A, \alpha) \ar[dashrightarrow]{d}{Lf}\\
			FGFB \ar[shift left]{r}{F\beta} \ar[shift right]{r}[below]{\epsilon_{FB}} & FB \ar{r}& L(B, \beta)
		\end{tikzcd}
	\end{center}
	So $\exists! Lf$ making the right hand square commute.
	Uniqueness ensures $L$ is functorial.
	
	Morphisms $L(A, \alpha) \to B$ in $\mathcal{D}$ correspond bijectively to morphisms
	$f: FA \to B$ such that $f(F\alpha) = f(\epsilon_{FA})$ and hence to morphisms
	$\bar{f}:A \to GB$ such that $f\bar{\alpha} = Gf|_{GFA} = G\epsilon_B \circ GF\bar{f}$,
	i.e. to algebra homomorphisms $(A, \alpha) \to (GB, G\epsilon_B) = KB$.
	
	So $(L \dashv K)$.
\end{proof}

\begin{definition}
	\begin{enumerate}[label=\alph*.]
		\item A parallel pair $A \overset{f}{\underset{g}{\rightrightarrows}} B$ is \textbf{reflexive}
		if $\exists r: B \to A$ such that $fr = gr = 1_B$.
		
		Note that $FGFA \parallelpair{F\alpha}{\epsilon_{FA}} FA$ is reflexive,
		with common splitting $FA \overset{F\eta_A}{\to} FGFA$.
		
		A \textbf{reflexive coequaliser} is the coequaliser of a reflexive pair.
		
		\item A \textbf{split coequaliser diagram} is a diagram
		\begin{center}
			\begin{tikzcd}
				A \ar[shift left]{r}{f} \ar[shift right]{r}[below]{g} & B \ar{r}{h} \ar[bend left=50]{l}{t} & C \ar[bend left=30]{l}{s}
			\end{tikzcd}
		\end{center}
		satisfying $hf=hg$, $hs=1_C$, $gt = 1_B$ and $ft = sh$.
		
		If these equations hold, $h$ \textbf{is} a coequaliser of $f$ and $g$:
		given $k: B \to D$ with $kf = kg$, we have $k=kgt=kft=ksh$,
		so $k$ factors through $h$, uniquely since $h$ is split epic.
		
		Note that \textbf{any} functor preserves split equalisers.
		
		\item Given $G: \mathcal{D} \to \mathcal{C}$,
		a pair $A \parallelpair{f}{g} B$ in $\mathcal{D}$ is \textbf{G-split}
		if there exists a split coequaliser
		\begin{center}
			\begin{tikzcd}
				GA \ar[shift left]{r}{Gf} \ar[shift right]{r}[below]{Gg} & GB \ar{r} \ar[bend left=50]{l}{t} & C \ar[bend left=30]{l}{s}
			\end{tikzcd}
		\end{center}
		in $\mathcal{C}$.
	\end{enumerate}
\end{definition}

The pair $FGFA \overset{F\alpha}{\underset{\epsilon_{FA}}{\rightrightarrows}} FA$ of \ref{510} is $G$-split:
\begin{center}
	\begin{tikzcd}
		GFGFA \ar[shift left]{r}{Gf\alpha} \ar[shift right]{r}[below]{G\epsilon_{FA}} & GFA \ar{r}{\alpha} \ar[bend left=50]{l}{\eta_{GFA}} & A \ar[bend left=30]{l}{\eta_A}
	\end{tikzcd}
\end{center}
is a split coequaliser diagram.

\begin{theorem}[Precise Monadicity Theorem]
	$G: \mathcal{D} \to \mathcal{C}$ is monadic $\iff$ $G$ has a left adjoint, and c reates coequalisers of $G$-split pairs.
	\label{512}
\end{theorem}

\begin{theorem}[Crude Monadicity Theorem]
	Suppose $G: \mathcal{D} \to \mathcal{C}$ has a left adjoint,
	that $\mathcal{D}$ has and $G$ preserves reflexive coequalisers,
	and $G$ reflects isomorphisms.
	Then $G$ is monadic.
	\label{513}
\end{theorem}

\begin{proof}
	For the forward implication in \ref{512},
	it's enough to show that $G^\mathbb{T}: \mathcal{C}^\mathbb{T} \to \mathcal{C}$ creates coequalisers of $G^\mathbb{T}$-split pairs.
	This follows from \ref{58},
	given that $T$ and $TT$ both preserve split coequalisers.
	
	Conversely in either case, $K: \mathcal{D} \to \mathcal{C}^\mathbb{T}$ has a left adjoint $L$ by \ref{510}.
	Now $LKB = \coeq(FGFGB \parallelpair{FG\epsilon_B}{\epsilon_{FGB}} FGB)$
	and the counit $LKB \to B$ is the factorisation of $FGB \overset{\epsilon_{FGB}}{\underset{\epsilon_B}{\rightrightarrows}} B$ through this coequaliser.
	
	But
	\begin{tikzcd}GFGFGB \ar[shift left]{r} \ar[shift right]{r}& GFGB \ar{r} \ar[bend left=30]{l}{\eta_{GFGB}} & GB \ar[bend left=30]{l}{\eta_{GB}}\end{tikzcd}
	is a split coequaliser diagram.
	
	So either set of hypotheses ensures that $LKB \to B$ is an isomorphism.
	
	$KL(A, \alpha) = K(\coeq(FGFA \parallelpair{F\alpha}{\epsilon_{FA}} FA))$.
	Either hypothesis implies that $G=G^\mathbb{T}K$ preserves this coequaliser, but
	\begin{center}
		\begin{tikzcd}
			GFGFGA \ar[shift left]{r}{GF\alpha} \ar[shift right]{r}[below]{G\epsilon_{FA}} & GFA \ar{r}{\alpha} \ar[bend left=50]{l}{\eta_{GFA}} & A \ar[bend left=30]{l}{\eta_{A}}
		\end{tikzcd}
	\end{center}
	is a split coequaliser, so $GL(A, \alpha) \cong A$.
	
	The unit $(A, \alpha) \to KL(A, \alpha)$ is mapped to this isomorphism by $G^\mathbb{T}$,
	so it's an isomorphism in $\mathcal{C}^\mathbb{T}$.
\end{proof}

\begin{remark}
	Let $\mathcal{C} \rightleft{F}{G} \mathcal{D}$ be an adjunction,
	and suppose $\mathcal{D}$ has reflexive coequalisers.
	The \textbf{monadic tower} of $(F \dashv G)$ is the diagram
	\begin{center}
		\begin{tikzcd}[column sep=small, row sep=tiny]
			&&&& \textcolor{white}{a}\\
		&&& (\mathcal{C}^\mathbb{T})^\mathbb{S} \ar[shift left]{ddl} \ar[shift left]{dlll} \ar[dotted, no head]{ur}\\
		\mathcal{D} \ar[shift left]{ddr}{G} \ar[shift left]{drr}{K} \ar[shift left]{urrr}\\
		&& \mathcal{C}^\mathbb{T} \ar[shift left]{dl}{G^\mathbb{T}} \ar[shift left]{uur} \ar[shift left]{ull}{L}\\[20pt]
		& \mathcal{C} \ar[shift left]{ur}{F^\mathbb{T}} \ar[shift left]{uul}{F}
		\end{tikzcd}
	\end{center}
	where $\mathbb{T}$ is the monad induced by $(F \dashv T)$, 
	$K$ is the Eilenberg-Moore comparison functor,
	$(L \dashv K)$ (\ref{510}),
	$\mathbb{S}$ is the monad induced by $(L \dashv K)$, etc.
	
	We say $(F \dashv G)$ has \textbf{monadic length} $n$ if we reach an equivalence after $n$ steps.
	$\text{\textbf{Top}} \to \Set$ has monadic length $\infty$.
\end{remark}

\section{Regular Categories}

\begin{definition}
	The \textbf{image} of a morphism $A \overset{f}{\to} B$ is the smallest subobject of $B$ through which $f$ factors, if this exists.
	
	We say $\mathcal{C}$ \textbf{has images} if every $f \in \mor \mathcal{C}$ has an image.
	
	$A \overset{f}{\to} B$ is a \textbf{cover} if its image is $1_B$,
	i.e. it doesn't factor through any proper subobject of $B$.
	We write $A \overset{f}{\rightarrowtriangle} B$ to indicate that $f$ is a cover.
\end{definition}

\begin{lemma}
	If $\mathcal{C}$ has finite limits,
	then covers in $\mathcal{C}$ coincide with strong epimorphisms.
\end{lemma}
\begin{proof}
	Recall that $f$ is strong epic if and only if given $(*)$
	\begin{tikzcd}
		A \ar{r}{g} \ar{d}{f} & C \ar[tail]{d}{k} \\
		B \ar{r}{h} \ar[dashed]{ur}{l} & D
	\end{tikzcd}
	with $k$ monic, $\exists l: B \to C$ with $kl = h$ and $lf = g$.
	
	Being a cover is the special case of this condition with $h=1_B$,
	so strong epimorphisms are covers.
	
	Conversely, if $f$ is a cover then it's epic,
	since if $gf = hf$ then $f$ factors through the equaliser of $g$ and $h$,
	so this must be an isomorphism. Given $(*)$, we can form the pullback
	\begin{tikzcd}
		P \ar{r}{n} \ar{d}{m} & C \ar[tail]{d}{k} \\
		B \ar{r}{h} & D
	\end{tikzcd},
	and $m$ is monic by \ref{416}.
	
	$f$ factors through $m$, so $m$ is an isomorphism and $B \overset{nm^{-1}}{\to} C$ is the diagonal fill in for $(*)$.
\end{proof}

It follows that image factorisation is functorial: given
\begin{center}
	\begin{tikzcd}[row sep=small]
		A \ar{rr}{f} \ar{ddd}{g}  \ar[Rightarrow]{rd}& & B \ar{ddd}{h}\\
		& I \ar[tail]{ur} \ar[dashed]{d}& \\
		& I' \ar[tail]{rd}& \\
		A' \ar[Rightarrow]{ur} \ar{rr}[below]{f'} && B'
	\end{tikzcd}
\end{center}
we get a unique $I \to I'$ making everything commute.
So image factorisation defines a functor $[\mathbf{2}, \mathcal{C}] \to [\mathbf{3}, \mathcal{C}]$.

\begin{definition}
	A \textbf{regular category} is a category which preserves finite limits and images,
	in which strong epimorphisms are stable under pullback.
	
	A \textbf{regular functor} is one which preserves finite limits and strong epimorphisms.
\end{definition}

\begin{theorem}
	In a regular category, the strong epimorphisms coincide with regular epimorphisms.
	\label{65}
\end{theorem}
\begin{proof}
	Regular $\implies$ strong is true in general (see sheet 1).
	Suppose $A \overset{f}{\rightarrowtriangle} B$ is strong epic.
	Let $R \parallelpair{a}{b} A$ be the \textbf{kernel-pair} of $f$,
	i.e. the pullback of $f$ against itself. Certainly $fa = fb$.
	
	Suppose $A \overset{g}{\to} C$ with $ga = gb$. Form the image
	\begin{center}
		\begin{tikzcd}
			A \ar[Rightarrow]{r} & I \ar[tail]{r}{(k,l)} & B \times C
		\end{tikzcd}
	\end{center}
	of $(f, g)$.
	
	Claim $k$ is an isomorphism: given this,
	the composite $B \overset{k^{-1}}{\to} I \overset{l}{\to} C$ satisfies $lk^{-1}f = lk^{-1}kh = lh = g$,
	and it's unique since $f$ is epic.
	
	We know $k$ is strong epic, since $kh=f$ is strong epic.
	So we need to show it's monic.
	
	Suppose $D \parallelpair{x}{y}$ satisfy $kx=ky$. Form the pullback
	\begin{center}
		\begin{tikzcd}
			F \ar{rr}{m} \ar{d}{(n,p)} && D \ar{d}{(x,y)} \\
			A \times A \ar[Rightarrow]{r}{1 \times h}& A \times I  \ar[Rightarrow]{r}{h \times 1}& I \times I
		\end{tikzcd}
	\end{center}
	then $h \times 1$ and $1 \times h$ are strong epimorphisms, and hence so is $m$.
	
	Now $fn = khn = kxm = kym = khp = fp$, 
	so $(n,p)$ factors through $(a, b)$,
	say by $E \overset{q}{\to} R$.
	Since $kha = khb$ and $lha = lhb$, and $(k, l)$ is monic, 
	we have $ha = hb$.
	So $xm = hn = haq = hbq = hp = ym$.
	But $m$ is epic, so $x=y$.
\end{proof}
\begin{remark}
	In many textbooks, regular categories are defined as categories with (some) finite limits,
	in which every morphism factors as a regular epimorphism and a monomorphism,
	and regular epimorphims are stable under pullback.
\end{remark}

Note that if $A \overset{f}{\to} B$ has kernel pair $R \parallelpair{a}{b} A$ and image factorisation
$A \overset{g}{\rightarrowtriangle} I \rightarrowtail B$,
then $(a, b)$ is also the kernel-pair of $g$ since $h$ is monic.
So $g$ may be obtained as the coequaliser of the kernel-pair of $f$,
and $f$ as the factorisation of $f$ through this.

\begin{definition}
	Let $R \parallelpair{a}{b} A$ be a parallel pair in a category with finite limits.
	\begin{enumerate}[label=\alph*.]
		\item $(a, b)$ is a \textbf{relation} if $R \overset{(a,b)}{\longrightarrow} A \times A$ is monic.
		\item $(a,b)$ is \textbf{reflexive} if there exists $A \overset{r}{\to} R$ with $ar = br = 1_A$.
		\item $(a, b)$ is \textbf{symmetric} if there exists $R \overset{s}{\to} R$ with $as = b$, $bs = a$.
		\item $(a, b)$ is \textbf{transitive} if, given the pullback
			\begin{tikzcd}
				T \ar{r}{q} \ar{d}{p} & R \ar{d}{a} \\
				R \ar{r}{b} & A
			\end{tikzcd},
			there exists $t: T \to R$ such that $at = ap$ and $bt = bq$.
		\item $(a, b)$ is an \textbf{equivalence relation} if all of (a-d) hold.	
	\end{enumerate}
	The kernel-pair of any $A \overset{f}{\to} B$ is an equivalence relation.
	
	$(a, b)$ is an \textbf{effective} equivalence relation if it occurs as a kernel pair,
	$\mathcal{C}$ is an \textbf{effective} regular category if all equivalence relations in $\mathcal{C}$ are effective (aka 'Barr-exact').	
\end{definition}

\begin{definition}
	The \textbf{support} $\sigma(A)$ of an object $A$ in a regular category is the image of $A \to 1$.
	$A$ is \textbf{well-supported} if $\sigma(A) \cong 1$,
	i.e. if $A \rightarrowtriangle 1$ is strong epic.
	
	$\mathcal{C}$ is \textbf{totally supported} if all its objects are well-supported.
	(E.g. \textbf{Gp} and \textbf{ApGp} are totally supported,
	since any $A \to 1$ is split epic).
	
	An object $0$ in a regular category is \textbf{strict} if any morphism $A \to 0$ is an isomorphism.
	(This implies that $0$ is initial: $O \times A \overset{\pi_1}{\to} 0$ is an isomorphism,
	so $0 \overset{\pi_1^{-1}}{\to} 0 \times A \overset{\pi_2}{\to} A$ exists for any $A$,
	but given $O \parallelpair{f}{g} A$,
	the equaliser must be an isomorphism.)
	
	$\mathcal{C}$ is \textbf{almost totally supported} if every object of $\mathcal{C}$ is either well-supported or strict (e.g. \textbf{Set}).
\end{definition}

\begin{theorem}[Barr's Embedding Theorem]
	Let $\mathcal{C}$ be a small regular category.
	Then there exists a small category $\mathcal{D}$ and a full and faithful regular functor $\mathcal{C} \to [\mathcal{D}, \Set]$.
	Moreover, if $\mathcal{C}$ is almost totally supported then $\mathcal{D}$ can be taken to be a monoid.
\end{theorem}

We'll prove the most important part of this:
$\mathcal{C}$ has an isomorphism reflecting regular functor to be a power of $\Set$.
We follow a proof due to F. Borceaux:

\begin{theorem}
	Let $\mathcal{C}$ be a small a.t.s. regular category.
	Then there exists an isomorphism reflecting regular functor $F: \mathcal{C} \to \Set$.
	\label{610}
\end{theorem}
\begin{proof}
	We construct $F$ as the colimit in $[\mathcal{C}, \Set]$ of a diagram of representable functors:
	explicitly, $J$ will be a meet-semilattice and $D: J \to \mathcal{C}$ a diagram such that
	each $D(j)$ is well-supported and each $D(j' \to j)$ is a strong epimorphism $D(j') \rightarrowtriangle D(j)$.
	
	Then $F = colim(J^op \overset{D}{\to} \mathcal{C}^op \overset{Y}{\to} [\mathcal{C}, \Set])$.
	
	Explicitly, elements of $FA$ are represented by morphisms $D(j) \overset{f}{\to} A$ for some $j$,
	where $f \sim f'$ if and only if
	\begin{center}
		\begin{tikzcd}[row sep=tiny]
			& D(j) \ar{rd}{f} & \\
			D(j \wedge j') \ar{ru} \ar{rd} && A \\
			& D(j') \ar{ru}{f'} &
		\end{tikzcd}
	\end{center}
	commutes.
	
	$F$ preserves finite products: $F1 = \{*\}$,
	and if $D(j) \overset{f}{\to} A$, $D(j') \overset{g}{\to} B$ represent elts of $FA$ and $FB$,
	$D(j \wedge j') \to D(j) \overset{f}{\to} A$ and
	$D(j \wedge j') \to D(j) \overset{g}{\to} B$
	induce an element of $F(A \times B)$,
	mapping to the given element of $FA \times FB$.
	
	Hence $F(A \times B) \to FA \times FB$ is surjective,
	and it's easily seen to be injective.
	
	$F$ preserves equalisers:
	notes that if $0$ exists in $\mathcal{C}$,
	then $F0 = \emptyset$.
	If $$E \overset{e}{\rightarrowtail} A \parallelpair{f}{g} B$$ is an equaliser diagram in $\mathcal{C}$ and $E$ is well-supported,
	then the equaliser of $FA \parallelpair{}{} FB$ consists of morphisms $D(j) \to A$ having equal composites with $f$ and $g$ (and hence factoring through $E$).
	And if $E=0$ then the equaliser of $FA \parallelpair{}{} FB$ is $\emptyset$.
	
	Now assume that, for every strong epimorphism $A \overset{f}{\to} D(j)$ in $\mathcal{C}$,
	there exists $j' \leq j$ such that $D(j' \to j) = f$.
	
	Then $F$ preserves strong epimorphisms:
	given $A \overset{f}{\rightarrowtriangle} B$ and a morphism $D(j) \overset{g}{\to}$ representing an element of $FB$,
	form the pullback
	\begin{center}
		\begin{tikzcd}
			D(j') \ar{r}{h} \ar{d} & A \ar[Rightarrow]{d}{f} \\
			D(j) \ar{r}{g} & B
		\end{tikzcd}
	\end{center}
	then $h$ represents an element of $FA$ whose image under $Ff$ is $g$.
	So $Ff$ is surjective.
	
	Assume every well-supported object of $\mathcal{C}$ occurs as $D(j)$ for some $j$.
	Then $F$ preserves properness of subobjects:
	the element of $FA$ represented by $D(j) \overset{1}{\to} A$ can't be in the image of $FA' \rightarrowtail FA$ for any proper subobject $A' \rightarrowtail A$
	(if it were, we'd have
	\begin{tikzcd}
		D(j') \ar{r} \ar[Rightarrow]{d} & A' \ar[tail]{d} \\
		D(j) \ar[dashed]{ur} \ar{r}{1} & A
	\end{tikzcd})
	
	Since $F$ preserves equalisers, it follows that it's faithful.
	Hence $F$ reflects monomorphisms,
	and so the argument above shows that it reflects isomorphisms.
	
	We'll construct $J$ as the union $\bigcup_{n=0}^\infty J_n$ of an increasing sequence of sub-semilattices $J_n$:
	$J_0 = \{1\}$ and $D(1) = 1$,
	the terminal object of $\mathcal{C}$.
	Objects of $J_1 \backslash J_0$ are non-empty finite sets $\{A_1, A_2, \dots, A_n \}$ of well-supported objects of $\mathcal{C}$,
	ordered by  $\supseteq$ (so $j \wedge j' = j \cup j'$),
	and $D(\{A_1, A_2, \dots, A_n \}) = \prod_{i=1}^n A_i$.
	(Note that is well-supported: if $A$ and $B$ are well supported, we have a pullback
	\begin{tikzcd}
		A \times B \ar[Rightarrow]{r} \ar[Rightarrow]{d} & A \ar[Rightarrow]{d} \\
		B \ar[Rightarrow]{r} & 1
	\end{tikzcd})
	
	If $j' \supseteq j$, $D(j \to j')$ is the product projection $\prod_{A \in j'} \rightarrowtriangle \prod_{A \in j} A$.
	
	An object of $J_2 \backslash J_1$ is a pair $(j_1, \{f_1, \dots, f_n\})$
	where $j_1 \in J_1 \backslash J_0$ and $\{f_1, \dots, f_n\}$ is a non-empty finite set of strong epimorphisms with codomain $D(j_1)$.
	(Equivalently, well-supported objects of $\mathcal{C}/D(j_1)$.
	
	The meet of $(j_1, \{f_1, \dots, f_n\})$ and $(j_1', \{g_1, \dots g_m\})$ has first coordinate $j_1 \wedge j_1'$,
	and then we take the union of the sets of strong epimorphisms obtained from pulling back the $f_i$ and $g_j$
	along $D(j_1) \wedge j_1' \rightarrowtriangle D(j_1)$ and $D(j_1 \wedge j_1') \rightarrowtriangle D(j_1')$.
	
	We define $D((j_1, \{f_1, \dots, f_n\}))$ to be the domain of the object $\prod_{i=1}^n f_i$ of $\mathcal{C}/D(j_1)$,
	and $D(j_2 \wedge j_2')$ is the composite of the appropriate product projection in $\mathcal{C}/D(j_1 \wedge j_1')$
	with the appropriate pullback of $D(j_1 \wedge j_1') \rightarrowtriangle D(j_1)$.
	
	Similarly, object of $J_3 \backslash J_2$ are pairs $(j_2, \{h_1, \dots, h_n\})$
	where $j_2 \in J_2 \backslash J_1$ and the $h_i$ are strong epimorphisms with codomain $D(j_2)$,
	and so on.
	
	Now we've satisfied the condition for $F$ to preserve strong epimorphisms:
	if $A = D(j)$ where $j \in J_n \backslash J_{n-1}$,
	and $B \overset{g}{\rightarrowtriangle} A$ is strong epic,
	then $B = D((j, \{g\}))$ and $g = D((j, \{g\}) \to j)$.
\end{proof}

\begin{remark}
	If we define $M$ to be the monoid of endomorphisms of $F: \mathcal{C} \to \Set$ in $[\mathcal{C}, \Set]$,
	then $M$ acts on every $FA$, 
	so we can regard $F$ as taking values in $[M, \Set]$.
	As such, it's still regular and faithful, but also full.
	\label{611}
\end{remark}

Given a general regular category $\mathcal{C}$ and $S \rightarrowtail 1$ in $\mathcal{C}$,
we write $\mathcal{C}_S$ for the full subcategory of $\mathcal{C}$ on objects $A$ with $\sigma(A) \cong S$.
This is closed in $\mathcal{C}$ under non-empty finite products, images and pullbacks of strong epimorphisms
(if we're given
\begin{tikzcd}
	P \ar{r}{h} \ar[Rightarrow]{d}{k} & A \ar[Rightarrow]{d}{f}\\
	B \ar{r}{g} & C
\end{tikzcd}
with $f$ strong epic then $k$ is strong epic, so $\sigma(P) = \sigma(B)$.)
It doesn't have all finite limits,
but if $D: J \to \mathcal{C}_S$ is a finite diagram whose limit in $\mathcal{C}$ is not in $\mathcal{C}_S$,
then there are no cones over $D$ in $\mathcal{C}_S$.

\begin{definition}
	Given a category $\mathcal{C}$,
	let $\mathcal{C}^+$ denote the category whose objects are those of $\mathcal{C}$ plus a new object $0$,
	with one morphism $0 \to A$ and no morphisms $A \to 0$ for each $A \in \ob \mathcal{C}$.
\end{definition}

In $\mathcal{C}_S^+$, every finite diagram has a limit:
if it lies in $\mathcal{C}_S$ and has a limit there, that is its limit in $\mathcal{C}_S^+$,
otherwise its limit is the unique cone with apex 0.

$\mathcal{C}_S^+$ is regular: the new morphisms $0 \to A$ are monic,
and hence their own images,
and strong epimorphisms are still stable under pullback.

It's almost totally supported: note that $\mathcal{C}/S$ may be identified with the full subcategory of $\mathcal{C}$ on objects with support $\leq S$,
so its well-supported objects are those of $\mathcal{C}_S$.

We have a functor $E: \mathcal{C}/S \to \mathcal{C}_S^+$ sending all objects of $\mathcal{C}_S$ to themselves,
and everything else to 0.
$E$ is regular!
And we have a regular functor $(-) \times S : \mathcal{C} \to \mathcal{C}/S$ (it preserves finite limits because it's right adjoint to the forgetful functor,
and images because they're stable under pullback along $S \to 1$).

\begin{theorem}
	For every small regular category $\mathcal{C}$,
	there exists a set $I$ and an isomorphism reflecting regular functor $\mathcal{C} \to \Set^I$.
\end{theorem}
\begin{proof}
	Take $I = \Sub_\mathcal{C}(1)$,
	and for each $S \in \mathcal{I}$ consider the composite
	$$G_S \overset{(-) \times S}{\longrightarrow} \mathcal{C}/S \overset{E}{\to} \mathcal{C}_S^+ \overset{F_s}{\longrightarrow} \Set$$
	where $F_S$ is defined as in \ref{610}.
	The $G_S$ are all regular,
	and they jointly reflect isomorphisms:
	if $A \overset{f}{\to} B$ is not an isomorphism in $\mathcal{C}$,
	let $S = \sigma(B)$.
	Then $E(f)$ is either $f$ itself (if $\sigma(A)=S$)
	or $0 \to B$ (otherwise).
	In either case it's not an isomorphism, so its image under $F_S$ is not an isomorphism.
\end{proof}

\section{Additive and Abelian Categories}
\begin{definition}
	Let $\mathcal{A}$ be a category equipped with a forgetful functor $U: \mathcal{A} \to \Set$.
	A locally small category $\mathcal{C}$ is \textbf{enriched} over $\mathcal{A}$ if
	$$\mathcal{C}(-, -): \mathcal{C}^{op} \times \mathcal{C} \to \Set$$
	factors through $U$.
	
	If $\mathcal{A} = \Set_*$,
	$\mathcal{C}$ is called a \textbf{pointed category}
	(i.e. we have a distingushed morphism $A \overset{0}{\to} B$) for each pair $(A, B)$ satisfying $0f = 0$ and $g0 = 0$).
	
	If $\mathcal{A} = \CMon$,
	$\mathcal{C}$ is called \textbf{semi-additive}
	(i.e. we have 0 and $(f, g) \mapsto f+g$ satisfying $(f+g)h = fh+gh$ and $k(f+g) = kf + kg$).
	
	If $\mathcal{A} = \AbGp$, $\mathcal{C}$ is called \textbf{additive}.
	
	We also talk about \textbf{pointed} and \textbf{(semi)-additive} functors between such categories.
\end{definition}

\begin{lemma}
	\begin{enumerate}[label=\alph*.]
		\item For an object 0 of a pointed category, TFAE:
			\begin{enumerate}[label=\roman*.]
				\item 0 is terminal
				\item 0 is initial
				\item $1_0 = 0$
			\end{enumerate}
		\item For three objects $A, B, C$ of a semi-additive category, TFAE:
			\begin{enumerate}[label=\roman*.]
				\item There exist $A \overset{\pi_1}{\longleftarrow} C \overset{\pi_2}{\longrightarrow} B$ making $C$ a product of $A$ and $B$
				\item There exist $A \overset{\nu_1}{\longrightarrow} C \overset{\nu_2}{\longleftarrow} B$ making $C$ a coproduct of $A$ and $B$
				\item There exist $A \rightleft{\nu_1}{\pi_1} C \rightleft{\nu_2}{\pi_2} B$ satisfying $\pi_1\nu_1=1_A$, $\pi_2\nu_2=1_B$, $\pi_1\nu_1=0=\pi_2\nu_1$ and $\nu_1\pi_1+\nu_2\pi_2 = 1_\mathcal{C}$
			\end{enumerate}
	\end{enumerate}
	\label{72}
\end{lemma}
\begin{proof}
	In each case we prove (i) $\iff$ (iii); (ii) $\iff$ (iii) is dual.
	\begin{enumerate}[label=\alph*.]
		\item (i) $\implies$ (iii) since if $0$ is terminal there's only one morphism $0 \to 0$.
		
		(iii) $\implies$ (i): if (iii) holds then any $f: A \to 0$ satisfies $f = 1_0 f = 0f = 0$.
		
		\item (i) $\implies$ (iii): given $\pi_1$ and $\pi_2$ making $C$ a product,
		we define $\nu_1$ and $\nu_2$ to be the unique morphisms satisfying the first four equations of (iii).
		
		Then 
		\begin{align*}
			\pi(\nu_1\pi_1 + \nu_2\pi_2) &= \pi_1\nu_1\pi_1 + \pi_1\nu_2\pi_2 \\
			&= 1_A \pi_1 + 0\pi_2 = \pi_1
		\end{align*}
		and $\pi_2(\nu_1\pi_1 + \nu_2\pi_2) = \pi_2$ similarly.
		
		So $\nu_1\pi_1 + \nu_2\pi_2 = 1_C$ since it's a factorisation of $(\pi_1, \pi_2)$ through itself.
		
		(iii) $\implies$ (i): given
		\begin{tikzcd}[row sep=tiny] & A \\ D \ar{ur}{f} \ar{dr}{g} \\ & B\end{tikzcd}
		, if there exists $h: D \to C$ such that $\pi_1 h = f$, $\pi_2 h = g$ then
		$$h = (\nu_1 \pi_1 + \nu_2\pi-2)h = \nu_1\pi_1 h + \nu_2 \pi_2 h = \nu_1 f + \nu_2 g$$
		But $\pi_1(\nu_1 f + \nu_2 g) = \pi_1\nu_1 f+ \pi_1\nu_2 g = f + 0 = f$,
		and $\pi_2(\nu_1 f + \nu_2 g) = g$ similarly.
	\end{enumerate}
\end{proof}

We say 0 is a \textbf{zero object} if it is both initial and terminal.

In (b) we call $C$ a \textbf{biproduct} of $A$ and $B$, and denote it $A \oplus B$.

\begin{lemma}
	\begin{enumerate}[label=\alph*.]
		\item A category with a zero object has a unique pointed structure.
		\item If $\mathcal{C}$ is pointed with finite products and coproducts,
		and for every pair $(A, B)$ the canonical $c: A + B \to A \times B$ defined by $\pi_i c \nu_j = \delta_{ij}$ is an isomorphism,
		then $\mathcal{C}$ has a unique semi-additive structure.
	\end{enumerate}
\end{lemma}
\begin{proof}
	\begin{enumerate}[label=\alph*.]
		\item We define $0: A \to B$ to be the unique composite $A \to 0 \to B$,
		where 0 is the zero object (and this is the only possibility).
		
		\item Given $(f,g) : A \rightrightarrows B$, we define $f +_L g$ to be the composite
		$$A \overset{{1 \choose 1}}{\longrightarrow} A \times A \overset{c^{-1}}{\longrightarrow} A + A \overset{(f,g)}{\longrightarrow} B $$
		and $f +_R g$ to be
		$$A \overset{{f \choose g}}{\longrightarrow} B \times B \overset{c^{-1}}{\longrightarrow} B + B \overset{(1,1)}{\longrightarrow} B $$
		Note that we have $h(f+_Lg) = hf +_L hg$ and $(f+_Rg)k = fk+_Rgk$ when the composites are defined.
		
		To show $f+_L0 = f$, consider
		\begin{center}
			\begin{tikzcd}[column sep=tiny]
				A \ar{rr}{{1 \choose 1}} \ar{rrrd}[below]{1} && A \times A \ar{rr}{c^{-1}} \ar{rd}{\pi_1} && A+A  \ar{rr}{(f, 0)} \ar{ld}[above]{(1,0)}&& B \\
				&&&A \ar{urrr}[below]{f}
			\end{tikzcd}
		\end{center}
		All three triangles commute,
		so $f +_L 0 = f$.
		Similarly $0 +_L f = f$ and dually $f+_R 0=0+_Rf = f$.
		
		Now consider the composite
		$$A \overset{\begin{psmallmatrix}1\\1\end{psmallmatrix}}{\longrightarrow} A \times A \overset{c^{-1}}{\longrightarrow} A+A
		\overset{\begin{psmallmatrix}f&g\\h&k\end{psmallmatrix}}{\longrightarrow} B \times B \overset{c^{-1}}{\longrightarrow} B+B \overset{(1,1)}{\longrightarrow} B$$
		
		The composite $A \to B \times B$ is $\begin{psmallmatrix} f+_L g \\ h +_L k\end{psmallmatrix}$,
		so the whole is $(f+_L g)+_R(h+_Lk)$.
		But it also equals $(f+_R h)+_L(g+_R k)$.

		Now putting $g=h=0$, we get $f+_Rk=f+_Lk$ so $+_R = +_L$.
	
		Putting $f=k=0$, we get $g+h = h+g$, so $+$ is commutative.  
		
		Putting $g=0$, we get $f+(h+k) = (f+h)+k$, so $+$ is associative.
		
		For uniqueness: given any semi-additive structure $+$,
		we must have $c^{-1} = \nu_1 \pi_1 + \nu_2 \pi_2$ by \ref{72}(b),
		so $+$ coincides with $+_L$ and $+_R$.
	\end{enumerate}
\end{proof}

\begin{corollary}
	If $\mathcal{C}$ and $\mathcal{D}$ are semi-additive with finite biproducts,
	then a functor $F:\mathcal{C} \to \mathcal{D}$ is additive $\iff$ it preserves finite (co)products.
	
	In particular, if $F$ has an adjoint (on either side) then it's additive.
	Moreover, the adjunction is enriched over $\CMon$,
	in the sense that the bijection $\mathcal{D}(FA, B) \to \mathcal{C}(A, GB)$ is an isomorphism of commutative monoids.
\end{corollary}

\begin{definition}
	Let $A \overset{f}{\to} B$ be a morphism in a pointed category.
	The \textbf{kernel} of $f$ is the equaliser $E \overset{\ker f}{\longrightarrow} A$ of $A \parallelpair{f}{0} B$.
	
	A monomorphism is \textbf{normal} if it occurs as a kernel.
	
	In additive categories, every regular monomorphism is normal,
	since the equaliser of $A \parallelpair{f}{g} B$ is the kernel of $f-g$.
	
	$A \overset{f}{\to} B$ is a \textbf{pseudo-monomorphism} if $\ker f$ is a zero map,
	i.e. if $fg = 0 \implies g = 0$.
	Again, in additive categories this holds if $f$ is monic, but in semi-additive categories it's weaker.
\end{definition}

\begin{lemma}
	In a pointed category with cokernels, every normal monomorphism is the kernel of its own cokernel.
\end{lemma}
\begin{proof}
	Suppose $f= \ker g$. Then $gf=0$ so $g$ factors through $\coker f$.
	\begin{center}
		\begin{tikzcd}[column sep=tiny]
			A \ar[tail]{r}{f} &[20pt] B \ar{rr}{g} \ar{dr}[below]{\coker f} && C \\
			E \ar{ur}{h} && D \ar{ur}
		\end{tikzcd}
	\end{center}
	So if $h:E \to B$ satisfies $(\coker f)h = 0$, then $gh=0$,
	and hence $h$ factors uniquely through $f$.
\end{proof}

In particular, if $\mathcal{C}$ has kernels and cokernels,
then $\ker$ and $\coker$ induce a bijection between (isomorphism classes) 
of normal subobjects and normal quotients of any object.

\begin{lemma}
	If $\mathcal{C}$ is pointed with kernels and cokernels then $\mathcal{C}$ has images.
\end{lemma}
\begin{proof}
	Given $A \overset{f}{\to} B$, $f$ factors through $\ker(B \overset{g}{\to} C)$
	\begin{align*}
		& \iff gf = 0 \\
		& \iff g \text{ factors through } \coker f \\
		& \iff \ker \coker f \text{ factors through } \ker g
	\end{align*}
	So $\ker\coker f$ is the smallest normal subobject of $\mathcal{C}$ through which $f$ factors.
\end{proof}

\begin{definition}
	A category $\mathcal{A}$ is \textbf{ablelian} if
	\begin{enumerate}[label=\roman*.]
		\item $\mathcal{A}$ is additive
		\item $\mathcal{A}$ has finite biproducts, kernels and cokernels (equivalently all finite limits and colimits)
		\item Every monomorphism is normal and very epimorphism is normal
	\end{enumerate} 
\end{definition}

\begin{lemma}
	In an additive category with finite biproducts,
	consider a square
	\begin{tikzcd}
		A \ar{r}{f} \ar{d}{g} & B \ar{d}{h} \\
		C \ar{r}{k} & D \\
	\end{tikzcd}
	of objects and morphisms.
	
	The flatting of the square is the diagram
	$$A \overset{\begin{psmallmatrix}f\\g\end{psmallmatrix}}{\longrightarrow} B \oplus C \overset{(h, -k)}{\longrightarrow} D$$
	
	Then
	\begin{enumerate}[label=\roman*.]
		\item the square commutes $\iff (h, -k)\begin{psmallmatrix}f\\g\end{psmallmatrix} = 0$
		\item the square is a pullback $\iff \begin{psmallmatrix}f\\g\end{psmallmatrix} = \ker(h, -k)$
		\item the square is a pushout $\iff (h, -k) = \coker \begin{psmallmatrix}f\\g\end{psmallmatrix}$
	\end{enumerate}
\end{lemma}
\begin{proof}
	The composite of the flattening is $hf-kg$.
	
	Then $\begin{psmallmatrix}f\\g\end{psmallmatrix}$ is universal among morphisms with $(h, -k)\begin{psmallmatrix}f\\g\end{psmallmatrix} = 0 \iff (f,g)$ is universal among pairs with $hf=kg$.
	
	(iii) is dual to (ii).
\end{proof}

\begin{corollary}
	In an abelian category, epimorphisms are stable under pullback.
\end{corollary}
\begin{proof}
	Suppose
	\begin{tikzcd}
		A \ar{r}{f} \ar{d}{g} & B \ar[two heads]{d}{h} \\
		C \ar{r}{k} & D
	\end{tikzcd}
	is a pullback with $h$ epic.
	
	Then $\begin{psmallmatrix}f\\g\end{psmallmatrix} = \ker(h, -k)$,
	but $(h, -k)$ is epic since $h$ is,
	so $(h, -k) = \coker \begin{psmallmatrix}f\\g\end{psmallmatrix}$ and the square is a pushout.

	Now suppose $C \overset{l}{\to} E$ satisfies $lg=0$.
	Then the pair $(l, B \overset{0}{\to} E)$ factors through $(k, h)$,
	say by $m: D \to E$,
	but then $mh=0$ and so $m=0$ since $h$ is epic.
	So $l=mk=0$.
\end{proof}


Hence abelian categories are regular. In fact,

\begin{theorem}
	$\mathcal{A}$ is abelian $\iff \mathcal{A}$ is additive and effective regular.
\end{theorem}
\begin{proof}
	Suppose that $\mathcal{A}$ is abelian.
	We've shown that $\mathcal{A}$ is regular, so we need to consider effectivity.
	
	Let $R \parallelpair{a}{b} A$ be an equivalence relation.
	Form the pullback
	\begin{tikzcd}
		K \ar{r}{l} \ar[tail]{d}{k} & R \ar[tail]{d}{(a,b)} \\
		A \ar{r}{\begin{psmallmatrix}1\\0\end{psmallmatrix}} & A \oplus A
	\end{tikzcd}
	$k$ is monic, so it's the kernel of some $A \overset{f}{\to} B$.
	We'll show $(a, b)$ is the kernel-pair of $f$.
	
	Suppose $C \parallelpair{x}{y} A$ satisfy $fx = fy$,
	then $x-y$ factors as $kz$ for some $z: C \to K$.
	
	Now consider $lz+ry: C \to R$ where $r: A \to R$ satisfies $ar = br = 1_A$.
	
	Now
	\begin{align*}
		a(lz+ry) &= alz+ary \\
		&= kz +y \\
		&= x-y+y = x
	\end{align*}
	\begin{align*}
	b(lz+ry) &= blz+bry \\
	&= 0z+y = y
	\end{align*}
	So $lz+ry$ is a factorisation of $(x, y)$ through $(a, b)$.
	
	Conversely: first we show that any reflexive relation in an additive category (with finite limits) is symmetric and transitive:
	given \begin{tikzcd}R \ar[shift left]{r}{a} \ar[shift right]{r}[below]{b} & A \ar[bend left=40]{l}{r} \end{tikzcd},
	let $s=ra+rb-1_R: R \to R$. Then
	\begin{align*}
		as &= ara+arb-a \\
		&= a + b - a = b
	\end{align*}
	and $bs = a$ similarly.
	
	Similarly, given the pullback
	\begin{tikzcd}
		T \ar{r}{q} \ar{d}{p} & R \ar{d}{a}\\
		R \ar{r}{b} & A
	\end{tikzcd}
	, set $t=p+q-raq : T \to R$.
	
	Then $at = ap+aq-araq = ap$ and $bt=bp+pq-brbp=bq$.
	
	Now suppose $\mathcal{A}$ is effective regular and additive.
	Then $\mathcal{A}$ has finite biproducts and kernels.
	
	Consider a monomorphism $K \overset{k}{\rightarrowtail} A$.
	Consider $K \oplus A \parallelpair{(k, 1)}{(0, 1)}$;
	this is jointly monic since if $(k, 1) \col{x}{y} = (0,1)\col{x}{y} = 0$ then $y=0$ and $kx=0$,
	so $x=0$.
	
	It's reflexive with common splittling $A \overset{\col{0}{1}}{\longrightarrow} K \oplus A$.
	So it's an equivalence relation,
	and hence a kernel-pair of some $A \overset{f}{\to} B$.
	Now if $fg=0$ for some $C \overset{g}{\to} A$,
	then $(g, 0)$ factors as
	\begin{center}
		\begin{tikzcd}
			C \ar{r}{\col{u}{v}} & K \oplus A \ar[shift left]{r}{(k, 1)} \ar[shift right]{r}[below]{(0, 1)} & A
		\end{tikzcd}
	\end{center}
	whence $r=0$, and $ku=g$.
	So $k=\ker f$.
	
	Since $\mathcal{A}$ has images,
	it's enough to show that monomorphisms have cokernels.
	Given a monomorphism $K \overset{k}{\rightarrowtail}{A}$,
	form $K \oplus A \parallelpair{(k, 1)}{(0, 1)} A$;
	this is an equivalence relation,
	so has a coequaliser by \ref{65}.
	But a coequaliser for this pair is a cokernel fork.
	
	Given an arbitrary epimorphism $A \overset{f}{\twoheadrightarrow} B$,
	we can factor it as $A \overset{q}{\rightarrowtriangle} I \overset{m}{\rightarrowtail} B$ where $m$ is both regular monic and epic,
	hence an isomorphism,
	and $q$ is a normal epimorphism.
	So $f$ is normal.
\end{proof}

\begin{definition}
	\begin{enumerate}[label=\alph*.]
		\item Given a sequence
		\begin{center}
			\begin{tikzcd}\dots \ar{r} & C_{n+1} \ar{r}{f_{n+1}} & C_n \ar{r}{f_n} & C_{n-1} \ar{r} & \dots \end{tikzcd}
		\end{center}
		in a pointed category with kernels and cokernels,
		we say it's \textbf{exact at} $\mathbf{C_n}$ if $\img f_{n+1} = \ker f_n$
		(equivalently, $\coker f_{n+1} = \coim f_n$).
		
		E.g.
		\begin{enumerate}[label=\roman*.]
			\item \begin{tikzcd}0 \ar{r} & A \ar{r}{f} & B\end{tikzcd} exact $\iff f$ monic
			\item \begin{tikzcd}0 \ar{r} & A \ar{r}{f} & B \ar{r}{g} & C\end{tikzcd} exact $\iff f = \ker g$
			\item \begin{tikzcd}0 \ar{r} & A \ar{r}{f} & B \ar{r}{g} & C \ar{r} &0\end{tikzcd} exact $\iff f = \ker g$ and $g=\coker f$
		\end{enumerate}
		
		\item A functor $F: \mathcal{A} \to \mathcal{B}$ between abelian categories is \textbf{exact} if it preserves exact sequences
		(equivalently, preserves kernels and cokernels).
		
		Note that a biproduct $A \oplus B$ is characterised by the exactness of
		\begin{center}
			\begin{tikzcd}
				0 \ar{r} & A \ar{r}{\col{0}{1}} & A \oplus B \ar{r}{(0, 1)} & B \ar{r} & 0
			\end{tikzcd}
		\end{center}
		plus the fact that $\col{1}{0}$ and $(0, 1)$ are both split.
		So any exact functor preserves biproducts,
		and hence all finite limits and colimits.
		
		\item $F$ is called \textbf{left exact} if it preserves kernels
		(equivalently, exact sequences of type (ii)).
		This also implies additivity,
		and hence preservation of all finite limits. 
	\end{enumerate}
\end{definition}

Note that if $F$ is left exact and preserves epimorphisms, then it's exact.
In particular, a regular functor between abelian categories is exact.

\begin{theorem}
	Let $\mathcal{A}$ be a small abelian category.
	Then there exists a faithful (equivalently isomorphism-reflecting) exact functor $\mathcal{A} \to \AbGp$
	(and a full and faithful exact functor $\mathcal{A} \to \Mod_R$ for some ring $R$).
\end{theorem}
\begin{proof}
	As a regular category,
	$\mathcal{A}$ is totally supported,
	so by \ref{610} there's a faithful regular functor $F = \colim_{J^{op}} \mathcal{A}(D(j), -): \mathcal{A} \to \Set$.
	
	If we regard the $\mathcal{A}(D(j), -)$ as functors $\mathcal{A} \to \AbGp$,
	we can take their colimit in $[\mathcal{A}, \AbGp]$,
	and the description of $F$ in \ref{610} still works.
	It's still regular,
	since $\AbGp \to \Set$ reflects finite limits and strong epimorphisms,
	so it's exact.
	It's still faithful.
	
	To get the full and faithful functor to $\Mod_R$,
	we proceed as in \ref{611},
	taking $R$ to be the ring of endomorphisms of $F$ in the additive category $[\mathcal{A}, \AbGp]$.
\end{proof}

We can now proof all of the standard 'diagram chasing' lemmas for abelian categories by proving them in module categories,
and then transferring them via the embedding theorem.

For example, we have the \textbf{Snake Lemma}:

\begin{lemma}
	Suppose given a commutative diagram
	\begin{center}
		\begin{tikzcd}
			& 0 \ar{d} & 0 \ar{d} & 0  \ar{d} \\
			& A_1 \ar{d} \ar[color=red]{r} & A_2 \ar{d} \ar[color=red]{r} & A_3 \ar{d} \ar[color=red, rounded corners, ]{dddll} \\
			& B_1 \ar{d} \ar{r} & B_2 \ar{d} \ar{r} & B_3 \ar{d} \ar{r} & 0 \\
			0 \ar{r} & C_1 \ar{d} \ar{r} & C_2 \ar{d} \ar{r} & C_3 \ar{d} \\
			&D_1 \ar{d} \ar[color=red]{r} & D_2 \ar{d} \ar[color=red]{r} & D_3 \ar{d} \\
			& 0 & 0 & 0 \\
		\end{tikzcd}
	\end{center}
	in an abelian category where the black rows and columns are exact.
	Then there exist red morphisms forming an exact sequence.
\end{lemma}

\begin{definition}
	A \textbf{complex} in an abelian category $\mathcal{A}$ is an infinite sequence
	\begin{center}
		\begin{tikzcd}\dots \ar{r} & C_{n+1} \ar{r}{f_{n+1}} & C_n \ar{r}{f_n} & C_{n-1} \ar{r} & \dots \end{tikzcd}
	\end{center}
	satisfying $f_n f_{n+1} = 0$ for all $n$
	(equivalently, $\ker f_n \geq \img f_{n+1}$).
\end{definition}

The complexes in $\mathcal{A}$ form an abelian category $c\mathcal{A} = \Add(\mathcal{Z}, \mathcal{A})$
where $\mathcal{Z}$ is the additive category with $\ob \mathcal{Z} = \mathbb{Z}$.
\begin{align*}
	\mathcal{Z}(p, q) &= \mathbb{Z} \text{ if } q=p \text{ or } p=1 \\
	&= \{0\} \text{ otherwise}
\end{align*}
and
\begin{align*}
p \overset{m}{\to} q \overset{n}{\to} r &= mn \text{ if } p \geq q \geq r \geq p-1 \\
&= 0 \text{ otherwise}
\end{align*}

Given a complex $C_\bullet$, we write
\begin{align*}
	Z_n(C_\bullet) &\rightarrowtail C_n \text{ for } \ker f_n \\
	B_n(C_\bullet) &\rightarrowtail C_n \text{ for } \img f_{n+1} \\
	Z_n(C_\bullet) &\rightarrowtail H_n(C_\bullet) \text{ for } \coker(B_n \rightarrowtail Z_n) \\
\end{align*}
If we also write $C_n \to Q_n(C_\bullet)$ for $\coker f_{n+1}$,
then we have a diagram
\begin{center}
	\begin{tikzcd}[column sep=small]
		C_{n+1} \ar[two heads]{rd} \ar{rrrr}{f_{n+1}} & & & & C_n \ar{rrrr}{f_n} \ar[two heads]{rd} & & & & C_{n-1} \\
		& Q_{n+1} \ar[two heads]{r} & B_n \ar[tail]{r} & Z_n \ar[two heads]{r} \ar[tail]{ur} & H_n \ar[tail]{r} & Q_n \ar[two heads]{r} & B_{n-1} \ar[tail]{r} & Z_{n-1} \ar[tail]{ur}
	\end{tikzcd}
\end{center}
and $H_n = \img(Z_n \to Q_n)$.
Note also that $B_n$, $Q_n$, $H_n$ and $Z_n$ are all (additive) functors $c\mathcal{A} \to \mathcal{A}$.

\begin{theorem}[Mayer-Vietoris]
	Let
	$$\begin{tikzcd}0 \ar{r} & C_\bullet \ar{r} & D_\bullet \ar{r} & E_\bullet  \ar{r} & 0\end{tikzcd}$$
	be a short exact sequence in $c\mathcal{A}$.
	Then there's an exact sequence
	$$\begin{tikzcd}[column sep=small]\dots \ar{r} & H_n(C_\bullet) \ar{r} & H_n(D_\bullet) \ar{r} & H_n(E_\bullet) \ar{r} & H_{n-1}(C_\bullet) \ar{r} & H_{n-1}(D_\bullet) \ar{r} & \dots\end{tikzcd}$$
	\label{718}
\end{theorem}
\begin{proof}
	First apply the snake lemma to
	\begin{center}
		\begin{tikzcd}
			& 0 \ar{d} & 0 \ar{d} & 0 \ar{d} \\
			0 \ar{r} & Z_n(C_\bullet) \ar{r} \ar{d} &  Z_n(D_\bullet) \ar{r} \ar{d} & Z_n(E_\bullet) \ar{d} \\
			0 \ar{r} & C_n \ar{r} \ar{d} &  D_n \ar{r} \ar{d} & E_n \ar{r} \ar{d} & 0 \\
			0 \ar{r} & C_{n-1} \ar{r} \ar{d} &  D_{n-1} \ar{r} \ar{d} & E_{n-1} \ar{r} \ar{d} & 0 \\
			& Q_{n-1}(C_\bullet) \ar{r} \ar{d} & Q_{n-1}(D_\bullet) \ar{r} \ar{d} & Q_{n-1}(E_\bullet) \ar{r} \ar{d} & 0 \\
			& 0 & 0 & 0
		\end{tikzcd}
	\end{center}
	
	Then apply it to
	\begin{center}
		\begin{tikzcd}
			& 0 \ar{d} & 0 \ar{d} & 0 \ar{d} \\
			& H_n(C_\bullet) \ar{r} \ar{d} &  H_n(D_\bullet) \ar{r} \ar{d} & H_n(E_\bullet) \ar{d} \ar{llddd} \\
			& Q_n(C_\bullet) \ar{r} \ar{d} &  Q_n(D_\bullet) \ar{r} \ar{d} & Q_n(E_\bullet) \ar{r} \ar{d} & 0 \\
			0 \ar{r} & Z_{n-1}(C_\bullet) \ar{r} \ar{d} &  Z_{n-1}(D_\bullet) \ar{r} \ar{d} & Z_{n-1}(E_\bullet) \ar{d}\\
			& H_{n-1}(C_\bullet) \ar{r} \ar{d} & H_{n-1}(D_\bullet) \ar{r} \ar{d} & H_{n-1}(E_\bullet) \ar{d} \\
			& 0 & 0 & 0
		\end{tikzcd}
	\end{center}
\end{proof}

\begin{definition}
	Let $f_\bullet, g_\bullet: C_\bullet \rightrightarrows D_\bullet$ be morphisms of complexes.
	A \textbf{homotopy} between $f_\bullet$ and $g_\bullet$ is a sequence of morphisms $h_n:C_n \to D_n$
	such that $f_n - g_n = d_{n+1}h_n+h_{n-1}C_n$
	\begin{center}
		\begin{tikzcd}
			C_{n+1} \ar{r}{c_{n+1}} & C_n \ar{r}{c_n} \ar[shift right]{d}[left]{f_n} \ar[shift left]{d}{g_n} \ar{ld}[above]{h_n} & C_{n-1} \ar{ld}{h_{n-1}} \\
			D_{n+1} \ar{r}{d_{n+1}} & D_n \ar{r}{d_n} & D_{n-1}
		\end{tikzcd}
	\end{center}
	
	We write $f_\bullet \simeq g_\bullet$ if there exists a homotopy.
	Note that $\simeq$ is an equivalence relation and it's a congruence
	(if $k_\bullet: D_\bullet \to E_\bullet$,
	then the composites $k_n h_n$ form a homotopy $k_\bullet f_\bullet \simeq k_\bullet g_\bullet$
	and similarly on the other side).
\end{definition}

So we can form the quotient category $c\mathcal{A}/\simeq$.

\begin{lemma}
	If $f_\bullet \simeq g_\bullet$ then $H_n(f_n) = H_n(g_n)$.
	\label{720}
\end{lemma}
\begin{proof}
	The difference $Z_n(f_n)-Z_n(g_n)$ is the restriction to $Z_n(C_\bullet)$ of $d_{n+1}h_n$
	since $C_n$ restricted to $Z_n$ is 0.
	
	When we compose with the quotient map $Z_n(D_\bullet) \to H_n(D_\bullet)$,
	this term also becomes 0,
	so $H_n(f_n)-H_n(g_n) = 0$.
\end{proof}

\begin{definition}
	\begin{enumerate}[label=\alph*.]
		\item A category $\mathcal{C}$ has \textbf{enough projectives} if for all $A \in \ob \mathcal{C}$,
		there exists an epimorphism $P \twoheadrightarrow A$ with $P$ projective.
		
		(Note that $\Mod_R$ has enough projectives, since free modules are projective.)
		
		\item A \textbf{projective resolution} of an object $A$ in an abelian category $\mathcal{A}$
		is an exact sequence
		$$\begin{tikzcd}[column sep=small]\dots \ar{r} & P_n \ar{r} & P_{n-1} \ar{r} & \dots \ar{r} & P_1 \ar{r} & P_0 \ar{r} & A \ar{r} & 0\end{tikzcd}$$
		with all $P_i$ projective.
		
		Equivalently it's a complex $P_\bullet$ with $P_n=0$ for all $n<0$,
		$P_n$ projective for all $n \geq 0$ and
		$$H_n(P_\bullet) = \begin{cases}
		A & n=0 \\
		0 & n \neq 0
		\end{cases}$$
	\end{enumerate}
\end{definition}

If $A$ has enough projectives,
every object has a projective resolution:
choose $P_0 \twoheadrightarrow A$ with $P_0$ projective.
Let $K_0 \to P_0 = \ker(P_0 \twoheadrightarrow A)$,
choose $P_1 \twoheadrightarrow K_0$ with $P_1$ projective, etc.

\begin{lemma}
	Let $P_\bullet$ and $Q_\bullet$ be projective resolutions of $A$ and $B$ respectively.
	For any $f: A \to B$,
	there exists $g_\bullet: P_\bullet \to Q_\bullet$ with $H_0(g_\bullet) = f$
	and any two such complex morphisms are homotopic.
	\label{722}
\end{lemma}
\begin{proof}
	Consider
	\begin{center}
		\begin{tikzcd}
			\dots \ar{r} & P_2 \ar{r} & P_1 \ar{r}{p_1} \ar{d}{g_1} & P_0 \ar[two heads]{r}{p_0} \ar{d}{g_0} & A \ar{d}{f} \\
			\dots \ar{r} & Q_2 \ar{r} & Q_1 \ar{r}{q_1} & Q_0 \ar[two heads]{r}{q_0} & B\\
		\end{tikzcd}
	\end{center}
	$g_0$ exists since $P_0$ is projective and $q_0$ epic.
	$q_0g_0p_1 = fp_0p_1=0$ so $g_0p_1$ factors through $\ker q_0 = \img q_1$,
	so $g_1$ exists since $P_1$ is projective and $\coim q_1$ is epic, etc.
	
	Suppose we had another such morphism $g_\bullet'$.
	Then $q_0(g_0-g_0') = fp_0-fp_0 = 0$.
	So $g_0-g_0'$ factors through $\img q_1$,
	so $\exists h_0$ with $q_1h_0 = g_0-g_0'$.
	
	Now
	\begin{align*}
		q_1(g_1-g_1'- h_0p_1) &= g_0p_1-g_0'p_1-q_1h_0p_1 \\
		&= (g_0-g_0'-q_1h_0)p_1 = 0
	\end{align*}
	So $g_1-g_1'-q_1h_0$ factors through $\ker q_1 = \img q_2$,
	and hence there exists $h_1: P_1 \to Q_2$ with $q_2h_1-g_1-g_1'-h_0p_1$, etc.
\end{proof}

Hence, we can regard the construction of projective resolutions as defining a functor $\mathcal{A} \to c\mathcal{A}/\simeq$.

\begin{definition}
	Let $\mathcal{A}, \mathcal{B}$ be abelian categories such that $\mathcal{A}$ has enough projectives
	and let $F: \mathcal{A} \to \mathcal{B}$ be an additive functor.
	The \textbf{left derived functors} $L^nF$ of $F$ are defined as follows:
	given $A$, let $P_\bullet$ be a projective resolution of $A$ and define
	$L^nF(A) = H_n(FP_\bullet)$ for all $n \geq 0$.
	\label{723}
\end{definition}

This is well defined by \ref{720} and \ref{722}:
it's the composite
$$\begin{tikzcd} \mathcal{A} \ar{r}{PR} & c\mathcal{A}/\simeq \ar{r}{cF/\simeq} & c\mathcal{B}/\simeq \ar{r}{H_n} & B\end{tikzcd}$$
(In fact, it's an additive functor.)
If $F$ is exact, then $L^0F \cong F$ and $L^n F =0$ for all $n > 0$.

If $F$ is right exact (i.e. preserves cokernels),
we still have $L^0F \cong F$, since $FP_1 \to FP_0 \to FA \to 0$ is exact,
but the $L^nF$ may be nonzero.

\begin{theorem}
	Let $\mathcal{A}, \mathcal{B}$ and $F$ be as in \ref{723}.
	Then for any short exact sequence $0 \to A \to B \to C \to 0$ in $\mathcal{A}$,
	there is a long exact sequence
	$$\dots \to L^1FC \to L^0FA \to L^0FB \to L^0FC \to 0$$
	in $\mathcal{B}$
	(so if $ F$ is right exact then $L^nF$ repair the lack of left exactness).
\end{theorem}
\begin{proof}
	We show that if $P_\bullet$ and $R_\bullet$ are projective resolutions of $A$ and $C$,
	then there's a projective resolution $Q_\bullet$ of $B$ for which $Q_n=P_n\oplus R_n$ for all $n$
	and the morphisms $P_n \to Q_n \to R_n$ are
	$P_n \overset{\col{1}{0}}{\longrightarrow} P_n \oplus R_n \overset{(0, 1)}{\longrightarrow} R_n$.
	
	\begin{center}
		\begin{tikzcd}
			& 0 \ar{d} & 0 \ar{d} & 0 \ar{d} \\
			P_1 \ar[two heads]{r} \ar{d}{\col{1}{0}} & K_0 \ar{r} \ar{d} & P_0 \ar{r}{p_0} \ar{d}{\col{1}{0}} & A \ar{r} \ar{d}{f} & 0 \\
			P_1 \oplus R_1 \ar{r} \ar{d}{(0, 1)} & L_0 \ar{r} \ar{d} & P_0 \oplus R_0 \ar{r}{(fp_0, t)} \ar{d}{(0, 1)} & B \ar{r} \ar{d}{g} & 0 \\
			R_1 \ar[two heads]{r} & M_0 \ar{r} \ar{d} & R_0 \ar{r}{r_0} \ar{ur}{t} \ar{d} & C \ar{r} \ar{d} & 0 \\
			& 0 & 0 & 0
		\end{tikzcd}
	\end{center}
	
	Let $t$ be such that $gt = r_0$.
	Suppose $x(fp_0, t)=0$, then $xfp_0 = 0 \implies xf = 0$,
	so $x = yg$ for some $y$ and $0=xt=ygt=yr_0$, so $y=0$. So $x=0$.
	
	Now $0 \to K_0 \to L_0 \to M_0 \to 0$ is exact by the Snake Lemma.
	
	Hence, we can construct an epimorphism $P_1 \oplus R_1 \twoheadrightarrow L_0$ from the epimorphisms
	$P_1 \twoheadrightarrow K_0$ and $R_1 \twoheadrightarrow M_0$ as before.
	
	Continue in the same way.
	
	Since $f$ is additive, it preserves the exactness of the columns
	$0 \to P_n \to P_n \oplus R_n \to R_n \to 0$,
	so the result follows from \ref{718} applied to the exact sequence
	$0 \to FP_\bullet \to FQ_\bullet \to FR_\bullet \to 0$ in $\mathcal{B}$.
\end{proof}
\end{document}