\documentclass[a4paper]{article}
\title{Part III Category Theory}
\author{Based on lectures by Prof P.T. Johnstone}
\date{Michaelmas 2016\\University of Cambridge}
\usepackage{mathtools}
\usepackage{amsthm}
\usepackage{amsmath}
\usepackage{amssymb}
\usepackage{textcomp}
\usepackage{enumerate}
\usepackage{graphicx}
\usepackage{tikz-cd}
\usepackage{enumitem}
\newtheorem{definition}{Definition}
\newtheorem{remark}[definition]{Remark}
\newtheorem{lemma}[definition]{Lemma}
\newtheorem{corollary}[definition]{Corollary}
\newtheorem{theorem}[definition]{Theorem}
\numberwithin{definition}{section}
\renewcommand{\baselinestretch}{1.3}
\newcommand*\conj[1]{\overline{#1}}
\newcommand*\dom[1]{\textnormal{dom}\,#1}
\newcommand*\cod[1]{\textnormal{cod}\,#1}
\newcommand*\ob[1]{\textnormal{ob}\,#1}
\newcommand*\mor[1]{\textnormal{mor}\,#1}
\DeclareMathOperator{\Set}{\textbf{Set}}
\begin{document}
\maketitle
\tableofcontents

\section{Definitions and Examples}
\begin{definition}[Category]
	A category $\mathcal{C}$ consists of 
	\begin{enumerate}[label=\alph*.]
		\item a collection $\ob{\mathcal{C}}$ of \textbf{objects} $A$, $B$, $C$, \dots
		\item a collection $\mor{\mathcal{C}}$ of \textbf{morphisms} $f$, $g$, $h$, \dots
		\item two operations \dom, \cod from morphisms to objects. We write $f:A\to B$ or $A\overset{f}{\to}B$ to mean '$f$ is a morphism and $\dom{f}=A$ and $\cod{f}=B$'
		\item an operation assigning to each object $A$ a morphism $1_A:A\to A$
		\item a partial binary operation $(f, g) \mapsto gf$, s.t. $gf$ is defined $\iff \dom{g}=\cod{f}$, and then $gf:\dom{f}\to\cod{g}$
	\end{enumerate}
	satisfying
	\begin{enumerate}[label=\alph*.]
		\setcounter{enumi}{5}
		\item $f 1_A = f$ and $1_B f=f$ $\forall f: A \to B$
		\item $h(fg) = (hg)f$ whenever $gf$ and $hg$ are defined
	\end{enumerate}
\end{definition}

\begin{definition}[Functor]
	Let $\mathcal{C}$ and $\mathcal{D}$ be categories. A \textbf{functor} $\mathcal{C}\to\mathcal{D}$ consists of
	\begin{enumerate}[label=\alph*.]
		\item a mapping $A \to FA$ from $\ob{\mathcal{C}}$ to $\ob{\mathcal{D}}$
		\item a mapping $f \to Ff$ from $\mor{\mathcal{C}}$ to $\mor{\mathcal{D}}$
	\end{enumerate}
	satisfying $\dom{Ff}=F\dom{f}$, $\cod{Ff}=F\cod{f}$ for all $f$, $F(1_A)=1_{FA}$ for all $A$, and $F(gf)=(Fg)(Ff)$ whenever $gf$ is defined.
\end{definition}

\begin{definition}
	By a \textbf{contravariant functor} $\mathcal{C} \to \mathcal{D}$ we mean a functor $\mathcal{C} \to \mathcal{D}^{op}$ (or equivalently $\mathcal{C}^{op} \to \mathcal{D}$). A functor $\mathcal{C} \to \mathcal{D}$ is sometimes said to be \textbf{covariant}.
\end{definition}

\begin{definition}[Natural transformation]
	Let $\mathcal{C}$ and $\mathcal{D}$ be two categories and $F, G: \mathcal{C} \rightrightarrows \mathcal{D}$ two functors.
	A \textbf{natural transformation} $\alpha:F\to G$ assigns to each $A \in \ob{\mathcal{C}}$ a morphism $\alpha_A:FA\to GA$ in $\mathcal{D}$, such that
	\begin{center}
		\begin{tikzcd}
			FA \arrow[r, "Ff"] \arrow[d, "\alpha_A"] & FB \arrow[d, "\alpha_B"]\\
			GA \arrow[r, "Gf"]& GB
		\end{tikzcd}
	\end{center}
	commutes.
\end{definition}

We can compose natural transformations: given $\alpha: F \to G$ and $\beta: G \to H$, the mapping $A \mapsto \beta_A \alpha_A$ is the $A$-component of a natural transformation $\beta\alpha: F \to H$.

\begin{definition}
	Given categories $\mathcal{C}, \mathcal{D}$, we write $[\mathcal{C},\mathcal{D}]$ for the category of all functors $\mathcal{C} \to \mathcal{D}$ and natural transformations between them.
\end{definition}

\begin{lemma}
	Given $F,G: \mathcal{C} \to \mathcal{D}$ and $\alpha: F \to G$, $\alpha$ is an isomorphism in $[\mathcal{C}, \mathcal{D}] \iff$ each $\alpha_A$ is an isomorphism in $\mathcal{D}$.
\end{lemma}

\begin{definition}[Faithful and full]
	Let $F: \mathcal{C} \to \mathcal{D}$ be a functor.
	\begin{enumerate}[label=\alph*.]
		\item We say that $F$ is \textbf{faithful} if, given $f, g \in \mor{\mathcal{C}}$, the equations $\dom{f}=\dom{g}$, $\cod{f}=\cod{g}$ and $Ff = Fg$ imply $f=g$.
		\item $F$ is \textbf{full} if, given any $g: FA \to FB$ in $\mathcal{D}$, there exists $f: A \to B$ in $\mathcal{C}$ with $Ff =g$.
		\item We say a subcategory $\mathcal{C}'$ of $\mathcal{C}$ is \textbf{full} if the inclusion $\mathcal{C}' \hookrightarrow \mathcal{C}$ is a full functor.
	\end{enumerate}
\end{definition}

For example, \textbf{Gp} is a full subcategory of the category \textbf{Mon} of monoids, but \textbf{Mon} is a non-full subcategory of the category \textbf{Sgp} of semigroups.

\begin{definition}[Equivalence of categories]
	Let $\mathcal{C}$ and $\mathcal{D}$ be categories.
	An \textbf{equivalence} between $\mathcal{C}$ and $\mathcal{D}$ is a pair of functors $F: \mathcal{C} \to \mathcal{D}$, $G: \mathcal{D} \to \mathcal{C}$ together with natural isomorphisms $\alpha: 1_\mathcal{C} \to GF$, $\beta: FG \to 1_\mathcal{D}$.
	We write $\mathcal{C} \simeq \mathcal{D}$ to mean that $\mathcal{C}$ and $\mathcal{D}$ are equivalent.
	
	We say a property $P$ of categories is \textbf{categorical} if whenever $\mathcal{C}$ has $P$ and $\mathcal{C}\simeq\mathcal{D}$ then $\mathcal{D}$ has $P$.
\end{definition}

For example, being a groupoid is a categorical property, but being a group is not.

\begin{definition}[Slice category]
	Given an object $B$ of a category $\mathcal{C}$,
	define the \textbf{slice category} $\mathcal{C}/B$ to have morphisms $A \overset{f}{\to} B$ as objects,
	and morphisms $(A \overset{f}{\to} B) \to (A' \overset{f'}{\to} B)$ are morphisms $h: A \to A'$ making
	\begin{center}
		\begin{tikzcd}[column sep=tiny]
				A \arrow[rr, "h"] \arrow[rd, "f"] & & A'\arrow[ld, "f'"]\\
				& B &
		\end{tikzcd}
	\end{center}
	commute.
\end{definition}

\begin{lemma}
	Let $F: \mathcal{C} \to \mathcal{D}$ be a functor.
	Then $F$ is part of an equivalence $\mathcal{C} \simeq \mathcal{D} \iff$ F is full, faithful and \textbf{essentially surjective},
	i.e. for every $B \in \ob \mathcal{D}$, there exists $A \in \ob \mathcal{C}$ s.t. $FA \cong B$.
\end{lemma}

\begin{definition}
	\begin{enumerate}[label=\alph*.]
		\item A \textbf{skeleton} of a category $\mathcal{C}$ is a full subcategory $\mathcal{C}'$ containing exactly one object from each isomorphism class of objects of $\mathcal{C}$.
		\item We say $\mathcal{C}$ is \textbf{skeletal} if it's a skeleton of itself. Equivalently, any isomorphism $f$ in $\mathcal{C}$ satisfies $\dom f = \cod f$.
	\end{enumerate}
\end{definition}

For example, $\text{\textbf{Mat}}_K$ is skeletal. The full subgategory of standard vector spaces $K^n$ is a skeleton of $\text{\textbf{fd Mod}}_K$.

\begin{remark}
	The following statements are each equivalent to the Axiom of Choice:
	\begin{enumerate}
		\item Every small category has a skeleton
		\item Any small category is equivalent to each of its skeletons
		\item Any two skeletons of a given small category are isomorphic
	\end{enumerate}
\end{remark}

\begin{definition}
	Let $f: A \to B$ be a morphism in a category $\mathcal{C}$.
	\begin{enumerate}[label=\alph*.]
		\item $f$ is a \textbf{monomorphism} if, given $g,h:D \rightrightarrows A$, the equation $fg = fh$ implies $g=h$. We write $A \rightarrowtail B$ if $f$ is monic.
		\item Dually, $f$ is an \textbf{epimorphism} if, given $k,l: B \rightrightarrows C$, $kf = lf$ implies $k=l$. We write $A \twoheadrightarrow B$ if $f$ is epic.
		\item $\mathcal{C}$ is a \textbf{balanced} category if every $f \in \mor \mathcal{C}$ which is both monic and epic is an isomorphism.
	\end{enumerate}
\end{definition}

\section{The Yoneda Lemma}
\begin{definition}
	A category $\mathcal{C}$ is \textbf{locally small} if, for any two objects $A, B$ of $\mathcal{C}$, the morphism $A \to B$ are parametrised by a set $\mathcal{C}(A, B)$.
\end{definition}

Given local smallness, $B \mapsto \mathcal{C}(A,B)$ becomes a functor $\mathcal{C}(A, -): \mathcal{C} \to \Set$: if $g: B \to B'$,
the mapping $f \mapsto gf : \mathcal{C}(A, B) \to \mathcal{C}(A, B')$ is functorial since $h(gf) = (hg)f$ for any $h: B' \to B''$.

Similarly, $A \mapsto \mathcal{C}(A, B)$ becomes a functor $\mathcal{C}^{op} \to \Set$.

\begin{lemma}[Yoneda]
	Let $\mathcal{C}$ be a locally small category, $A \in \ob \mathcal{C}$ and $F: \mathcal{C} \to \Set$. Then
	\begin{enumerate}[label=\roman*.]
		\item There is a bijection between natural transformations $\mathcal{C}(A, -) \to F$ and elements of $FA$.
		\item Moreover, this bijection is natural in both $A$ and $F$.
	\end{enumerate}
\end{lemma}
\begin{proof}
	Bijection: given $\alpha : \mathcal{C}(A, -) \to F$, define $\Phi(\alpha) = \alpha_A(1_A) \in FA$.
	
	Given $x \in FA$, define $\Psi(x): \mathcal{C}(A, -) \to F$ by $$\Psi(x)_B(A \overset{f}{\to} B) = (Ff)(x) \in FB$$ $\Psi(x)$ is natural: given $g: B \to C$, we have
	\begin{align*}
	\Psi(x)_C(\mathcal{C}(A, g)(f)) &= \Psi(x)_C(gf) \\
	&= F(gf)(x) \\
	&= (Fg)(Ff)(x) \\
	&= (Fg)\Psi(x)_B(f)
	\end{align*}
	
	$\Phi\Psi(x) = x$ since $F(1_A)(x) = x$, and $\Psi\Phi(\alpha) = \alpha$ since, for any $f: A \to B$,
	\begin{align*}
	\Psi\Phi(\alpha)_B(f) &= Ff(\Phi(\alpha)) \\
	&= Ff(\alpha_A(1_A)) \\
	&= \alpha_B(\mathcal{C}(A,f)(1_A)) \\
	&= \alpha_B(f)
	\end{align*}
\end{proof}

\begin{corollary}
	The mapping $A \to \mathcal{C}(A, -)$ is a full and faithful functor $\mathcal{C}^{op} \to [\mathcal{C}, \Set]$.
\end{corollary}
\begin{proof}
	Given two objects $A,B$, 2.2(i) gives us a bijection from $\mathcal{C}(B,A)$ to the collection of natural transformations $\mathcal{C}(A, -) \to \mathcal{C}(B, -)$ (by taking $F: C \mapsto \mathcal{C}(B, C)$).
	We need to show this is functorial, but given $f \in \mathcal{C}(B, A)$, $\Psi(F)_A$ sends $1_A$ to $\mathcal{C}(B, f)(1_A) = f$, so it's the natural transformation $g \mapsto gf$.
	
	Hence, given $e: C \to B$, $\Psi(fe)(g) = g(fe) = (gf)(e) = \Psi(e)\Psi(f)g$
\end{proof}

We call this functor the \textbf{Yoneda embedding}. Hence any locally small category $\mathcal{C}$ is equivalent to a full subcategory of $[\mathcal{C}^{op}, \Set]$.

\begin{definition}
	A functor $\mathcal{C} \to \Set$ is \textbf{representable} if it's isomorphic to $\mathcal{C}(A, -)$ for some $A$.
	
	A \textbf{representation} of $F: \mathcal{C} \to \Set$ is a pair $(A, x)$ where $A \in \ob \mathcal{C}$, $x \in FA$ and $\Psi(x): \mathcal{C}(A, -) \to F$ is an isomorphism.
	We also call $x$ a \textbf{universal element} of $F$.
\end{definition}

\begin{corollary}['Representations are unique up to unique isomorphism']
	If $(A, x)$ and $(B, y)$ are both representations of $F: \mathcal{C} \to \Set$, then there's a unique isomorphism $f: A \to B$ s.t $Ff(x)=y$.
\end{corollary}

\begin{definition}[Product and coproduct]
	Given two objects $A, B$ of a locally small category $\mathcal{C}$, we define their \textbf{product} to be a representation of the functor
	$$\mathcal{C}(-,A) \times \mathcal{C}(-, B): \mathcal{C}^{op} \to \Set$$
	i.e. an object $A \times B$ equipped with morphisms $\pi_1 : A \times B \to A$, $\pi_2: A \times B \to B$
	s.t. given any pair $(f:C \to A, g: C \to B)$, there exists a unique $h: C \to A \times B$
	s.t. $\pi_1h=f$ and $\pi_2h=g$.
	
	More generally, we can define the product $\prod_{i \in I} A_i$ of a family $\{A_i \,|\, i \in I\}$ of objects,
	or the product of the empty family, i.e. a \textbf{terminal object} 1 s.t. for every $A$ there's a unique $A \to 1$.
	
	Dualizing, we get the notion of \textbf{coproduct} or \textbf{sum}.
\end{definition}

\begin{definition}[Equaliser and coequaliser]
	Given a parallel pair $f, g: A \rightrightarrows B$ in a locally small category $\mathcal{C}$,
	the assignment $C \mapsto FC = \{h: C \to A \,|\, fh=gh\}$ is a subfunctor $F$ of $\mathcal{C}(-, A)$.
	A representation of $F$ is called an \textbf{equaliser} of $(f, g)$.
	
	In elementary terms, it's an object $E$ equipped with $e: E \to A$ s.t. $fe=ge$,
	s.t. any $h$ with $fh = gh$ factors uniquely as $h=ek$
	
	Dually, we have the notion of \textbf{coequaliser},
	i.e. a morphism $q:B \to Q$ satisfying $qf=qg$, and universal among such.
\end{definition}

\begin{definition}
	\begin{enumerate}[label=\alph*.]
		\item We say a monomorphism is \textbf{regular} if it occurs as an equaliser (dually, regular epimorphism).
		\item We say $f: A \to B$ is a \textbf{split monomorphism} if there exists $g: B \to A$ with $gf = 1_A$.
	\end{enumerate}
\end{definition}

Every split monomorphism is regular:
if $gf = 1_A$, $f$ is an equaliser of $(1_B, fg)$ [see sheet 1, q2].

\begin{definition}
	Let $\mathcal{C}$ be a (locally small) category, $\mathcal{G}$ a collection of objects of $\mathcal{C}$.
	\begin{enumerate}[label=\alph*.]
		\item Say $\mathcal{G}$ is a \textbf{separating family} if the functors $\mathcal{C}(G, -),\, G \in \mathcal{G}$ are jointly faithful,
		i.e. if given $f,g : A \rightrightarrows B$ with $f \neq g$,
		there exists $G \in \mathcal{G}$ and $h: G \to A$ with $fh \neq gh$.
		\item Say $\mathcal{G}$ is a \textbf{detecting family} if the $\mathcal{C}(G, -),\, G \in \mathcal{G}$ jointly reflect isomorphisms,
		i.e. if given $f: A \to B$ s.t. every $g: G \to B$ with $G \in \mathcal{G}$ factors uniquely through $f$,
		$f$ is an isomorphism.
	\end{enumerate}
\end{definition}

\begin{lemma}
	\begin{enumerate}[label=\roman*.]
		\item If $\mathcal{C}$ is balanced, then any separating family is detecting
		\item If $\mathcal{C}$ has equalisers, then every detecting family is separating
	\end{enumerate}
\end{lemma}

\begin{definition}
	An object $P$ is \textbf{projective} if $\mathcal{C}(P, -)$ preserves epimorphisms, i.e. if given
	\begin{center}
		\begin{tikzcd}
			& P \arrow[d, "f"] \\
			A \arrow[r, twoheadrightarrow, "e"] & B
		\end{tikzcd}
	\end{center}
	there exists $g: P \to A$ with $eg = f$.
	
	Dually, $P$ is \textbf{injective} in $\mathcal{C}$ if it's projective in $\mathcal{C}^{op}$.
	
	If $P$ satisfies this property $\forall e$ in some class $\mathcal{E}$ of epimorphisms,
	we call it $\mathcal{E}$-projective.
\end{definition}

\begin{corollary}
	Representable functors are (pointwise) projective in $[\mathcal{C}, \Set]$
\end{corollary}
\begin{proof}
	Given
	\begin{center}
		\begin{tikzcd}
			& \mathcal{C}(A, -) \arrow[d, "\beta"] \\
			F \arrow[r, twoheadrightarrow, "\alpha"] & G
		\end{tikzcd}
	\end{center}
	$\beta$ corresponds to some $y \in GA$. 
	$\alpha_A$ is surjective, so $\exists x \in FA$ with $\alpha_A(x)=y$.
	$x$ corresponds to $\gamma: \mathcal{C}(A, -) \to F$ with $\alpha\gamma = \beta$.
\end{proof}

\section{Adjunctions}
\begin{definition}[D.M. Khan, 1958]
	Let $\mathcal{C}$ and $\mathcal{D}$ be categories and $F: \mathcal{C} \to \mathcal{D}$, $G: \mathcal{D} \to \mathcal{C}$ be two functors.
	An \textbf{adjunction} between $F$ and $G$ is a bijection between morphisms $FA \to B$ in $\mathcal{D}$ and morphisms $A \to GB$ in $\mathcal{C}$, 
	which is natural in $A$ and $B$.

	(If $\mathcal{C}$ and $\mathcal{D}$ are locally small,
	this says that $(A, B) \to \mathcal{D}(FA, B)$ and $(A, B) \to \mathcal{C}(A, GB)$
	are naturally isomorphic functors $\mathcal{C}^{op} \times \mathcal{D} \to \Set$).
	
	We say $F$ is \textbf{left adjoint} to $G$,
	or $G$ is \textbf{right adjoint} to $F$,
	and write $F \dashv G$.
\end{definition}

\begin{theorem}
	Let $G: \mathcal{D} \to \mathcal{C}$ be a functor.
	Given $A \in \ob \mathcal{C}$,
	let $(A \downarrow G)$ be the category whose objects are pairs $(B, f)$
	with $B \in \ob \mathcal{D}$,
	$f: A \to GB$ and whose morphisms $(B, f) \to (B', f')$
	are morphisms $g: B \to B'$ in $\mathcal{D}$ such that
	\begin{center}
		\begin{tikzcd}
			A \arrow[r, "f"] \arrow[rd, "f'"] & GB \arrow[d, "Gg"] \\
			& GB'
		\end{tikzcd}
	\end{center}
	commutes.
	Then specifying a left adjoint for $G$ is equivalent to specifying an initial object of $(A \downarrow G)$ for each $A$.
\end{theorem}

\begin{proof}
	First suppose $G$ has a left adjoint $F$.
	Let $\eta_A: A \to GFA$ be the morphism corresponding to $1_{FA}:FA \to FA$.
	The pair $(FA, \eta_A)$ is an object of $(A \downarrow G)$.
	We'll show it's initial.
	
	Given $g: FA \to B$,
	the composite
	\begin{tikzcd}A \arrow[r, "\eta_A"]& GFA \arrow[r, "Gg"]& GB\end{tikzcd}
	must correspond to
	\begin{tikzcd}FA \arrow[r, "1"] & FA \arrow[r, "g"] & B\end{tikzcd}
	under the adjunction.
	
	So, for any object $(B, f)$ of $(A \downarrow G)$,
	the unique morphism $(FA, \eta_A) \to (B, f)$ in $(A \downarrow G)$ is the morphism $FA \to B$ corresponding to $f$.
	
	Conversely, suppose we're given an initial object $(FA, \eta_A)$ of $(A \downarrow G)$ for each $G$.
	Given $f: A \to A'$,
	the composite
	\begin{tikzcd}A \arrow[r, "f"] & A' \arrow[r, "\eta_{A'}"] & GFA'\end{tikzcd}
	is an object of $(A \downarrow G)$,
	so there's a unique morphism $Ff: FA \to FA'$ making
	\begin{center}
		\begin{tikzcd}
			A \arrow[r, "\eta_A"] \arrow[d, "f"] & GFA \arrow[d, "GFf"] \\
			A' \arrow[r, "\eta_{a'}"] & GFA'
		\end{tikzcd}
	\end{center}
	commute.
	
	$f \mapsto Ff$ is functorial:
	given $f': A' \to A''$,
	then $(Ff')(Ff)$ and $F(f'f)$ are both morphisms $(FA, \eta_A) \to (FA'', \eta_{A''}f'f)$ in $(A \downarrow G)$,
	so they're equal.
	
	Finally, given $f: A \to GB$,
	the morphism $g: FA \to B$ corresponding to it is the unique morphism $(FA, \eta_A) \to (B, f)$ in $(A \downarrow G)$.
	
	The naturality of this bijection is given by naturality of $\eta$,
	and naturality in $B$ is immediate.
\end{proof}

\begin{corollary}
	If $F,\, F'$ are both left-adjoint to $G$,
	then there's a canonical natural isomorphism $F \to F'$.
\end{corollary}
\begin{proof}
	For each $A$,
	$(FA, \eta_A)$ and $(F'A, \eta'_A)$ are both initial in $(A \downarrow G)$,
	so there's a unique isomorphism $\alpha_A: (FA, \eta_A) \to (F'A, \eta'_A)$.
	
	$\alpha$ is natural: given $f: A \to A'$,
	$\alpha_{A'}f$ and $(Ff)\alpha_A$ are both morphisms $(FA, \eta_A) \to (F'A', \eta'_{A'}f)$ in $(A \downarrow G)$.
	So they're equal.
\end{proof}

\begin{lemma}
	Given
	\begin{tikzcd}
		\mathcal{C} \arrow[r, "F", shift left] & \mathcal{D} \arrow[r, "H", shift left] \arrow[l, "G", shift left] & \mathcal{E} \arrow[l, "K", shift left]
	\end{tikzcd},
	if $F \dashv G$ and $H \dashv K$ then $HF \dashv GK$.
\end{lemma}
\begin{proof}
	We have bijections
	$$\mathcal{E}(HFA, C) \cong \mathcal{D}(FA, KC) \cong \mathcal{C}(A, GKC)$$
	which are natural in $A$ and $C$.
\end{proof}

\begin{corollary}
	Given a commutative square
	\begin{tikzcd}
		\mathcal{C} \arrow[r, "F"]  \arrow[d, "G"] & \mathcal{D} \arrow[d, "H"] \\
		\mathcal{E} \arrow[r, "K"] & \mathcal{F}
	\end{tikzcd}
	of categories and functors,
	suppose all the functors in the diagram have left adjoints.
	Then the diagram
	\begin{tikzcd}
		\mathcal{F} \arrow[r] \arrow[d] & \mathcal{E} \arrow[d] \\
		\mathcal{D} \arrow[r] & \mathcal{C}
	\end{tikzcd}
	of left adjoints commutes up to natural isomorphism.
\end{corollary}

Given $F \dashv G$,
we have a natural transformation $\eta: 1_\mathcal{C} \to GF$
defined as in 3.2. We call $\eta$ the \textbf{unit} of the adjunction.

Dually, we have $\epsilon: FG \to 1_\mathcal{D}$, the \textbf{counit}.
$\epsilon_B: FGB \to B$ corresponds to $1_{GB}: GB \to GB$.

\begin{theorem}
	Suppose we're given $F: \mathcal{C} \to \mathcal{D}$ and $G: \mathcal{D} \to \mathcal{C}$.
	Specifying an adjunction $F \dashv G$ is equivalent to specifying natural transformations
	$\eta: 1_\mathcal{C} \to GF$ and $\epsilon: FG \to 1_\mathcal{D}$ such that
	\begin{center}
	\begin{tikzcd}
		F \arrow[r, "F\eta"] \arrow[rd, "1_F"] & FGF \arrow[d, "\epsilon_F"] & \text{and} & G \arrow[r, "\eta_G"] \arrow[rd, "1_G"]  & GFG \arrow[d, "G\epsilon"] \\
		                                       &  F                          &            &                                          &  G
	\end{tikzcd}
	\end{center}
	commute.
	(We say $\eta$ and $\epsilon$ satisfy the \textbf{triangular identities}).
\end{theorem}
\begin{proof}
	Given $F \dashv G$, we define $\eta$ and $\epsilon$ as already described.
	Since $\epsilon_{FA}: FGFA \to FA$ corresponds to $1_{GFA}$,
	the composite $\epsilon_{FA}(F\eta_A)$ corresponds to
	\begin{tikzcd}A \arrow[r, "\eta_A"] & GFA \arrow[r, "1_{GFA}"] & GFA \end{tikzcd},
	so it must be $1_{FA}$.
	
	Similarly for the other identity.
	
	Conversely, given $\eta$ and $\epsilon$ satisfying the $\triangle^r$ identities,
	we map $f: A \to GB$ to the composite
	\begin{tikzcd}FA \arrow[r, "Ff"] & FGB \arrow[r, "\epsilon_B"] & B\end{tikzcd}
	and $g: FA \to B$ to the composite
	\begin{tikzcd}A \arrow[r, "\eta_A"] & GFA \arrow[r, "Gg"] & GB\end{tikzcd}.
	
	We have
	\begin{align*}
	\Phi(\begin{tikzcd}[ampersand replacement=\&] A \arrow[r, "f"]\& GB\end{tikzcd}) &=
		\begin{tikzcd}[ampersand replacement=\&]FA \arrow[r, "Ff"] \& FGB \arrow[r, "\epsilon_B"] \& B\end{tikzcd} \\
	\Psi(\begin{tikzcd}[ampersand replacement=\&] FA \arrow[r, "g"]\& B\end{tikzcd}) &=
		\begin{tikzcd}[ampersand replacement=\&]A \arrow[r, "\eta_A"] \& GFA \arrow[r, "Gg"] \& GB\end{tikzcd}
	\end{align*}
	
	So
	\begin{align*}
		\Psi\Phi(f) & =
			\begin{tikzcd}[ampersand replacement=\&] A \arrow[r, "\eta_A"] \& GFA \arrow[r, "GFf"] \& GFGB \arrow[r, "G\epsilon_B"] \& GB\end{tikzcd} \\
			&= \begin{tikzcd}[ampersand replacement=\&] A \arrow[r, "f"] \& GB \arrow[r, "\eta_{GB}"] \& GFGB \arrow[r, "G\epsilon_B"] \& GB\end{tikzcd} \\
			&= f
	\end{align*}
	
	And dually $\Phi\Psi(g) = g$.
	
	Naturality of $\Phi$ in $A$ is immediate from its definition,
	and naturality in $B$ follows from that of $\epsilon$.
\end{proof}

\begin{lemma}
	Suppose given
	\begin{tikzcd}
		\mathcal{C} \arrow[r, "F", shift left] & \mathcal{D} \arrow[l, "G", shift left]
	\end{tikzcd}
	and natural isomorphisms $\alpha: 1_\mathcal{C} \to GF$,
	$\beta: FG \to 1_\mathcal{D}$.
	Then there exist natural isomorphisms $\alpha',\,\beta'$ 
	which additionally satisfy the triangular identities.
	In particular ($F \dashv G$).
\end{lemma}
\begin{proof}
	We define $\alpha' = \alpha$ and take $\beta'$ to be the composite
	\begin{center}
		\begin{tikzcd}
			FG \arrow[r, "\beta_{FG}^{-1}"] & FGFG \arrow[r, "F_{\alpha_G}^{-1}"] & FG \arrow[r, "\beta"] & 1_\mathcal{D}
		\end{tikzcd}
	\end{center}
	Note that, since
	\begin{tikzcd}
		FGFG \arrow[r, "FG\beta"] \arrow[d, "\beta_{FG}"] & FG \arrow[d, "\beta"] \\
		FG \arrow[r, "\beta"] & 1_\mathcal{D}
	\end{tikzcd}
	commutes and $\beta$ is monic,
	we have $FG\beta = \beta_FG$.
	
	Similarly, $GF\alpha = \alpha_{GF}: GF \to GFGF$.
	
	Now
	\begin{align*}
		\beta_F' \circ F_{\alpha'} &=
			\begin{tikzcd}[ampersand replacement=\&]F \arrow[r, "F\alpha"]\& FGF \arrow[r, "\beta_{FGF}^{-1}"]\& FGFGF \arrow[r, "F\alpha_{GF}^{-1}"]\& FGF \arrow[r, "\beta_F"] \& F\end{tikzcd} \\
		&= \begin{tikzcd}[ampersand replacement=\&]F \arrow[r, "\beta_F^{-1}"]\& FGF \arrow[r, "FGF\alpha"]\& FGFGF \arrow[r, "FGF\alpha^{-1}"]\& FGF \arrow[r, "\beta_F"] \& F\end{tikzcd} \\
		&= 1_F
	\end{align*}
	and
	\begin{align*}
	G\beta' \circ \alpha'_G &=
	\begin{tikzcd}[ampersand replacement=\&]G \arrow[r, "\alpha_G"]\& GFG \arrow[r, "GFG\beta^{-1}"]\& GFGFG \arrow[r, "GF^{-1}_{\alpha_G}"]\& GFG \arrow[r, "G\beta"] \& G\end{tikzcd} \\
	&= \begin{tikzcd}[ampersand replacement=\&]G \arrow[r, "G\beta^{-1}"]\& GFG \arrow[r, "\alpha_{GFG}"]\& GFGFG \arrow[r, "\alpha_{GFG}^{-1}"]\& GFG \arrow[r, "\beta_F"] \& G\end{tikzcd} \\
	&= 1_G
	\end{align*}
\end{proof}

\begin{lemma}
	Suppose
	\begin{tikzcd}
		\mathcal{C} \arrow[r, "F", shift left] & \mathcal{D} \arrow[l, "G", shift left]
	\end{tikzcd},
	($F \dashv G$) is an adjunction with counit $\epsilon$.
	Then
	\begin{enumerate}[label=\roman*.]
		\item $\epsilon$ is (pointwise) epic $\iff$ $G$ is faithful
		\item $\epsilon$ is an isomorphism $\iff$ $G$ is full and faithful
	\end{enumerate} 
\end{lemma}
\begin{proof}
	\begin{enumerate}[label=\roman*.]
		\item Given $g: B \to B'$,
		the morphism $Gg: GB \to GB'$ corresponds to
		\begin{center}
			\begin{tikzcd} FGB \arrow[r, "\epsilon_B"] & B \arrow[r, "g"] & B' \end{tikzcd}
		\end{center}
		So, for fixed $B$,
		composition with $\epsilon_B$ is injective on morphisms $B \to B'$
		$\iff$ $(g \mapsto Gg)$ is injective on morphisms $B \to B'$.
		
		Hence $G$ is faithful $\iff \epsilon_B$ is epic $\forall B$.
		
		\item Similarly, $\epsilon_B$ is 0 $\forall B$
		$\implies G$ is bijective on morphisms with given domain and codomain,
		i.e. $G$ is full and faithful.
		
		Conversely, if $G$ is full and faithful,
		$1_{FGB}$ factors uniquely as \newline
		\begin{tikzcd}FGB \arrow[r, "\epsilon_B"] & B \arrow[r, "g"] & FGB\end{tikzcd},
		so $\epsilon_B$ is split monic.
		But it's epic by (i), hence an isomorphism.
	\end{enumerate}
\end{proof}

\begin{definition}
	\begin{enumerate}[label=\roman*.]
		\item A \textbf{reflection} is an adjunction satisfying the conditions of 3.8(ii).
		\item A \textbf{reflective} subcategory of $\mathcal{C}$ is a full subcategory $\mathcal{C}'$
		for which the inclusion $\mathcal{C}' \hookrightarrow \mathcal{C}$ has a left adjoint.
	\end{enumerate}
	
	Dually, \textbf{coreflection} and \textbf{coreflective} subcategory.
\end{definition}

\section{Limits}
\begin{definition}
	\begin{enumerate}[label=\alph*.]
		\item Let $J$ be a category (almost always small, often finite).
		A \textbf{diagram of shape J} in a category $\mathcal{C}$ is a functor $D: J \to \mathcal{C}$.
		
		E.g. if $J$ is the finite category
		\begin{tikzcd}
			\cdot \arrow[r] \arrow[d] \arrow[rd] & \cdot \arrow[d]\\
			\cdot \arrow[r] & \cdot
		\end{tikzcd}
		, a diagram of shape $J$ is a commutative square.
		If $J$ is the category
		\begin{tikzcd}
			\cdot \arrow[r] \arrow[d] \arrow[rd, shift left] \arrow[rd, shift right] & \cdot \arrow[d]\\
			\cdot \arrow[r] & \cdot
		\end{tikzcd}
		, a diagram of shape $J$ is a not-necessarily-commutative square.
		
		The objects $D(j),\,j \in \ob J$ are called \textbf{vertices} of $D$,
		and the morphisms $D(\alpha),\, \alpha \in \mor J$ are called \textbf{edges} of $D$.
		
		\item Let $D: J \to \mathcal{C}$ be a diagram in $\mathcal{C}$.
		A \textbf{cone over D} is a pair $(A,\, (\lambda_j \,|\, j \in \ob J))$ where $\lambda_j: A \to D(j)\ \forall j$,
		and
		\begin{tikzcd}[column sep=tiny]
			& A \arrow[ld, "\lambda_j"] \arrow[rd, "\lambda_{j'}"] & \\
			D(j) \arrow[rr, "D(\alpha)"]&& D(j')
		\end{tikzcd}
		commutes for each $\alpha: j \to j'$ in $J$.
		
		$A$ is called the \textbf{apex} of the cone,
		and the $\lambda_j$ are its \textbf{legs}.
		
		Equivalently, $\lambda$ is a natural transformation $\triangle A \to D$,
		where $\triangle A$ is the \textbf{constant diagram} with all vertices A and all edges $1_A$.
		
		A \textbf{morphism}
		$f: (A, (\lambda_j)) \to (B, (\mu_j))$ of cones over $D$ is a morphism $f: A \to B$ s.t.
		\begin{tikzcd}[column sep=tiny]
		A \arrow[rr, "f"] \arrow[rd, "\lambda_j"] && B \arrow[ld, "\mu_j"]\\
		& D(j) &
		\end{tikzcd}
		commutes for each $j$.
		We have a category \textbf{Cone}($D$) of cones over $D$.
		
		Note that $A \mapsto \triangle A$ is a functor $\mathcal{C} \to [J, \mathcal{C}]$ and
		\textbf{Cone}(D) is in fact the category $(\triangle \downarrow D)$.
	\end{enumerate}
\end{definition}
\end{document}