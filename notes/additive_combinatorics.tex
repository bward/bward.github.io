\documentclass[a4paper]{article}
\title{Part III Topics in Additive Combinatorics}
\author{Based on lectures by Prof W.T. Gowers}
\date{Michaelmas 2016\\University of Cambridge}
\usepackage{mathtools}
\usepackage{amsthm}
\usepackage{amssymb}
\usepackage{textcomp}
\usepackage{enumerate}
\usepackage{graphicx}
\usepackage{tikz-cd}
\renewcommand{\baselinestretch}{1.3}
\setlength{\parskip}{1em}
\newtheorem*{definition}{Definition}
\newtheorem{lemma}{Lemma}
\newtheorem{theorem}[lemma]{Theorem}
\newcommand*\conj[1]{\overline{#1}}
\newcommand*\abs[1]{\left|#1\right|}
\newcommand*\norm[1]{\abs{\abs{#1}}}
\begin{document}
\maketitle
\tableofcontents

\section{Discrete Fourier Analysis and Roth's Theorem}
Let $N\in\mathbb{N}$, $\omega = e^{\frac{2\pi i}{N}}$. Write $\mathbb{Z}_N$ for the cyclic group of integers mod $N$. Use the notation $\mathbb{E}_xf(x)$ to stand for the average $N^{-1}\sum_{x\in\mathbb{Z}_N}f(x)$.

\begin{definition}[Discrete Fourier Transform]
	Given a function $f: \mathbb{Z}_N \to \mathbb{C}$, define its \textbf{discrete Fourier transform} $\hat{f}$ by the formula $$\hat{f}(r)=\mathbb{E}_xf(x)\omega^{-rx}$$
\end{definition}

\begin{definition}[Convolution]
	We define the \textbf{convolution} $f * g$ of $f$ and $g$ by $$f*g(x) = \mathbb{E}_{y+z=x}f(y)g(z)$$
	$$\hat{f}*\hat{g}(r) = \sum_{s+t=r}\hat{f}(s)\hat{g}(t)$$
\end{definition}

We also define two inner products
$$\langle f,g \rangle = \mathbb{E}_xf(x)\conj{g(x)}$$
$$\langle \hat{f}, \hat{g} \rangle = \sum_r \hat{f}(r)\conj{\hat{g}(r)}$$

Have the following basic properties:
\begin{enumerate}
	\item Parseval's Identity: $$\langle \hat{f}, \hat{g} \rangle = \langle f, g \rangle$$ for any $f, g: \mathbb{Z}_N \to \mathbb{C}$.
	
	\item Convolution Law: for any $f,g:\mathbb{Z}_N \to \mathbb{C}$, $r \in \mathbb{Z}_N$ $$\widehat{f*g}(r) = \hat{f}(r)\hat{g}(r)$$
	
	\item Inversion Formula: let $f:\mathbb{Z}_N\to\mathbb{C}$. Then $$f(x)=\sum_r \hat{f}(r)\omega^{rx}$$
	
	\item Dilation Rule: let $a$ be invertible mod $N$ and define $f_a(x)$ to be $f(a^{-1}x)$. Then $$\hat{f_a(r)} = \hat{f}(ar)$$
\end{enumerate}

If $A \subset \mathbb{Z}_N$, we shall write $A(x)$ for $\left\{\begin{tabular}{r} $1 \quad x \in A$ \\ $0 \quad x \not\in A$\end{tabular}\right.$. If $\abs{A} = \alpha N$, then $\hat{A}(0)=\mathbb{E}_xA(x)=\alpha$.

We shall define $\norm{f}_p$ to be $(\mathbb{E}_x\abs{f(x)}^p)^{\frac{1}{p}}$ and $\norm{\hat{f}}_p$ to be $\left(\sum_{r}\abs{\hat{f}(x)}^p\right)^{\frac{1}{p}}$.

Then if $A \subset \mathbb{Z}_N$, $\norm{A}_2^2 = \langle A,A\rangle = \alpha$. By Parseval, we get $$\sum_r \abs{\hat{A}(r)}^2 = \alpha\ \left(= \norm{\hat{A}}_2^2\right)$$

\begin{theorem}[Roth]
	For every $\delta > 0\ \exists N$ s.t. every subset $A \subset [N]$ of size at least $\delta N$ contains an arithmetic progression of length 3.
\end{theorem}

Broad strategy: a density increment argument.

The idea is to show that if $A$ has density $\alpha$ and contains no 3-AP then there is a reasonably long AP $P$ s.t. $\frac{\abs{A\cap P}}{\abs{P}}$ is significantly larger than $\alpha$. There we are either done or can pass to $P$ and start again with a larger density. Then repeat, and eventually, since $\alpha$ can't exceed 1, we must get a 3-AP.

In order to use Fourier analysis, we want to think of $A$ as a subset of $\mathbb{Z}_n$. For this purpose, define sets $B=C=A \cap \left[\frac{N}{3},\frac{2N}{3}\right]$, and observe that if $(x, y, z)$ is an AP in $A \times B \times C$ in $\mathbb{Z}_N$, then it also is in $[N]$.

Let $\alpha$ be the density of $A$. Assume that $N$ is odd. If $\abs{B} < \frac{\alpha N}{5}$ then one of $\abs{A \cap \left[1, \frac{N}{3}\right]}$ and $\abs{A \cap \left[\frac{2N}{3}, N\right]}$ is at least $\frac{2\alpha N}{5}$, so we get an interval in which $A$ has density at least $\frac{6\alpha}{5}$, which is a very healthy density increment.

Otherwise, $\abs{B}=\abs{C} > \frac{\alpha N}{5}$, so let's assume that.

Define the \textbf{3-AP-density} of $(A, B, C)$ to be $\mathbb{E}_{x+z=2y}A(x)B(y)C(z)$. This is the probability that a random $(x, y, z)$ with $x+z=2y$ lies in $A \times B \times C$.
\begin{align*}
	\mathbb{E}_{x+z=2y}A(x)B(y)C(z) & = \mathbb{E}_u\left(\mathbb{E}_{x+z=u}A(x)C(z)\right)B(u/2)\\
	&= \mathbb{E}_u A*C(u)B_2(u) \\
	&= \langle A*C, B_2 \rangle \\
	&= \langle \widehat{A*C}, \hat{B_2} \rangle \\
	&= \langle \hat{A}\hat{C}, \hat{B_2} \rangle \\
	&= \sum_r\hat{A}(r)\hat{C}(r)\conj{\hat{B_2}(r)} \\
	&= \sum_r\hat{A}(r)\hat{C}(r)\hat{B}(-2r) \\
	&= \alpha\beta\gamma + \sum_{r\neq 0}\hat{A}(r)\hat{C}(r)\hat{B}(-2r)
\end{align*}
where $\beta=\gamma=$ density of $B$ (or $C$). Now
\begin{align*}
	\abs{\sum_{r\neq 0}\hat{A}(r)\hat{B}(-2r)\hat{C}(r)} &\leq \max_{r\neq 0}\abs{\hat{A}(r)}\sum_r\hat{B}(-2r)\hat{C}(r)\\
	&\leq \max_{r \neq 0}\abs{\hat{A}(r)}\norm{\hat{B}}_2\norm{\hat{C}}_2 \text{ (Cauchy-Schwarz)}\\
	&= \beta^{\frac{1}{2}}\gamma^{\frac{1}{2}}\max_{r \neq 0}\abs{\hat{A}(r)}
\end{align*}
Therefore, if $\max_{r \neq 0}\abs{\hat{A}(r)}\beta^{\frac{1}{2}}\gamma^{\frac{1}{2}} \leq \frac{\alpha\beta\gamma}{2}$, i.e. $\max_{r \neq 0}\abs{\hat{A}(r)} \leq \frac{1}{2}\alpha(\beta\gamma)^{\frac{1}{2}}$ then the 3-AP-density of $(A, B, C)$ is at least $\frac{\alpha\beta\gamma}{2}$. Since $\beta\gamma \geq \frac{\alpha^2}{25}$, this tells us that we get 3-APs provided $\max_{r \neq 0}\abs{\hat{A}(r)} \leq \frac{\alpha^2}{10}$ and $\frac{\alpha^3}{50} > \frac{1}{N}$ (ensures that the progression is non-trivial). So we may assume that $\exists r$ s.t. $\abs{\hat{A}(r)} \geq \frac{\alpha^2}{10}$.

\begin{lemma}
	Let $\epsilon > 0$ and let $r \in \mathbb{Z}_N$. Then the set $[N]$ can be partitioned into arithmetic progressions of length at least $\frac{\epsilon}{8\pi}N^{\frac{1}{2}}$ on each of which the function $x \mapsto \omega^{rx}$ varies by at most $\epsilon$.
\end{lemma}

\begin{proof}
	Let $m=\lfloor N^{\frac{1}{2}}\rfloor$. Of the numbers $1,\, \omega^r,\dots,\, \omega ^{mr}$ there must be two, say $\omega^{ur}$ and $\omega^{vr}$ with $u<v$, that differ by at most $\frac{2\pi}{m}$.
	
	Let $t=v-u$ and note that $\abs{\omega^{ur}-\omega^{vr}} = \abs{1-\omega^{tr}}$, so $\abs{1-\omega^{tr}} \leq \frac{2\pi}{m}$.
	
	Note also that if $a < b$, then
	\begin{align*}
		\abs{\omega^{btr}-\omega^{atr}} & \leq \sum_{j=1}^{b-a}\abs{\omega^{(a+j)tr}-\omega^{(a+j-1)tr}}\\
		&\leq(b-a)\frac{2\pi}{m}
	\end{align*}
	by the triangle inquality.
	
	Now partition $[N]$ into congruence classes mod $t$, and partition each congruence class into `intervals' of length at most $\frac{\epsilon m}{2\pi}$ and at least $\frac{\epsilon m}{4\pi}$. This is possible, since $t \leq m \leq \sqrt{N}$ (exercise). These progressions do the job, since $\frac{\epsilon m}{4\pi} \geq \frac{\epsilon N^{\frac{1}{2}}}{8\pi}$.
\end{proof}
\end{document}