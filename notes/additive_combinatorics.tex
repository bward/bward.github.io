\documentclass[a4paper]{article}
\title{Part III Topics in Additive Combinatorics}
\author{Based on lectures by Prof W.T. Gowers}
\date{Michaelmas 2016\\University of Cambridge}
\usepackage{mathtools}
\usepackage{amsthm}
\usepackage{amssymb}
\usepackage{textcomp}
\usepackage{enumerate}
\usepackage{graphicx}
\usepackage{tikz-cd}
\newtheorem*{definition}{Definition}
\newtheorem{lemma}{Lemma}
\newcommand*\conj[1]{\overline{#1}}
\begin{document}
\maketitle
\tableofcontents

\section{Discrete Fourier Analysis and Roth's Theorem}
Let $N\in\mathbb{N}$, $\omega = e^{\frac{2\pi i}{N}}$. Write $\mathbb{Z}_N$ for the cyclic group of integers mod $N$. Use the notation $\mathbb{E}_xf(x)$ to stand for the average $N^{-1}\sum_{x\in\mathbb{Z}_N}f(x)$.

\begin{definition}[Discrete Fourier Transform]
	Given a function $f: \mathbb{Z}_N \to \mathbb{C}$, define its \textbf{discrete Fourier transform} $\hat{f}$ by the formula $$\hat{f}(r)=\mathbb{E}_xf(x)\omega^{-rx}$$
\end{definition}

\begin{definition}[Convolution]
	We define the \textbf{convolution} $f * g$ of $f$ and $g$ by $$f*g(x) = \mathbb{E}_{y+z=x}f(y)g(z)$$
	$$\hat{f}*\hat{g}(r) = \sum_{s+t=r}\hat{f}(s)\hat{g}(t)$$
\end{definition}

We also define two inner products
$$\langle f,g \rangle = \mathbb{E}_xf(x)\conj{g(x)}$$
$$\langle \hat{f}, \hat{g} \rangle = \sum_r \hat{f}(r)\conj{\hat{g}(r)}$$

Have the following basic properties:
\begin{enumerate}
	\item Parseval's Identity: $$\langle \hat{f}, \hat{g} \rangle = \langle f, g \rangle$$ for any $f, g: \mathbb{Z}_N \to \mathbb{C}$.
	
	\item Convolution Law: for any $f,g:\mathbb{Z}_N \to \mathbb{C}$, $r \in \mathbb{Z}_N$ $$\widehat{f*g}(r) = \hat{f}(r)\hat{g}(r)$$
	
	\item Inversion Formula: let $f:\mathbb{Z}_N\to\mathbb{C}$. Then $$f(x)=\sum_r \hat{f}(r)\omega^{rx}$$
	
	\item Dilation Rule: let $a$ be invertible mod $N$ and define $f_a(x)$ to be $f(a^{-1}x)$. Then $$\hat{f_a(r)} = \hat{f}(ar)$$
\end{enumerate}
\end{document}