\documentclass[a4paper]{article}
\title{Part III Topics in Additive Combinatorics}
\author{Based on lectures by Prof W.T. Gowers}
\date{Michaelmas 2016\\University of Cambridge}
\usepackage{mathtools}
\usepackage{amsthm}
\usepackage{amssymb}
\usepackage{textcomp}
\usepackage{enumerate}
\usepackage{graphicx}
\usepackage{tikz-cd}
\renewcommand{\baselinestretch}{1.3}
\setlength{\parskip}{1em}
\newtheorem*{definition}{Definition}
\newtheorem{lemma}{Lemma}
\newtheorem{theorem}[lemma]{Theorem}
\newtheorem{corollary}[lemma]{Corollary}
\newcommand*\conj[1]{\overline{#1}}
\newcommand*\abs[1]{\left|#1\right|}
\newcommand*\norm[1]{\abs{\abs{#1}}}
\DeclareMathOperator{\im}{Im}
\begin{document}
\maketitle
\tableofcontents

\section{Discrete Fourier Analysis and Roth's Theorem}
Let $N\in\mathbb{N}$, $\omega = e^{\frac{2\pi i}{N}}$. Write $\mathbb{Z}_N$ for the cyclic group of integers mod $N$. Use the notation $\mathbb{E}_xf(x)$ to stand for the average $N^{-1}\sum_{x\in\mathbb{Z}_N}f(x)$.

\begin{definition}[Discrete Fourier Transform]
	Given a function $f: \mathbb{Z}_N \to \mathbb{C}$, define its \textbf{discrete Fourier transform} $\hat{f}$ by the formula $$\hat{f}(r)=\mathbb{E}_xf(x)\omega^{-rx}$$
\end{definition}

\begin{definition}[Convolution]
	We define the \textbf{convolution} $f * g$ of $f$ and $g$ by $$f*g(x) = \mathbb{E}_{y+z=x}f(y)g(z)$$
	$$\hat{f}*\hat{g}(r) = \sum_{s+t=r}\hat{f}(s)\hat{g}(t)$$
\end{definition}

We also define two inner products
$$\langle f,g \rangle = \mathbb{E}_xf(x)\conj{g(x)}$$
$$\langle \hat{f}, \hat{g} \rangle = \sum_r \hat{f}(r)\conj{\hat{g}(r)}$$

Have the following basic properties:
\begin{enumerate}
	\item Parseval's Identity: $$\langle \hat{f}, \hat{g} \rangle = \langle f, g \rangle$$ for any $f, g: \mathbb{Z}_N \to \mathbb{C}$.
	
	\item Convolution Law: for any $f,g:\mathbb{Z}_N \to \mathbb{C}$, $r \in \mathbb{Z}_N$ $$\widehat{f*g}(r) = \hat{f}(r)\hat{g}(r)$$
	
	\item Inversion Formula: let $f:\mathbb{Z}_N\to\mathbb{C}$. Then $$f(x)=\sum_r \hat{f}(r)\omega^{rx}$$
	
	\item Dilation Rule: let $a$ be invertible mod $N$ and define $f_a(x)$ to be $f(a^{-1}x)$. Then $$\hat{f_a(r)} = \hat{f}(ar)$$
\end{enumerate}

If $A \subset \mathbb{Z}_N$, we shall write $A(x)$ for $\left\{\begin{tabular}{r} $1 \quad x \in A$ \\ $0 \quad x \not\in A$\end{tabular}\right.$. If $\abs{A} = \alpha N$, then $\hat{A}(0)=\mathbb{E}_xA(x)=\alpha$.

We shall define $\norm{f}_p$ to be $(\mathbb{E}_x\abs{f(x)}^p)^{\frac{1}{p}}$ and $\norm{\hat{f}}_p$ to be $\left(\sum_{r}\abs{\hat{f}(x)}^p\right)^{\frac{1}{p}}$.

Then if $A \subset \mathbb{Z}_N$, $\norm{A}_2^2 = \langle A,A\rangle = \alpha$. By Parseval, we get $$\sum_r \abs{\hat{A}(r)}^2 = \alpha\ \left(= \norm{\hat{A}}_2^2\right)$$

\begin{theorem}[Roth]
	For every $\delta > 0\ \exists N$ s.t. every subset $A \subset [N]$ of size at least $\delta N$ contains an arithmetic progression of length 3.
\end{theorem}

Broad strategy: a density increment argument.

The idea is to show that if $A$ has density $\alpha$ and contains no 3-AP then there is a reasonably long AP $P$ s.t. $\frac{\abs{A\cap P}}{\abs{P}}$ is significantly larger than $\alpha$. There we are either done or can pass to $P$ and start again with a larger density. Then repeat, and eventually, since $\alpha$ can't exceed 1, we must get a 3-AP.

In order to use Fourier analysis, we want to think of $A$ as a subset of $\mathbb{Z}_n$. For this purpose, define sets $B=C=A \cap \left[\frac{N}{3},\frac{2N}{3}\right]$, and observe that if $(x, y, z)$ is an AP in $A \times B \times C$ in $\mathbb{Z}_N$, then it also is in $[N]$.

Let $\alpha$ be the density of $A$. Assume that $N$ is odd. If $\abs{B} < \frac{\alpha N}{5}$ then one of $\abs{A \cap \left[1, \frac{N}{3}\right]}$ and $\abs{A \cap \left[\frac{2N}{3}, N\right]}$ is at least $\frac{2\alpha N}{5}$, so we get an interval in which $A$ has density at least $\frac{6\alpha}{5}$, which is a very healthy density increment.

Otherwise, $\abs{B}=\abs{C} > \frac{\alpha N}{5}$, so let's assume that.

Define the \textbf{3-AP-density} of $(A, B, C)$ to be $\mathbb{E}_{x+z=2y}A(x)B(y)C(z)$. This is the probability that a random $(x, y, z)$ with $x+z=2y$ lies in $A \times B \times C$.
\begin{align*}
	\mathbb{E}_{x+z=2y}A(x)B(y)C(z) & = \mathbb{E}_u\left(\mathbb{E}_{x+z=u}A(x)C(z)\right)B(u/2)\\
	&= \mathbb{E}_u A*C(u)B_2(u) \\
	&= \langle A*C, B_2 \rangle \\
	&= \langle \widehat{A*C}, \hat{B_2} \rangle \\
	&= \langle \hat{A}\hat{C}, \hat{B_2} \rangle \\
	&= \sum_r\hat{A}(r)\hat{C}(r)\conj{\hat{B_2}(r)} \\
	&= \sum_r\hat{A}(r)\hat{C}(r)\hat{B}(-2r) \\
	&= \alpha\beta\gamma + \sum_{r\neq 0}\hat{A}(r)\hat{C}(r)\hat{B}(-2r)
\end{align*}
where $\beta=\gamma=$ density of $B$ (or $C$). Now
\begin{align*}
	\abs{\sum_{r\neq 0}\hat{A}(r)\hat{B}(-2r)\hat{C}(r)} &\leq \max_{r\neq 0}\abs{\hat{A}(r)}\sum_r\hat{B}(-2r)\hat{C}(r)\\
	&\leq \max_{r \neq 0}\abs{\hat{A}(r)}\norm{\hat{B}}_2\norm{\hat{C}}_2 \text{ (Cauchy-Schwarz)}\\
	&= \beta^{\frac{1}{2}}\gamma^{\frac{1}{2}}\max_{r \neq 0}\abs{\hat{A}(r)}
\end{align*}
Therefore, if $\max_{r \neq 0}\abs{\hat{A}(r)}\beta^{\frac{1}{2}}\gamma^{\frac{1}{2}} \leq \frac{\alpha\beta\gamma}{2}$, i.e. $\max_{r \neq 0}\abs{\hat{A}(r)} \leq \frac{1}{2}\alpha(\beta\gamma)^{\frac{1}{2}}$ then the 3-AP-density of $(A, B, C)$ is at least $\frac{\alpha\beta\gamma}{2}$. Since $\beta\gamma \geq \frac{\alpha^2}{25}$, this tells us that we get 3-APs provided $\max_{r \neq 0}\abs{\hat{A}(r)} \leq \frac{\alpha^2}{10}$ and $\frac{\alpha^3}{50} > \frac{1}{N}$ (ensures that the progression is non-trivial). So we may assume that $\exists r$ s.t. $\abs{\hat{A}(r)} \geq \frac{\alpha^2}{10}$.

\begin{lemma}
	Let $\epsilon > 0$ and let $r \in \mathbb{Z}_N$. Then the set $[N]$ can be partitioned into arithmetic progressions of length at least $\frac{\epsilon}{8\pi}N^{\frac{1}{2}}$ on each of which the function $x \mapsto \omega^{rx}$ varies by at most $\epsilon$.
\end{lemma}

\begin{proof}
	Let $m=\lfloor N^{\frac{1}{2}}\rfloor$. Of the numbers $1,\, \omega^r,\dots,\, \omega ^{mr}$ there must be two, say $\omega^{ur}$ and $\omega^{vr}$ with $u<v$, that differ by at most $\frac{2\pi}{m}$.
	
	Let $t=v-u$ and note that $\abs{\omega^{ur}-\omega^{vr}} = \abs{1-\omega^{tr}}$, so $\abs{1-\omega^{tr}} \leq \frac{2\pi}{m}$.
	
	Note also that if $a < b$, then
	\begin{align*}
		\abs{\omega^{btr}-\omega^{atr}} & \leq \sum_{j=1}^{b-a}\abs{\omega^{(a+j)tr}-\omega^{(a+j-1)tr}}\\
		&\leq(b-a)\frac{2\pi}{m}
	\end{align*}
	by the triangle inquality.
	
	Now partition $[N]$ into congruence classes mod $t$, and partition each congruence class into `intervals' of length at most $\frac{\epsilon m}{2\pi}$ and at least $\frac{\epsilon m}{4\pi}$. This is possible, since $t \leq m \leq \sqrt{N}$ (exercise). These progressions do the job, since $\frac{\epsilon m}{4\pi} \geq \frac{\epsilon N^{\frac{1}{2}}}{8\pi}$.
\end{proof}

The \textbf{balanced function} $f$ of $A$ is defined by $f(x) = A(x) - \alpha$. Note that $\mathbb{E}_xf(x)=0$ and $\hat{f}(r)=\hat{A}(r)$ when $r \neq 0$.

Let $r \neq 0$ be such that $\abs{\hat{f}(r)} \geq \frac{\alpha^2}{10}$. Then
\begin{align*}
	\frac{a^2}{10} &\leq \abs{\hat{f}(r)} \\
	&= \abs{\mathbb{E}_xf(x)\omega^{-rx}} \\
	&= N^{-1}\abs{\sum_xf(x)\omega^{-rx}}
\end{align*}

Now let $\epsilon = \frac{\alpha^2}{20}$ and let $P_1, \dots, P_m$ be given by Lemma 2.
\begin{align*}
	N^{-1}\abs{\sum_xf(x)\omega^{-rx}} &\leq N^{-1}\sum_i\abs{\sum_{x \in P_i}f(x)\omega^{-rx}}\\
	&\leq N^{-1}\sum_i\abs{\sum_{x \in P_i} f(x)(\omega^{-rx}-\omega^{-rx_i})} + N^{-1}\sum_i\abs{\sum_{x \in P_i}f(x)\omega^{-rx_i}} \\
	&\text{where $x_i \in P_i$ is arbitrary}\\
	&\leq \frac{\alpha^2}{20} + N^{-1}\sum_i\abs{\sum_{x \in P_i} f(x)}
\end{align*}

So we may conclude that $\sum_i \abs{\sum_{x \in P_i} f(x)} \geq \frac{\alpha^2}{20}N$. Also, $\sum_i \sum_{x \in P_i} f(x) = 0$. Therefore, $\sum_i\left(\abs{\sum_{x \in P_i}f(x)} + \sum_{x \in P_i}f(x)\right) \geq \frac{\alpha^2}{20}N$.

So $\exists i$ s.t. $\abs{\sum_{x \in P_i} f(x)} + \sum_{x \in P_i} f(x) \geq \frac{\alpha^2 \abs{P_i}}{20}$, which implies that $$\sum_{x\in P_i} f(x) \geq \frac{\alpha^2}{40}\abs{P_i}$$

Or equivalently, $\abs{A \cap P_i} \geq \left(\alpha + \frac{\alpha^2}{40} \right) \abs{P_i}$.

Back of envelope calculation: each time we iterate, $\alpha$ goes to at least $\alpha + \frac{\alpha^2}{40}$, so after $\frac{40}{\alpha}$ iterations, the density at least doubles. So the total number of iterations (before we get a 3-AP) is at most $\frac{40}{\alpha} + \frac{40}{2\alpha} + \frac{40}{4\alpha} + \dots = \frac{80}{\alpha}$.

Each time we iterate, $N$ goes to $\frac{\alpha^2}{20}\frac{N^{\frac{1}{2}}}{8\pi}$, so as long as $N \geq ??$ this is at least $N^{\frac{1}{3}}$. So all the iterative processes have that the new $N$ is at least $N^{(\frac{1}{3})^{\frac{80}{\alpha}}}$, which we need to be greater than $\frac{50}{\alpha^3}$.

To solve $N^{(\frac{1}{3})^{\frac{80}{\alpha}}} > \frac{50}{\alpha^3}$ take logs twice.
\begin{align*}
	\left(\frac{1}{3}\right)^{\frac{80}{\alpha}} \log N &> \log 50 + \log(\alpha^{-3}) \\
	\implies \frac{80}{\alpha}\log(\frac{1}{3}) + \log\log N &> \log(\log 50 + \log(\alpha^{-3}))
\end{align*}
So for an appropriate constant $C$, we are done if $$\log\log N \geq \frac{C}{\alpha} \text{, or } \alpha \geq \frac{C}{\log\log N}$$

\begin{theorem}[Behrend, 1947]
	For every $N$ there exists a subset $A \subset [N]$ of size $\frac{N}{e^{c\sqrt{\log N}}}$ that contains no 3-AP.
\end{theorem}
\begin{proof}
	For this proof let $[N]$ mean $\{0,1,\dots,N-1\}$.
	
	Let $m, d$ be positive integers and consider the grid $[m]^{d}$. Note that in $\mathbb{R}^d$, no sphere $\{x: x_1^2 + \dots + x_d^2 =t \}$ contains three distinct points $x$, $y$, $z$ with $x+z=2y$.
	
	But on $[m]^d$, $x_1^2+ \dots + x_d ^2$ takes at most $dm^2$ different values. Therefore, we can find a sphere that intersects $[m]^d$ in at least $\frac{m^d}{m^2d}$ points.
	
	Let $\phi: [m]^d \to [(2m)^d]$ be defined by $$\phi(x) = x_1 + 2mx_2 + (2m)^2x_3 + \dots + (2m)^{d-1}x_d$$
	
	So $\phi$ sends $(x_1, \dots, x_d)$ to the integer with base-$2m$ representation $x_dx_{d-1}\dots x_1$.
	
	If we add $\phi(x)$ and $\phi(y)$ then no carrying takes place base-$2m$ since all digits are $<m$. So if $\phi(x) + \phi(z) = 2\phi(y)$ it follows that $x+z=2y$, i.e. no new 3-APs are created.
	
	So (ignoring divisibility etc.) we can find a subset of $[(2m)^d]$ of size $\frac{m^d}{m^2d}$ that contains no 3-AP. If we let $N=(2m)^d$, then $m=\frac{N^{\frac{1}{d}}}{2}$ and $\frac{m^d}{m^2d} = \frac{4N}{2^dN^\frac{2}{d}d}$. So we'd like to minimise $2^dN^\frac{2}{d}d$.
	
	Take logs: $d\log 2 + \frac{2}{d}\log N + \log d$, so $d = \sqrt{\log N}$ is a pretty good choice. So we get $$\frac{N}{2^{\sqrt{\log N}}e^{2 \sqrt{\log N}}\sqrt{\log N}} \geq \frac{N}{e^{c \sqrt{ \log N}}}$$
\end{proof}

\section{Bohr Sets and Boglyubov's Method}
\setcounter{lemma}{0}
\begin{definition}[Bohr set]
	Let $K\subset \mathbb{Z}_N$ and let $\epsilon > 0$. The \textbf{Bohr set} $B(K, \epsilon)$ is defined to be $$\{x \in \mathbb{Z}_N \,|\, \abs{1-\omega^{rx}} \leq \epsilon\ \forall r \in K \}$$
\end{definition}

\begin{definition}[Sumset]
	Let $A$ be a subset of an abelian group $G$. The \textbf{sumset} $A+A$ is $\{x+y \,|\, x, y \in A \}$. The \textbf{difference set} $A-A$ is $\{x-y \,|\, x, y \in A \}$. More generally, $\pm A_1 \pm A_2 \pm \dots \pm A_k = \{\pm x_1 \pm \dots \pm x_k \,|\, x_i \in A_i\}$. We write $rA$ for $A+A+\dots+A$ ($r$ times).
\end{definition}

\begin{lemma}[Boglyubov's method]
	Let $A \in \mathbb{Z}_N$ be a subset of density $\alpha$. Then $2A-2A$ contains a Bohr set $B(K, \sqrt{2})$ with $\abs{K} \leq \alpha^{-2}$.
\end{lemma}
\begin{proof}
	Let $K = \left\{r: \abs{\hat{A}(r)} \geq \alpha^{\frac{3}{2}} \right\}$. Observe that $x \in 2A-2A \iff A * A * (-A) * (-A)(x) \neq 0$ (i.e. $\mathbb{E}_{a+b-c-d=x}A(a)A(b)A(c)A(d) \neq 0$).
	
	But
	\begin{align*}
	A*A*(-A)*(-A)(x) &= \sum_r \widehat{A*A*(-A)*(-A)(r)\omega^{rx}} \text{ (inversion)} \\
	&= \sum_r \abs{\hat{A}(r)}^4 \omega^{rx} \text{ (convolution)} \\
	&= \alpha^4 + \sum_{r \in K, r \neq 0} \abs{\hat{A}(r)}^4 \omega^{rx} + \sum_{r \not\in K} \abs{\hat{A}(r)}^4 \omega^{rx}
	\end{align*}
	
	for each $x \in B(K, \sqrt{2})$ and each $r \in K$, $\Re(\omega^{rx}) \geq 0$. So if $x \in B(K, \sqrt{2})$, then the second term has real part $\geq 0$.
	
	Also,
	\begin{align*}
		\abs{\sum_{r \not\in K} \abs{\hat{A}(r)}^4 \omega^{rx}} &\leq \sum_{r \not\in K} \abs{\hat{A}(r)}^4 \\
		&\leq \max_{r \not\in K}\abs{\hat{A}(r)}^2 \sum_{r \not\in K} \abs{\hat{A}(r)}^2\\
		&< \alpha^3 \cdot \alpha = \alpha^4
	\end{align*}
	
	It follows that the sum is not 0. So $B(Km \sqrt{2})$ does the job. Note also that $\alpha^3 \abs{K} \leq \sum_r\abs{\hat{A}(r)}^2 = \alpha$, so $\abs{K} \leq \alpha^{-2}$, as claimed.
\end{proof}

\begin{lemma}
	Writing $B[K, \delta]$ for $\{x \in \mathbb{Z}_N \,|\, \forall r \in K,\, rx \in [-\delta N, \delta N] \mod N \}$, we have that $B[K, \delta]$ has density at least $\delta^k$, where $k = \abs{K}$.
\end{lemma}

\begin{proof}
	Let $K=\{r_1,\dots,r_k\}$ and assume (wlog) that $0 \not\in K$. Define $\phi: \mathbb{Z}_N \to \mathbb{Z}_N^k$ by $\phi: x \mapsto (r_1x,\dots,r_kx)$.
	
	Pick a random translate $\boldsymbol{u} + [\delta N]^k$ of $[\delta N]^k$. On average, this contains at least $\delta^k N$ points of $\im \phi$.
	
	It follows that there is some translate $\boldsymbol{u} + [\delta N]^k$ that contains at least $\delta^k N$ points of $\im \phi$. But if $\phi(x), \phi(y) \in \boldsymbol{u} + [\delta N]^k$, then $\phi(x-y) = \phi(x)-\phi(y) \in [-\delta N, \delta N]^k$.
	
	$\implies x-y \in B[K, \delta]$. There are at least $\delta^k N$ distinct such $x-y$.
\end{proof}

\begin{corollary}
	The Bohr set $B[K, \delta]$ contains an AP (mod $N$) of length at least $\delta N ^{\frac{1}{\abs{K}}}$
\end{corollary}
\begin{proof}
	By Lemma 2, $\abs{B[K, \theta]} \geq \theta^{\abs{K}}N$, so if $\theta > N^{-\frac{1}{\abs{K}}}$ then $\abs{B[K, \theta]} > 1$. By compactness there is some non-zero $x \in B[K, N^{-\frac{1}{\abs{K}}}]$. Then for any $m$ we have that $mx \in B[K, \abs{m}N^{-\frac{1}{\abs{K}}}]$, so $mx \in B[K, \delta]$ whenever $\abs{m} \leq \delta N^{\frac{1}{\abs{K}}}$, which proves the result.
\end{proof}

\begin{definition}
	Let $A$ and $B$ be subsets of abelian groups. A map $\phi:A \to B$ is a \textbf{Freiman homomorphism of order k} if
	\begin{align*}
		a_1 + a_2 + \dots + a_k &= a_{k+1} + \dots + a_{2k} \text{ (all $a_i \in A$)} \\
		\implies \phi(a_1) + \phi(a_2) + \dots + \phi(a_k) &= \phi(a_{k+1}) + \dots + \phi(a_{2k})
	\end{align*}
	
	It is a \textbf{Freiman isomorphism of order k} if it is a bijection and its inverse is also a Freiman homomorphism of order k (that is, the implication can be reversed).
\end{definition}

The case $k=2$ is particularly important. It says $$x+y = z+w \implies \phi(x)+ \phi(y) = \phi(z) = \phi(w)$$ or equivalently $$x-y=z-w \implies \phi(x)-\phi(y) = \phi(z)-\phi(w)$$
Freiman homomorphisms preserve 'additive structure'. Note in particular that Freiman isomorphisms preserve arithmetic progressions.
\end{document}